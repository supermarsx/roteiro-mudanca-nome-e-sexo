% !TEX program = xelatex
%-------------
% MAIN CONFIG
%-------------
\documentclass[12pt,a4paper]{article}

%---------
% PACOTES
%---------
%---------
% PACOTES
%---------

\usepackage[utf8]{inputenc}
\usepackage[T1]{fontenc}
\usepackage{lmodern}

\usepackage{fontspec}

\usepackage{graphicx}
\usepackage{xcolor}
\usepackage{tikz}
\usepackage[absolute,overlay]{textpos}

\usepackage{tikz}
\usetikzlibrary{calc}
\usepackage[absolute,overlay]{textpos}


\usepackage{geometry}
\geometry{margin=2.5cm}

\usepackage{fancyhdr}
\usepackage[hyphens]{url}
\usepackage{xurl}

\usepackage{hyperref}

\usepackage{zref-totpages}

\usepackage{booktabs}
\usepackage{tcolorbox}
\usepackage{eso-pic}
\usepackage{atbegshi}

\usepackage{contour}

\usepackage{pagecolor}

%-----------
% METADADOS
%-----------
%-----------
% METADADOS
%-----------
\newcommand{\DocTitle}{Roteiro prático da mudança de nome e sexo}
\newcommand{\DocSubtitle}{Um guia completo para navegar a burocracia da mudança de nome e sexo em Portugal}
\newcommand{\DocAuthor}{Mariana Mota}
\newcommand{\DocRights}{Feito com <3 para o domínio público}
\newcommand{\DocDate}{Junho 2025}
\newcommand{\DocEdition}{1ª Edição - Tema escuro}
\newcommand{\DocPublication}{Manual de instruções técnicas e práticas}
\newcommand{\DocDimensions}{A4 Standard (21×29.7 cm)}
\newcommand{\DocLanguage}{Português}
\newcommand{\DocPages}{\ztotpages}
\newcommand{\DocBanner}{imagens/banner-transgenero-preto-branco.png}
\newcommand{\DocHeader}{imagens/banner-arcoiris-2-preto-branco.png}
\newcommand{\DocFeatured}{imagens/duas-pessoas-com-fundo-psicadelico-preto-branco.png}

\title{\DocTitle}
\author{\DocAuthor}
\date{\DocDate}

\definecolor{pagebg}{HTML}{222222}
\definecolor{textfg}{HTML}{FFFFFF}
\definecolor{framecolor}{HTML}{CCCCCC}
\definecolor{linkcolor}{HTML}{E3E3E3}

%------------------------
% CONFIGURAÇÕES DIVERSAS
%------------------------
%------------------------
% CONFIGURAÇÕES DIVERSAS
%------------------------


\pagecolor{pagebg}
\color{textfg}

%----------
% LIGAÇÕES
%----------

\hypersetup{
	colorlinks=true,
	linkcolor=linkcolor,
	urlcolor=linkcolor,
	citecolor=linkcolor,
	pdftitle={\DocTitle},
	pdfauthor={\DocAuthor}
}

%-----------
% CONTORNOS
%-----------

\definecolor{defaultContour}{rgb}{0.0196, 0.0431, 0.0314}
\contourlength{4pt}
\contournumber{1024}

%----------------
% TIPOS DE LETRA
%----------------
%----------------
% TIPOS DE LETRA
%----------------

% TIPO DE LETRA SERIF PRINCIPAL
\setmainfont[
Path = fontes/eb-garamond/static/,
Extension = .ttf,
UprightFont = eb-garamond-regular,
BoldFont    = eb-garamond-bold,
ItalicFont  = eb-garamond-italic
]{EBGaramond}

% TIPO DE LETRA SANS SERIF
\setsansfont[
Path = fontes/andada/,
Extension = .otf,
UprightFont = andada-regular,
BoldFont    = andada-bold,
ItalicFont  = andada-italic
]{Andada}

% TIPO DE LETRA MONOESPAÇO
\setmonofont[
Path = fontes/inconsolata/,
Extension = .ttf,
Scale = 0.80,
UprightFont = inconsolata-regular,
BoldFont    = inconsolata-700
]{Inconsolata}



%---------------------------
% TIPOS DE LETRA ADICIONAIS
%---------------------------
%---------------------------
% TIPOS DE LETRA ADICIONAIS
%---------------------------

% TIPO DE LETRA ROUSSEAU DECO
\newfontfamily \FontRousseau[
Path = fontes/rousseau-deco/,
Extension = .ttf,
UprightFont = rousseau-deco,
BoldFont    = rousseau-deco,
ItalicFont  = rousseau-deco
]{Rousseau Deco}

% TIPO DE LETRA ANDADA SC
\newfontfamily \FontAndadaSC[
Path = fontes/andada-sc/,
Extension = .otf,
UprightFont = andada-sc-regular,
BoldFont    = andada-sc-bold,
ItalicFont  = andada-sc-italic
]{Andada}


%----------------------
% MARCADORES DE SECÇÃO
%----------------------
%----------------------
% MARCADORES DE SECÇÃO
%----------------------

% CAPTURAR NÚMERO DA SECÇÃO E TÍTULO
\renewcommand{\sectionmark}[1]{%
	\markright{\FontAndadaSC\scriptsize\thesection { } #1}%
}

\renewcommand{\subsectionmark}[1]{}


%------------------------------------
% PARAMETRIZAÇÃO DO ESTILO DE PÁGINA
%------------------------------------
%------------------------------------
% PARAMETRIZAÇÃO DO ESTILO DE PÁGINA
%------------------------------------

\pagestyle{fancy}
\fancyhf{}
\fancyhf[HL]{\FontRousseau\small\DocTitle}
\fancyhf[HR]{\rightmark}
\fancyfoot[C]{%
	\tikz[baseline={(0,0.5ex)}]{%
		\node[
		draw=black,            % COR DA MOLDURA
		line width=4pt,        % ESPESSURA DA LINHA
		rounded corners=1pt,   % ARREDONDAMENTO
		fill=pagebg,            % COR DE FUNDO
		inner sep=5pt          % ENCHIMENTO
		] {\thepage};
	}%
}

\newlength\bannerheight
\setlength\bannerheight{52mm}  

\newlength\headerheight
\setlength\headerheight{30mm}

\AtBeginShipout{%
	\ifnum\value{page}>1\relax
	\AddToShipoutPictureBG{%
		\begin{tikzpicture}[remember picture,overlay]
			% RODAPÉ
			\node[
			anchor=south,
			xshift=-\hoffset,
			yshift=-\bannerheight
			] at (current page.south) {%
				\includegraphics[width=\paperwidth]{\DocBanner}%
			};
			% CABEÇALHO
			\node[
			anchor=north,
			xshift=-\hoffset,
			yshift=\headerheight
			] at (current page.north) {%
				\includegraphics[width=\paperwidth]{\DocHeader}%
			};
		\end{tikzpicture}%
	}%
	\else
	\AddToShipoutPictureBG{}% SEM BANDA NAS PÁGINAS 1 E 2
	\fi
}

%-----------
% CONTEÚDOS
%-----------
%-----------
% CONTEÚDOS
%-----------
\begin{document}
	
	%-----------------------
	% FALSA PÁGINA DE ROSTO
	%-----------------------
	%-----------------------
% FALSA PÁGINA DE ROSTO
%-----------------------
\begin{titlepage}
	\thispagestyle{empty}
	
	% IMAGEM DE FUNDO
	\begin{tikzpicture}[remember picture,overlay]
		\node at (current page.center)
		{\includegraphics[width=\paperwidth,height=\paperheight]{\DocFeatured}};
	\end{tikzpicture}
	
	
	% TÍTULO, SUBTÍTULO, AUTOR E DATA
	\vfill
	\begin{center}
		{\FontRousseau \Huge \color{white} \contour{defaultContour} \DocTitle \par}
		\vspace{10pt}
		{\FontAndadaSC \color{white} \fcolorbox{black}{black}  \DocSubtitle \par}
		\vspace{1cm}
		{\FontAndadaSC \small \color{white} \itshape \contour{defaultContour} \DocAuthor \par}
		\vspace{2cm}
		{\FontAndadaSC \large \color{white} \contour{defaultContour} \DocDate \par}
	\end{center}
	
	\newpage
\end{titlepage}
	
	%----------------
	% FOLHA DE ROSTO
	%----------------
	%----------------
% FOLHA DE ROSTO
%----------------
\begin{titlepage}
	\thispagestyle{empty}
	
	% TÍTULO, SUBTÍTULO, AUTOR E DATA
	\vfill
	\begin{center}
		{\FontRousseau \Huge \DocTitle \par}
		\vspace{10pt}
		{\FontAndadaSC \DocSubtitle \par}
		\vspace{1cm}
		{\FontAndadaSC \small \itshape \DocAuthor \par}
		\vspace{2cm}
		{\FontAndadaSC \large \DocDate \par}
	\end{center}
	
	\newpage
\end{titlepage}
	
	%---------------
	% FICHA TÉCNICA
	%---------------
	%---------------
% FICHA TÉCNICA
%---------------
\newpage

\phantomsection
\thispagestyle{empty}

\markright{}
\section*{Ficha Técnica}

\vspace{10pt}

% ELEMENTOS DA FICHA TÉCNICA
\textbf{Título} \\
\DocTitle \\[5pt]
\textbf{Subtítulo} \\
\DocSubtitle \\[5pt]
\textbf{Autora} \\
\DocAuthor \\[5pt]
\textbf{Direitos de autor} \\
\DocRights \\[5pt]
\textbf{Edição} \\
\DocEdition \\[5pt]
\textbf{Publicação} \\
\DocPublication \\[5pt]
\textbf{Páginas} \\
\DocPages \\[5pt]
\textbf{Dimensões} \\
\DocDimensions \\[5pt]
\textbf{Idioma} \\
\DocLanguage \\[5pt]
\textbf{Mês e Ano} \\
\DocDate \\[5pt]

	
	%--------
	% ÍNDICE
	%--------
	%---------------
% ÍNDICE
%---------------
\newpage

\phantomsection

\renewcommand{\contentsname}{Índice}

\markright{}

\tableofcontents
	
	%---------------------
	% NOTAS INTRODUTÓRIAS
	%---------------------
	%---------------------
% NOTAS INTRODUTÓRIAS
%---------------------

\newpage

\section{Notas Introdutórias}

\subsection{Introdução}

Este roteiro nasce da necessidade de \textbf{identificar os passos} a
tomar no \textbf{processo de alteração de nome e sexo} bem como as suas
ramificações numa base individual e personalizada. Este roteiro foi
elaborado a pensar nas pessoas que estão a \textbf{ponderar} efetuar a
mudança de nome e sexo bem como aquelas que \textbf{já o fizeram}, ou
até mesmo outras que já o fizeram mas \textbf{não completaram todas as 
	alterações} por \textbf{desconhecimento} ou até mesmo
\textbf{impossibilidade}. \\
\\
O roteiro é abrangente e pretende atender a um número alargado de
pessoas, ainda que o \textbf{principal público-alvo seja maiores de
	idade}, com algumas competências básicas e com quotidianos profundamente
estabelecidos, onde uma mudança de nome e sexo pode implicar uma
\textbf{sequência extensa de processos burocráticos} algo demorados e
inconvenientes. Ainda que nem todo o roteiro seja necessariamente
acessível a todos os tipos de público certamente poderá ajudar boa parte
deles. Existem alguns pressupostos sobre os quais o roteiro foi
estabelecido onde poderá existir lacunas informacionais para alguns tipo
público. \\
\\
Pretende-se desta forma criar um \textbf{suporte de relativa
	fiabilidade} que com base na experiência e aconselhamento prévio irá
facilitar e responder a muitas das \textbf{questões} que a pessoa que
pretenda mudar de nome e sexo poderá ter. É importante salientar que o
propósito central deste roteiro é \textbf{evidenciar os processos
	administrativos}, algumas questões ligadas ao direito, e
\textbf{respostas práticas} ao inverso de outros guias cujo foco é mais
universalista e incidente sobre as questões sociais, psicológicas,
médicas ou até mesmo políticas. \\
\\
Ainda que o roteiro tenha como objetivo ser bastante abrangente haverá
situações e circunstâncias específicas difíceis de cobrir numa base
comum, isto é, sem ser demasiado detalhista. Nesses cenários menos
incomuns ou invulgares a pessoa deverá recorrer ou fazer uso de recursos
alternativos para melhor suprir as suas necessidades específicas e
individuais. \textbf{Este roteiro não é um substituto de acompanhamento
	profissional especializado}. \\
\\
Este roteiro foi escrito de acordo com o \textbf{novo acordo
	ortográfico}.
	
\subsection{Limitação de responsabilidade}

É importante salientar que \textbf{poderão existir erros}, gralhas,
falhas de descrição processual, ou até mesmo falhas nos processos quer
por atualização desses processos quer por \textbf{simples lapso
	documental}. Dado o contexto de criação do roteiro, é impossível que
este acompanhe todas as atualizações futuras aos processos. Este roteiro
portanto \textbf{não é representativo de aconselhamento especializado}
no âmbito das questões do direito e o seu \textbf{conteúdo é apenas
	informativo}. Sendo o conteúdo apenas de caráter puramente
informacional, não poderá representar aconselhamento jurídico,
significando assim que não são dadas quaisquer garantias de qualquer
informação ao utilizador do roteiro. Fora do âmbito do roteiro e em caso
de dúvidas que possa vir a ter, \textbf{deverá recorrer a recursos de
	acompanhamento especializado} tais como advogados ou solicitadores para
esses efeitos. Outras dúvidas ou problemas que possa encontrar relativo
a outro foro deve então obter apoio externo nas respetiva área de
competência, como por exemplo, psicólogos para questões sociais.

\newpage

\subsection{Anexos}

Para que o roteiro ficasse o mais completo possível foram anexadas todas
as \textbf{bases}, \textbf{modelos} e \textbf{documentação relevante}. O
conjunto de documentos cobre todas as bases e fundamentos legais para as
mudanças que irá fazer, modelos com os quais poderá com maior facilidade
efetuar as alterações junto de empresas, entidades ou organizações. Os
anexos são fornecidos numa base de facilidade burocrática e
administrativa, não representando assim uma fonte absoluta de verdade ou
atualidade dos factos, poderão em certas circunstâncias existir versões
mais recentes dos documentos anexados. Cabe ao utilizador do roteiro
verificar a veracidade, atualidade e aplicabilidade dos anexos à sua
circunstância específica. \\
\\
\textbf{Lista de anexos} \\[4pt]
\\
\textbf{Anexos gerais}:
\begin{itemize}
	\item Anexo A - Destaques jurídicos de especial interesse, temática militar;
	\item Anexo B - Destaques jurídicos de especial interesse, temática de dados pessoais e legislação comunitária;
	\item Anexo C - Lista de nomes próprios aprovados pelo Instituto dos Registos e Notariado (IRN);
	\item Anexo D - Manual da aplicação móvel \emph{gov.pt};
	\item Anexo E - Condições Gerais de Emissão e Utilização do Cartão navegante;
	\item Anexo F - O documento eletrónico: suporte e formato, Ordem dos Advogados.
\end{itemize}
\leavevmode\\
\textbf{Formulários}:
\begin{itemize}
	\item Anexo A - Requerimento para mudança da menção do sexo e nome próprio, maiores de idade;
	\item Anexo B - Requerimento para mudança da menção do sexo e nome próprio, menores de idade;
	\item Anexo C - Requerimento de Registo Automóvel;
	\item Anexo D - Requisição de Passe Navegante (TML);
	\item Anexo E - Declaração de objeção de consciência perante o serviço militar;
	\item Anexo F - Declaração abonatória (Relativo ao estatuto de objetor de consciência);
	\item Anexo G - Formulário para a atualização de dados e consentimento para o tratamento de dados pessoais, cartão navegante;
	\item Anexo H - Formulário de exercício de direitos dos titulares de dados pessoais, cartão navegante.
\end{itemize}
\leavevmode\\

\newpage

\textbf{Modelos}:
\begin{itemize}
	\item Modelo de carta - Pedido de alteração/retificação de dados pessoais;
	\item Modelo de carta - Pedido de 2ª (segunda) via do Certificado de Habilitações;
	\item Modelo de carta - Pedido de 2ª (segunda) via de Relatório Médico ou Avaliação Psicológica;
	\item Modelo de carta - Reconhecimento de nome social escolar;
	\item Modelo de mensagem de correio eletrónico para o Dia da Defesa Nacional/Balcão Único da Defesa (DDN/BUD);
	\item Modelo de formulário de renúncia ao estatuto de objetor de consciência.
\end{itemize}
\leavevmode\\
\textbf{Suportes jurídicos}:
\begin{itemize}
	\item Carta dos direitos fundamentais da União Europeia n.º 2016/C 202/02;
	\item Código de Processo Civil (CPC), Lei n.º 41/2013, de 26 de junho;
	\item Código do Registo Civil (CRC), Decreto-Lei n.º 131/95, de 6 de junho;
	\item Constituição da República Portuguesa (CRP), Decreto de Aprovação da Constituição, de 10 de abril;
	\item Execução nacional do RGPD, Lei n.º 58/2019, de 8 de agosto;
	\item Direito à autodeterminação da identidade de género e expressão de género e à proteção das características sexuais de cada pessoa, Lei n.º 38/2018 de 7 de agosto;
	\item Lei do Serviço Militar (LSM), Lei n.º 174/99, de 21 de setembro;
	\item Procedimento de mudança de sexo, Lei n.º 7/2011, de 15 de março;
	\item Regulamento da Lei do Serviço Militar, Decreto-Lei n.º 289/2000, de 14 de novembro;
	\item Regulamento Geral da Proteção de Dados, Regulamento da União Europeia n.º 2016/679 de 27 de abril de 2016;
	\item Sistema alternativo e voluntário de autenticação dos cidadãos (\ldots) denominado Chave Móvel Digital, Lei n.º 37/2014, de 26 de junho;
	\item Alteração às Leis (\ldots) que cria o Cartão de Cidadão (\ldots) Chave Móvel Digital, e (\ldots) recenseamento eleitoral (\ldots), Lei n.º19-A/2024, de 7 de fevereiro;
	\item Regulamento eIDAS, identificação eletrónica, Regulamento da União Europeia 910/2014 de 23 de julho de 2014;
	\item Execução nacional do eIDAS, Decreto-Lei n.º 12/2021, de 9 de fevereiro.
\end{itemize}

\newpage

\subsection{Inteligência Artificial}
 
O roteiro foi composto recorrendo \textbf{ocasionalmente} a ferramentas
de inteligência artificial de modo a facilitar a composição e
organização dos conteúdos, no entanto, todos os textos foram finalizados
e revistos manualmente dadas as limitações dessas ferramentas. \\
\\
Foi também criado um GPT personalizado com base neste roteiro e todos os
seus anexos, tais como modelos e diplomas legislativos e regulamentares.
Esta poderá ser uma ferramenta útil para navegar os processos numa base
mais individual e específica. Há que apontar que estas ferramentas por
vezes não são 100\% fiáveis portanto deve confirmar as informações dadas
por esta. \\
\\
\textbf{Ligação de referência}:
\begin{itemize}
	\item ChatGPT, guia para a mudança de nome e sexo, \url{https://chatgpt.com/g/g-68504ffb04848191a5df58246b201561-guia-para-a-mudanca-de-nome-e-sexo}
\end{itemize}

\subsection{Afiliações e conflitos de interesse}

Nenhuma das referências a marcas, nomes, tecnologias e outras menções
específicas sujeitas ou relacionadas com direitos ou propriedade
intelectual publicamente reconhecíveis utilizadas neste roteiro
representam uma qualquer afiliação, incentivo, promoção ou de qualquer
outro modo um apelo à ação ou consumo de uma determinada referência, não
existindo assim nenhum motivo financeiro ou qualquer outro para a
influência do utilizador deste roteiro que não o auxílio na retificação
dos seus dados pessoais. Todas e quaisquer recomendações ou sugestões
são apenas representativas de um método ou forma de ação baseada numa ou
mais experiências individuais relatadas. O utilizador do roteiro é
responsável pelo seguimento ou utilização das referências externas
fornecidas sendo que os autores e distribuidores do roteiro não são
responsáveis pelos conteúdos de destino nas referências mencionadas. É
autodeclarada a inexistência de conflitos de interesse no prosseguimento
dos fins do roteiro.

\subsection{Contexto jurisdicional}

O roteiro foi pensado no contexto jurisdicional Português, sendo que a
sua aplicabilidade é circunscrita a Portugal, à exceção dos itens
explicitamente mencionados onde a aplicabilidade pode estender-se à UE.
A utilidade fora do contexto jurisdicional previamente entendido é
limitada e o roteiro não deve ser considerado nesses casos pela sua
razão inadequada ou incorreta.

\subsection{Agradecimentos}

Queria agradecer todo o apoio que recebi e a todos os que tornaram este
roteiro possível. É com uma grande felicidade que dou o meu especial
agradecimento à Rute que tudo fez para me apoiar neste caminho. Agradeço
também em ponto grande ao Bruno, ao João, e ao Emanuel, pessoas pelas
quais tenho elevada consideração, bem como à MUSSOC por ser um pilar de
apoio da comunidade com neutralidade e integridade. Agradeço também a
todos os que de alguma forma contribuíram para a elaboração do roteiro.
Sem vós este roteiro e a sua inspiração não seria possível, muito
obrigada.

\newpage

\subsection{Nota da autora}

Este roteiro inspira-se numa jornada difícil, que se aparentou fácil
inicialmente, onde não sabia para onde virar-me com tanta coisa que era
preciso alterar e verificar. Senti-me completamente perdida, como é que
iria tratar dos mais de 50 (cinquenta) itens, locais e demais papeladas.
A minha pesquisa inicial deu pouco frutos, retornou apenas o básico,
aquilo que é óbvio e estritamente necessário mudar mas esquece-se
completamente do resto, algumas delas importantes mas escondidas. A
minha vida é feita de mais que um Cartão de Cidadão, é conduzir, viajar,
trabalhar, é o quer que seja que eu queira. O facto? Nenhum guia
mostrou-me a realidade disto e o lado prático do que é mesmo mudar algo
tão ``simples'' como o nome e o sexo na identificação. Isto é um guia
completo, estruturado q.b. de tudo aquilo que é possível fazer para
tratar de todos os registos que precisa ou precisará de alterar. É um
contributo simbólico para as pessoas que estão a embarcar nesta jornada
saberem ao que vêm e ter um guia com um bom grau de solidez e confiança.
Neste caminho consultei advogados, psicólogos e amigos que de alguma
forma tentaram ajudar a completar e organizar a informação. Criei
modelos para facilitar os processos e as comunicações. É importante
notar que sou apenas profissional das tecnologias de informação e
curiosa sobre temas do meu especial interesse. Li de fio a pavio peças
de legislação, regulamentos e uns tantos outros para construir e
executar algo que acho que vale a pena.

\subsection{Edição}

Este roteiro está na sua primeira 1ª (primeira) edição, esta é
correspondente e consistente com as informações reunidas e atualizadas
entre janeiro e junho de 2025. A versão é apenas um dos indicadores da
validade e fiabilidade da informação apresentada. \\
\\
\textbf{Histórico de edições} \\
\\
\textbf{2025-1}: Primeira edição: a primeira compilação extensa e detalhada do
processo de mudança de nome e sexo bem como as suas ramificações,
inclusão de anexos úteis, suportes informacionais e práticos.

\subsection{Sugestões, opiniões ou outras comunicações}

No decorrer do roteiro poderá notar a existência de erros, lacunas
informacionais ou imprecisões processuais. Na eventualidade de querer
contribuir com novas informações, correções de erros e imprecisões
pede-se que envie as suas sugestões utilizando como base o modelo de
mensagem de correio eletrónico (que poderá encontrar nos anexos) para o
endereço de correio eletrónico roteiromudanca@imariana.com. Poderá
também enviar as suas opiniões e outras comunicações relativas a este
documento que achar pertinentes para o mesmo endereço. \\
\\
\textbf{Ligação de referência}:
\begin{itemize}
	\item Tally.so, formulário para comunicações relativas ao roteiro, enviar a sua opinião, \url{https://tally.so/r/3lXd1p}
\end{itemize}
	
	%---------------
	% NOTAS PRÉVIAS
	%---------------
	%---------------
% NOTAS PRÉVIAS
%---------------

\newpage

\section{Notas prévias}

\subsection{Pressupostos}

\textbf{Palavras-chave}: condicionantes, utilidade, invalidez, circunstâncias, individualidades. \\
\\
A extensão da utilidade deste roteiro baseia-se em alguns pressupostos
que podem impactar significativamente a eficácia e utilidade do
documento, ainda que estas possam ser afetadas pelo não cumprimento de
pressupostos, esses pressupostos não determinam a invalidade do roteiro
como um todo mas sim apenas o facto da não aplicabilidade à
circunstância específica da pessoa. \\
\\
Apontam-se alguns desses pressupostos:
\begin{itemize}
	\item Ter \textbf{nacionalidade Portuguesa};
	\item Ser detentor de \textbf{Cartão de Cidadão};
	\item Estar fisicamente \textbf{presente em território nacional};
	\item Ser \textbf{maior de idade};
	\item Ter capacidade psicológica, autonomia de decisão e \textbf{idoneidade};
	\item Ter competências informáticas básicas;
	\item Não estar a cumprir uma pena de prisão;
	\item Não estar numa situação de internamento;
	\item Não estar em processo de insolvência;
	\item Não estar numa circunstância de incapacidade agravada ou permanente;
	\item \ldots{} .
\end{itemize}
\leavevmode\\
A lista de pressupostos é demonstrativa e não exaustiva, não são
mencionadas todas as potenciais condicionantes dado a existência de
muitas experiências de vida diferentes onde seria difícil enumerar todos
os ângulos da experiência de vida de modo a refletir uma visão completa
das condicionantes.

\subsection{Pré-requisitos}

\textbf{Palavras-chave}: requisitos necessários, requisitos preferenciais, documentos, recursos financeiros. \\
\\
Antes de começar o seu processo de mudança de nome e sexo, é fundamental
observar um conjunto de requisitos, a existência dos elementos listados
como pré-requisito irão facilitar os processos individuais posteriores
que irá despoletar ao iniciar o seu processo.

\subsubsection{Requisitos essenciais}

Identifica-se assim os requisitos essenciais para iniciar o seu
processo:
\begin{itemize}
	\item \textbf{Cartão de cidadão}, deve estar em bom estado, legível e o respetivo chip funcional (fator preferencial);
	\item \textbf{Tempo}, poderá ter de dispensar até 6 (seis) horas na identificação essencial, no entanto pode chegar a mais de 30 (trinta) dependendo da quantidade de locais e dados a alterar;
	\item \textbf{Financeiros}: No mínimo 17,00€ (dezassete euros) para o Cartão de Cidadão e concluir a identificação essencial, podendo no entanto ser superior a 150,00€ (cento e cinquenta euros) consoante o número de documentos, cartões e processos pelos quais tenha de pagar.
\end{itemize}

\subsubsection{Requisitos preferenciais}

Os seguintes requisitos são preferenciais para que consiga efetuar o seu
processo de forma mais célere, menos burocrática e mais prática:
\begin{itemize}
	\item \textbf{Chave Móvel Digital (CMD)}, deve estar ativa e sem bloqueios;
	\item \textbf{Aplicação móvel \emph{gov.pt}}, ativada com todos os seus documentos de identificação relevantes, tais como o Cartão de Cidadão (CC), Certificado de Matrícula (DUA), Carta de Condução, e outros);
	\item \textbf{Acesso ao portal internet da Autoridade Tributária (AT)}, ativo e sem bloqueios;
	\item \textbf{Acesso ao portal internet Segurança Social Direta}, ativo e sem bloqueios;
	\item \textbf{Acesso ao portal e/ou aplicação móvel \emph{SNS24}}, ativo e sem bloqueios;
	\item \textbf{Acessos individuais aos vários portais de cada uma das empresas, organizações ou instituições com as quais tenha relações comerciais ou institucionais preestabelecidas}, acessos devem estar funcionais e sem condicionantes;
	\item \textbf{Acesso a novo endereço de correio eletrónico adequado para a nova identificação}, é ideal a existência de um novo endereço de correio eletrónico adequado, contendo por exemplo o novo nome;
	\item \textbf{Leitor de cartões (opcional)}, em bom estado e funcional, poderá ser útil em determinadas circunstâncias;
	\item \textbf{Aplicação móvel \emph{SIGA}}, poderá facilitar o agendamento ou obtenção de uma senha para o serviço pretendido;
	\item \textbf{Domicílio/Morada atualizada}, é importante por uma questão de facilidade ter a sua morada do cartão de cidadão atualizada;
	\item \textbf{Número de telemóvel atualizado}, poderá ser importante garantir previamente que o número de telemóvel associado aos vários acessos esteja atualizado, especialmente na Chave Móvel Digital (CMD).
\end{itemize}

\subsubsection{Perguntas e respostas}

\textbf{Palavras-chave}: Q\&A, FAQ, perguntas frequentes. \\

\paragraph{É necessário algum relatório médico para efetuar a mudança?}
\leavevmode\\
\textbf{Maiores de idades} não necessitam de qualquer relatório médico; \\
\textbf{Menores de idade} (entre os 16 e os 18 anos) necessitam apenas
de um relatório médico de um médico ou psicólogo registado na respetiva
Ordem profissional a atestar a decisão livre e informada da pessoa. \\

\paragraph{Menores que 16 anos podem efetuar a mudança de nome na sua identificação?}
\leavevmode\\
\textbf{Não}, podem apenas pedir a anotação do nome e identidade
preferencial ao abrigo da legislação em vigor, relativo à utilização de
nome social/preferencial. \\

\paragraph{É possível reverter o processo de mudança de nome?}
\leavevmode\\
\textbf{Sim (mas)}, no entanto apenas poderá fazê-lo através de
autorização judicial. É importante dar a mudança como permanente e
evitar a necessidade de reversão do processo. \\

\subsubsection{Inventariação dos itens a alterar}

\textbf{Palavras-chave}: organização, lista, to-do, a fazeres. \\
\\
Para que consiga efetuar a sua mudança com sucesso será útil
\textbf{inventariar por itens todas as alterações} que terá de efetuar
futuramente, esta listagem irá permitir dar a \textbf{prioridade}
adequada aos itens consoante a sua relevância e/ou impacto no seu
dia-a-dia. \\
\\
A melhor forma de começar a listagem é pensar naquilo que utiliza na
vida diária sem exceção e a partir daí ir pensando no restante que
utiliza com menos frequência. Outro método para conseguir listar os
itens poderá ser através da análise dos seus apontamentos (se tiver) ou
listagens de acessos a plataformas online ou sites, isso poderá dar um
ponto de partida alternativo ou até mesmo auxiliar o inventário daquilo
que irá precisar de alterar no futuro. \\
\\
A título de exemplo podemos considerar o seguinte pensamento, ``no
dia-a-dia o mais essencial é a minha \textbf{identificação} (o Cartão de
Cidadão), o \textbf{passe de transportes} públicos, e como também
conduzo com frequência também irei precisar de uma nova \textbf{Carta de
	Condução}''. O objetivo é conseguir determinar aquilo que é o
\textbf{mais importante} e daí ir listando outros itens menos
importantes ou prioritários para mais tarde. \\

\newpage

\paragraph{Lista de exemplo}
\leavevmode\\\\
Processo de mudança de nome, alterações a efetuar:
\begin{itemize}
	\item Assento ou Certidão de Nascimento;
	\item Cartão de Cidadão;
	\item Passe de transportes públicos;
	\item Carta de Condução;
	\item Cartão de funcionário;
	\item Certificado de Matrícula (DUA);
	\item \ldots{} .
\end{itemize}

\subsection{Considerações prévias}

As considerações prévias são um conjunto de notas importantes a ter em
conta no início e durante todos os processos, ainda que não obrigatórias
ou estritamente necessárias, irão dar uma ajuda no controlo e sucesso de
cada processo. Por outro lado serve também como forma de acautelar
situações imprevisíveis ou aspetos que possam estar fora do controlo do
iniciante do processo. \\

\subsubsection{Envio de mensagens de correio eletrónico}

\textbf{Palavras-chave}: correio eletrónico, avisos, recibos, leitura, entrega, confirmações. \\
\\
Ao enviar quaisquer mensagens de correio eletrónico deve idealmente ter
o cuidado de (quando possível) \textbf{pedir um recibo de entrega e um
de leitura} de forma a poder ter maior controlo e informação sobre o
estado de uma mensagem. É importante notar que alguns serviços tais como
o \emph{Gmail} (correio eletrónico da \emph{Google}) não permitem esse
tipo de pedido através da aplicação de navegador internet. No entanto é
possível pedir o recibo de entrega e de leitura se utilizar uma
aplicação ``clássica'' de correio eletrónico num computador, como por
exemplo o ``Outlook clássico''. Poderá pesquisar na internet como é que
deve proceder para ativar ambos os recibos de acordo com a sua situação
e configuração específica. \\
\\
\textbf{Ligações de referência}:
\begin{itemize}
	\item Microsoft, adicionar e pedir recibos de leitura e notificação de entrega, \url{https://support.microsoft.com/pt-pt/office/adicionar-e-pedir-recibos-de-leitura-e-notifica\%C3\%A7\%C3\%B5es-de-entrega-no-outlook-a34bf70a-4c2c-4461-b2a1-12e4a7a92141}
	\item Google, peça ou devolva um recibo de leitura, \url{https://support.google.com/mail/answer/9413651?hl=pt}
\end{itemize}

\subsubsection{Inconsistências durante os processos de identificação essencial}

\textbf{Palavras-chave}: cartão de cidadão, assento de nascimento, pedidos, erros. \\
\\
\textbf{Nota}: Entende-se por identificação essencial os processos de
alteração da certidão ou assento de nascimento e o pedido de novo Cartão
de Cidadão. \\
\\
Durante o processo de obtenção de um novo Cartão de Cidadão podem
verificar-se inconsistências, especialmente nos \textbf{processos
	``automáticos''} despoletados por este até que o Cartão de Cidadão e
Chave Móvel Digital estejam devidamente ativos, portanto, devem ser
consideradas finalizadas apenas as alterações dos processos
\textbf{verificadas após a ativação do Cartão de Cidadão} e entrada com
a Chave Móvel Digital em cada portal (preferível para maior comodidade).

\subsubsection{Dificuldades genéricas na alteração de dados}

\textbf{Palavras-chave}: ajuda, pedidos, mudança, alteração, alternativas. \\
\\
Em caso de dúvida ou dificuldade nas alterações, é preferível que se
desloque-se a um \textbf{estabelecimento físico}, contacte o
\textbf{suporte ou apoio ao cliente} do respectivo serviço, empresa ou
instituição ou \textbf{envie uma mensagem de correio eletrónico} de modo
a obter mais informações acerca de qual será a melhor opção.\\
\\
Na eventualidade de não ser possível através de apoio telefónico,
correio eletrónico ou balcão de atendimento da empresa ou instituição em
questão, deverá assim ser \textbf{enviada uma carta registada com aviso
	de receção} para a morada da empresa ou instituição com pedido de
retificação dos seus dados pessoais e os anexos necessários para a sua
validação, tal como uma cópia do documento com marca de água adequada se
possível.\\
\\
Em último recurso, a retificação dos seus dados só poderá ser resolvida
\textbf{judicialmente}, isto caso nenhum dos pedidos previamente
apresentados através dos métodos anteriormente referidos seja
respeitado. Todavia, poderá sempre desistir do processo de alteração ou
tentar novamente noutro momento que seja mais oportuno. \\
\\
\textbf{Recursos úteis}:
\begin{itemize}
	\item Modelo de carta - Pedido de alteração/retificação de dados pessoais;
	\item Suportes jurídicos, diversos em especial o RGPD (2016/679 de 27 de abril de 2016) e execução nacional (Lei n.º 58/2019, de 8 de agosto).
\end{itemize}

\newpage

\subsubsection{Casos específicos de dificuldade na alteração de dados}

\textbf{Palavras-chave}: rgpd, proteção de dados, questões jurídicas, alterações, pedidos. \\
\\
Em certos casos de menor importância onde o seu acesso, perfil ou conta
existente em portais, aplicações e lojas será mais difícil ou até mesmo
impossível de alterar ou retificar, poderá fazer mais sentido recriar a
conta eliminando-a e criando outra dado a rigidez ou falta de
flexibilidade administrativa desses sítios. Poderá no entanto sempre
invocar os seus direitos ao abrigo do Regulamento Geral de Proteção de
Dados (RGPD) com o pedido de retificação de dados, ainda que
possivelmente seja mais rápida a utilização de processos já bem
estabelecidos tais como eliminação e criação de conta. \\
\\
Podem existir sistemas ou suportes de dados nos quais a alteração possa
ser dificultada ou até mesmo praticamente impossível, aponta-se aqui por
exemplo dados históricos em arquivos ou suporte de papel, e/ou outros
suportes duradouros e de difícil acesso. Nesses casos será complicado ou
até mesmo impossível proceder à alteração dos dados. \\
\\
\textbf{Recursos úteis}:
\begin{itemize}
	\item Modelo de carta - Pedido de alteração/retificação de dados pessoais;
	\item Suportes jurídicos, diversos em especial o RGPD (2016/679 de 27 de abril de 2016) e execução nacional (Lei n.º 58/2019, de 8 de agosto).
\end{itemize}

\subsubsection{\texorpdfstring{Utilizar a aplicação móvel \emph{gov.pt}}{Utilizar a aplicação móvel gov.pt}}

\textbf{Palavras-chave}: app, identificação, cc, cartão de cidadão. \\
\\
Ao longo dos vários processos poderá ser útil obter e configurar a
aplicação móvel \emph{gov.pt}, esta aplicação permite centralizar os
seus documentos de identificação bem como exportá-los, apresentando-se
como uma alternativa e um complemento útil. Ao autenticar-se com a chave
móvel a aplicação permite que aceda com maior facilidade aos vários
serviços centralizando também os códigos de acesso únicos que costumam
ser gerados no momento de início de sessão. Esta aplicação é
particularmente útil para aquelas pessoas que têm dificuldades na
organização e lembrarem-se onde deixam as coisas, os documentos uma vez
na aplicação têm a mesma validade legal e podem ser utilizados no
dia-a-dia sem agravar o risco de perda desses documentos. \\
\\
\textbf{Documento de referência}:
\begin{itemize}
	\item Anexos gerais, Anexo D, Manual da aplicação móvel \emph{gov.pt}
\end{itemize}
\leavevmode\\
\textbf{Ligações de referência}:
\begin{itemize}
	\item AMA, página da aplicação \emph{gov.pt}, \url{https://www.ama.gov.pt/web/agencia-para-a-modernizacao-administrativa/gov.pt}
	\item gov.pt, adicionar documentos de identificação na aplicação \emph{gov.pt}, \url{https://www.gov.pt/servicos/adicionar-documentos-de-identificacao-na-app-id-gov-pt}
\end{itemize}

\subsubsection{\texorpdfstring{Exportação da identificação através da aplicação móvel \emph{gov.pt}}{Exportação da identificação através da aplicação móvel gov.pt}}

\textbf{Palavras-chave}: cópia de documentos, cópia cc, cartão de cidadão, certificação. \\
\\
A primeira exportação do Cartão de Cidadão através da aplicação móvel
\emph{gov.pt} após o pedido de novo Cartão de Cidadão não irá conter o
seu número de identificação fiscal (NIF), para suprir essa lacuna,
poderá solicitar uma certidão de morada fiscal no portal da Autoridade
Tributária. \\
\\
Após a receção e ativação do seu novo Cartão de Cidadão, será então
possível exportar um documento em formato \emph{PDF} assinado com o
número de identificação fiscal (NIF) utilizando a aplicação móvel
\emph{gov.pt} ou a aplicação para ambiente de trabalho
\emph{Autenticação Gov}. \\
\\
\textbf{Documento de referência}:
\begin{itemize}
	\item Anexos gerais, Anexo D, Manual da aplicação móvel \emph{gov.pt}
\end{itemize}

\subsubsection{Processos para cidadãos estrangeiros}

\textbf{Palavras-chave}: estrangeiros, nacionalidade, apátridas, identidade. \\
\\
Todos os cidadãos estrangeiros e sem nacionalidade portuguesa terão de
obrigatoriamente seguir a legislação em vigor no seu país de origem,
logo os processos iniciais de identificação essencial não serão
aplicáveis. Por vezes a execução desse processo poderá ser inviável,
pelas mais diversas razões legais ou processuais, em alternativa poderá
optar pela invocação da lei que lhe confere o direito à autodeterminação
e expressão da identidade de género (Lei n.º 38/2018 de 7 de agosto),
nos registos de identificação das entidades terceiras. Há no entanto uma
limitação relevante, a invocação do direito à autodeterminação poderá
ter uma baixa taxa de sucesso devido à ausência de suportes informáticos
adequados à acomodação e integração dos seus dados adicionais, poderá em
muitos casos ser mesmo inviável ou impossível.

\subsubsection{Digitalização de documentos}

\textbf{Palavras-chave}: digitalizações, scan, segurança. \\
\\
Em certas circunstâncias é importante garantir a qualidade dos
documentos digitalizados que poderá vir a precisar de enviar no
seguimento de alguns processos. É preferível e sempre que possível deve
optar por digitalizar os documentos utilizando um equipamento
digitalizador adequado tal como uma impressora multifunções, nesta
deverá escolher a qualidade máxima ou uma qualidade superior a 300 PPI.
Caso não saiba alterar essas definições faça apenas uma digitalização
normal. Em alternativa e caso não tenha acesso a uma digitalizadora
adequada poderá utilizar uma aplicação móvel para ``digitalização'' de
documentos usando a câmara do seu equipamento, ainda que esta deva ser
uma opção de recurso. \\
\\
Aconselha-se no âmbito da segurança documental, que sempre que
digitalizar um documento de identificação, antes de o enviar, tirar uma
cópia e traçar essa mesma cópia com identificação do propósito da cópia
do documento e só aí voltar a digitalizar e enviar de modo a combater a
potencial utilização indevida de documentos de identificação. \\
\\
\textbf{Ligação de referência}:
\begin{itemize}
	\item WikiHow, como digitalizar documentos, \url{https://pt.wikihow.com/Digitalizar-Documentos}
\end{itemize}

\subsubsection{Como traçar um documento de identificação}

\textbf{Palavras-chave}: digitalização segura, segurança, identidade. \\
\\
Para traçar um documento de identificação ou outro de particular
sensibilidade, para fins de segurança documental e individual deve
efetuar os seguintes passos:
\begin{itemize}
	\item Fazer uma cópia do documento;
	\item Efetuar dois traços diagonais por cima do documento;
	\item Nas linhas diagonais deve escrever o propósito da cópia;
	\item Poderá opcionalmente assinar;
	\item Se o pedido foi em formato digital deve digitalizar a cópia traçada e	enviar.
\end{itemize}
\leavevmode\\
Ao traçar os documentos antes de os enviar estará a diminuir a
possibilidade de os seus documentos serem utilizados de forma
fraudulenta ou ilegal. É importante fazer aquilo que estiver ao seu
alcance para combater o roubo de identidade e a fraude.

\subsubsection{Esclarecimentos das menções a aplicações}

\textbf{Palavras-chave}: apps, referências, portais, aplicações, telemóvel. \\
\\
Para melhor entendimento do utilizador do roteiro são feitas menções ao
longo do documento a aplicações e portais e é importante garantir o
entendimento e a distinção de cada uma delas:
\begin{itemize}
	\item \textbf{Aplicação móvel \emph{gov.pt}}, é uma aplicação para telemóvel que permite exportar os seus dados de identificação e auxiliar na autenticação em diversos locais utilizando a Chave Móvel Digital.
	\item \textbf{Portal \emph{Autenticação Gov}}, é um portal acessível através do seu navegador utilizando um leitor de catões e o seu Cartão de Cidadão ou Chave Móvel Digital.
	\item \textbf{Aplicação \emph{Autenticação Gov}}, é uma aplicação de ambiente de trabalho que permite assinar documentos, verificar os seus dados de identificação e inclusivamente exportá-los.
\end{itemize}
\leavevmode\\
\textbf{Ligações de referência}:
\begin{itemize}
	\item Autenticação gov.pt, aplicação móvel gov.pt, \url{https://www.autenticacao.gov.pt/aplicacao/autenticacao-gov-movel}
	\item Autenticação gov.pt, aplicação Autenticação Gov, \url{https://www.autenticacao.gov.pt/web/guest/cc-aplicacao}
	\item Autenticação gov.pt, portal Autenticação Gov, \url{https://www.autenticacao.gov.pt/}
\end{itemize}

\subsubsection{Escolha do(s) nome(s) próprio(s)}

\textbf{Palavras-chave}: escolher um nome, questões jurídicas, lista de nomes. \\
\\
É importante notar que a escolha do(s) nome(s) próprio(s) tem de
obrigatoriamente obedecer à lista de nomes aprovados pelo Instituto dos
Registos e Notariado (IRN), anexo c do roteiro, isto significa que só
poderá pedir o registo de um nome que exista na lista em anexo. Além
disso é importante que este cumpra com as regras de composição do nome
em vigor. Se o nome preferido não constar na lista de nomes aprovados
poderá pedir um parecer onomástico com um custo de 75,00€ (setenta e
cinco euros). Noutras circunstâncias específicas e que saiam do contexto
de aprovação regular é preferível que consulte um especialista,
preferencialmente um advogado ou solicitador. \\
\\
\textbf{Documento de referência}:
\begin{itemize}
	\item Anexos gerais, Anexo C - Lista de nomes próprios aprovados pelo IRN.
\end{itemize}
\leavevmode\\
\textbf{Ligações de referência}:
\begin{itemize}
	\item IRN, composição do nome, \url{https://irn.justica.gov.pt/Servicos/Cidadao/Nascimento/Composicao-do-nome}
	\item IRN, lista dos nomes próprios, \url{https://irn.justica.gov.pt/Portals/33/Regras\%20Nome\%20Proprio/Lista\%20Nomes\%20Pr\%C3\%B3prios.pdf?ver=WNDmmwiSO3uacofjmNoxEQ\%3D\%3D}
	\item IRN, custos dos serviços, \url{https://irn.justica.gov.pt/Custos-dos-servicos}
\end{itemize}

\subsubsection{Conservação dos documentos anteriores}

É importante conservar todos os documentos relativos ao seu processo de
mudança de nome e sexo, nomeadamente os seus documentos de
identificação, isto porque numa situação onde não seja possível
identificar a pessoa pelo novo documento, por lapso documental ou
problema de sistema, o seu anterior documento dá-lhe uma forma
alternativa de garantir a sua identificação. Aponta-se assim um exemplo
prático, numa situação onde o caderno eleitoral não está atualizado
devidamente, o facto de ter o seu anterior documento irá permitir que
possa exercer o seu direito ao voto, de outra forma o direito ao voto
poderia ser vedado. 

\newpage

\subsubsection{Aplicações úteis}

Adiante estão listadas as aplicações lhe podem ser úteis neste mudança e
para o futuro:
\begin{itemize}
	\item \textbf{Aplicação móvel \emph{gov.pt}}, para os seus documentos de identificação e autenticação em vários serviços;
	\item \textbf{Aplicação móvel \emph{SIGA}}, mais cómodo para obter senha nos serviços que necessita tais como o Cartão de Cidadão ou Passaporte;
	\item \textbf{Aplicação móvel Segurança Social direta}, útil para rever os seus dados e informações junto da Segurança Social no futuro;
	\item \textbf{Aplicação móvel \emph{SNS24}}, útil para rever as as suas informações mas também ter acesso às suas consultas, exames e prescrições.
\end{itemize}
\leavevmode\\
\textbf{Ligações de referência}:
\begin{itemize}
	\item Aplicação móvel para sistema Android:
	\begin{itemize}
		\item \emph{gov.pt}, \url{https://play.google.com/store/apps/details?id=id.gov.pt\&hl=pt\_PT}
		\item \emph{SIGA}, \url{https://play.google.com/store/apps/details?id=pt.segsocial.iies.sigaapp.prod\&hl=pt\_PT}
		\item Segurança Social, \url{https://play.google.com/store/apps/details?id=pt.segsocial.mobile.segurancasocial\&hl=pt\_PT}
		\item \emph{SNS24}, \url{https://play.google.com/store/apps/details?id=pt.minsaude.spms.ces\&hl=pt\_PT}
	\end{itemize}
	\item Aplicação móvel para sistema iOS:
	\begin{itemize}
		\item \emph{gov.pt}, \url{https://apps.apple.com/pt/app/gov-pt/id1384884826}
		\item \emph{SIGA}, \url{https://apps.apple.com/pt/app/sigaapp/id1127868225}
		\item Segurança Social, \url{https://apps.apple.com/pt/app/seguran\%C3\%A7a-social/id1469920521}
		\item \emph{SNS24}, \url{https://apps.apple.com/pt/app/sns-24/id1192353854}
	\end{itemize}
\end{itemize}

\subsubsection{Consideração dos custos financeiros}

Ao longo do roteiro são apontados custos financeiros a cargo do
utilizador, esses apenas representam os encargos direitos na execução
dos processos, esses custos diretos não englobam todos os outros
potenciais custos, considerados como indiretos, tais como, impressões,
deslocações, consultorias externas, e outros demais serviços, itens ou
elementos de suporte administrativo e prático aos vários processos. Isto
significa que a execução de um determinado processo poderá ser
significativamente mais dispendiosa do que o antecipado pois o roteiro
apenas prevê o custo e processo genérico. A sua circunstância poderá ser
significativamente diferente das condições ideais previstas agravando
assim os custos totais de execução englobando todos os custos.
	
	%---------------------
	% ALTERAÇÕES INICIAIS
	%---------------------
	%---------------------
% ALTERAÇÕES INICIAIS
%---------------------

\newpage

\section{Alterações iniciais}

\subsection{Preparações prévias}

\textbf{Palavras-chave}: correio eletrónico, preparação, assinatura manual.

\subsubsection{Relevância das preparações prévias}

Antes de iniciar o seu processo de mudança de nome e sexo será útil
fazer umas preparações prévias de modo a agilizar e facilitar todos os
processos que irá iniciar posteriormente, entre as várias preparações
destacam-se a alteração do endereço de correio eletrónico e a nova
assinatura manuscrita.

\subsubsection{Endereço de correio eletrónico}

Para obter um novo endereço de correio eletrónico de forma gratuita
poderá inscrever-se junto de um do grandes fornecedores tais como a
\emph{Google} ou a \emph{Microsoft}. A \emph{Google} fornece correio
através do serviço \emph{GMail} e a \emph{Microsoft} através do serviço
Outlook online. É importante que o novo endereço seja composto por
termos neutrais, que inclua os seus nomes (pelo menos primeiro e
último), sem nomes estranhos (como por exemplo ``\emph{fofinhax}'' ou
``\emph{xpto}'') ou referências populares e profissionais, o seu futuro
eu irá agradecer. \\
\\
\textbf{Ligações de referência}:
\begin{itemize}
	\item Google, correio eletrónico da Google (GMail), \url{https://mail.google.com/mail/u/0/}
	\item Microsoft, correio eletrónico da Microsoft (Outlook), \url{https://outlook.live.com/mail/about/index\_pt.html}
\end{itemize}

\subsubsection{Assinatura manuscrita}

Ao mudar de nome provavelmente também terá de alterar a sua assinatura
manuscrita, para isso é importante que antes de ir fazer o seu novo
Cartão de Cidadão pratique pelo menos algumas vezes a sua nova
assinatura. Poderá achar que é uma boa altura para fazer uma assinatura
diferente do seu habitual, para conseguir fazer essa mudança com sucesso
é sugerido que faça uma pesquisa prévia, procure alguma inspiração na
internet e treine a bastante a sua nova assinatura. \\
\\
\textbf{Ligações de referência}:
\begin{itemize}
	\item WikiHow, como fazer uma assinatura bonita, \url{https://pt.wikihow.com/Ter-uma-Assinatura-Bonita}
\end{itemize}

\subsubsection{Ativar a Chave Móvel Digital}

Para ativar a sua chave móvel poderá utilizar um dos guias disponíveis
na internet. \\
\\
\textbf{Ligações de referência}:
\begin{itemize}
	\item Autenticação gov, pedido de Chave Móvel Digital, \url{https://www.autenticacao.gov.pt/cmd-pedido-chave}
	\item gov.pt, ativar a sua Chave Móvel, \url{https://www.gov.pt/servicos/ativar-a-chave-movel-digital}
\end{itemize}

\subsection{Identificação essencial}

\subsubsection{Introdução e requisitos}

Para iniciar o processo de mudança de nome e sexo terá de começar por
alterar dois documentos de identificação essenciais, a Certidão/Assento
de Nascimento e a seguir o seu Cartão de Cidadão. Estes primeiros passos
podem levar até um mês para serem finalizados. Há que notar que assim
que tenha a nova Certidão/Assento de Nascimento já poderá começar a
efetuar algumas das alterações sendo que o mais útil será esperar pelo
menos até ter o pedido de novo Cartão de Cidadão em mãos, pois
geralmente com esse documento já irá conseguir alterar outros com mais
facilidade. Ambos os processos apenas podem ser feitos presencialmente
numa Conservatória do Registo Civil sendo que o agendamento prévio não é
estritamente necessário. Poderá pedir aconselhamento enquanto estiver na
conservatória se fará sentido agendar a renovação do cartão de cidadão
por uma questão de conveniência ou facilidade processual. \\
\\
\textbf{Custos diretos totais expectáveis nesta fase}: Pelo menos 17,00€ (dezassete euros). \\
\textbf{Tempo total expectável de execução da fase}: Cerca de 22 (vinte e dois) dias. \\
\\
\textbf{Pré-requisitos desta fase:}
\begin{itemize}
	\item \textbf{Cartão de cidadão} válido e em relativo bom estado;
	\item \textbf{Requerimento} para \textbf{mudança da menção de nome e sexo} devidamente preenchido e assinado de acordo com a assinatura do Cartão de Cidadão;
	\item Presença dos pais (na eventualidade de se tratar de um menor entre os 16 (dezasseis) e os 18 (dezoito) anos).
\end{itemize}
\leavevmode\\
\textbf{Objetivos desta fase}:
\begin{itemize}
	\item Retificação e obtenção de novo \textbf{Assento de Nascimento};
	\item Obtenção de um novo \textbf{Cartão de Cidadão}.
\end{itemize}
\leavevmode\\
\textbf{Fluxo genérico de acontecimentos desta fase}:
\begin{itemize}
	\item Requer alteração do Assento de Nascimento;
	\item Obter o novo Assento de Nascimento;
	\item Pedir novo Cartão de Cidadão;
	\item Obter o novo Cartão de Cidadão;
	\item Ativar o novo Cartão de Cidadão;
	\item Ativar a assinatura digital qualificada;
	\item Ativar a Chave Móvel Digital.
\end{itemize}

\subsubsection{Certidão de nascimento}

\textbf{Palavras-chave}: marcador de sexo, assento de nascimento, registo civil, conservatória. \\
\\
Para efetuar o pedido de nova certidão de nascimento deverá deslocar-se
a uma conservatória do registo civil, com o requerimento de mudança de
nome e sexo já devidamente preenchido preferencialmente. No local poderá
tirar uma senha de ``Registo Civil - Certidões / Informações'', no
entanto se tiver a aplicação móvel \emph{SIGA} poderá tirar uma senha
digitalmente para maior comodidade e atendimento mais rápido. Quando a
sua senha for chamada deverá entregar no balcão o seu requerimento e
atual cartão de cidadão. Que estiver a efetuar o atendimento irá fazer
alguma questões e confirmar a vontade de alteração, preencher uns
documentos adicionais e pedir a sua assinatura no pedido formalizado.
Após efetuar o pedido será informado o tempo médio até à conclusão do
seu pedido onde provavelmente irão dizer qual a data para voltar e poder
recolher a sua nova certidão de nascimento. Neste processo poderá
dependendo da vontade do outro lado de agendar a execução do novo cartão
de cidadão. \\
\\
\textbf{Documentos necessários}:
\begin{itemize}
	\item Cartão de Cidadão em bom estado;
	\item Requerimento para mudança da menção de nome e sexo devidamente preenchido e assinado de acordo com a assinatura do Cartão de Cidadão;
	\item Presença dos pais (na eventualidade de se tratar de um menor).
\end{itemize}
\leavevmode\\
\textbf{Objetivo}: Obter uma nova Certidão de Nascimento. \\
\textbf{Método}: Deslocação a uma Conservatória do Registo Civil. \\
\textbf{Custo deste processo}: Gratuito. \\
\textbf{Entrega da nova	certidão}: No local do pedido em mãos. \\
\textbf{Tempo expectável de	execução}: Até 7 (sete) dias. \\
\\
\textbf{Ligações de referência}:
\begin{itemize}
	\item justiça.gov.pt, mudança de nome e sexo, \url{https://justica.gov.pt/Registos/Civil/Mudanca-de-sexo-e-de-nome-proprio}
	\item gov.pt, pedir o registo de mudança de sexo e de nome próprio, \url{https://www.gov.pt/servicos/pedir-o-registo-de-mudanca-de-sexo-e-de-nome-proprio}
\end{itemize}

\newpage

\paragraph{Dica para residentes em Lisboa}
\leavevmode\\
Como alternativa mais célere para os residentes em Lisboa, é possível o
envio de uma mensagem de correio eletrónico, para os seguintes
endereços:\\
\\
\href{mailto:direcao.apoio.civil.lisboa@irn.mj.pt}{\nolinkurl{direcao.apoio.civil.lisboa@irn.mj.pt}} \\
\href{mailto:civil.lisboa@irn.mj.pt}{\nolinkurl{civil.lisboa@irn.mj.pt}}
\\
\leavevmode\\
Na sua mensagem deve incluir a seguinte informação:
\begin{itemize}
	\item Nome Completo;
	\item Dados completos do atual Cartão de Cidadão (CC), poderá ser uma cópia
	traçada;
	\item Contacto telefónico;
	\item Endereço de correio eletrónico;
	\item Nome(s) pretendido(s);
	\item Género pretendido.
\end{itemize}
\leavevmode\\
A Conservatória do Registo Civil irá contactar posteriormente no sentido
de confirmar o agendamento e dar seguimento à confirmação, bem como uma
resposta acerca das informações necessárias em resposta à mensagem
enviada previamente. \\
\\
Existe relatos de situações onde não houve lugar a pagamento do Cartão
de Cidadão, não deve no entanto ser considerado a norma.

\paragraph{Certidão de nascimento dos filhos}
\leavevmode\\
O averbamento dos assentos de nascimento dos filhos da pessoa que
efetuou a mudança de nome e sexo tem algumas limitações importantes
relevar:
\begin{itemize}
	\item O averbamento da certidão de nascimento de filho(a) já nascido maior só poderá ser feito mediante pedido desse(a) filho(a);
	\item O averbamento da certidão de nascimento de filho(a) menor não é possível de acordo com a legislação vigente.
\end{itemize}

\paragraph{Certidão de casamento}
\leavevmode\\
Se a pessoa for casada só poderá ver o assento de casamento alterado por
consentimento de ambas as partes.

\paragraph{Reversão da alteração do assento}
\leavevmode\\
É importante notar que na eventualidade de a pessoa querer reverter o
seu assento de nascimento à menção de nome e sexo original só o poderá
ver feito mediante autorização judicial.

\paragraph{Confidencialidade do ato}
\leavevmode\\
A mudança da menção de nome e sexo é um ato secreto, onde apenas a
própria pessoa, os seus herdeiros, e entidades judiciais ou policiais no
seguimento de um processo de instrução criminal ou decisão judicial o
podem revelar. O pedido de certidão de nascimento que habitualmente é
aberto ao público nestes casos torna-se vedado, podendo apenas o próprio
requer tal certidão.

\subsubsection{Cartão de Cidadão}

\textbf{Palavras-chave}: identificação, essencial, bilhete de identidade, conservatória, marcador de sexo. \\

\paragraph{Início do processo}
\leavevmode\\
O novo cartão de cidadão deve ser pedido e feito numa conservatória do
registo civil. É preferível agendar previamente de modo a agilizar o
processo e evitar atrasos ou falta de senha na conservatória pretendida.
No dia do seu agendamento deve levar a nova certidão de nascimento (ou
se for levantar a sua nova certidão irão encaminhar a sua certidão) para
então fazer o novo cartão de cidadão, idealmente deverá ter praticado a
sua nova assinatura. Finalizado o processo irá obter um documento que
confirma o pedido de novo cartão de cidadão, com esse documento já
poderá pedir ou alterar alguns dos seus dados noutros locais, tais como,
passe de transporte ou uma ficha cliente. \\
\\
\textbf{Documentos necessários}:
\begin{itemize}
	\item Cartão de Cidadão atual;
	\item Novo Assento/Certidão de Nascimento.
\end{itemize}
\leavevmode\\
\textbf{Objetivo}: Obter um novo Cartão de Cidadão. \\
\textbf{Método}: Deslocação a uma Conservatória do Registo Civil. \\
\textbf{Custo deste	processo}: A partir de 17,00€ (dezassete euros). \\
\textbf{Entrega do novo	cartão de cidadão}: No local de pedido ou no domicílio. \\
\textbf{Tempo expectável de execução}: Até 15 (quinze) dias. \\
\\
\textbf{Ligações de referência}:
\begin{itemize}
	\item justica.gov.pt, Agendar Cartão de Cidadão, \url{https://justica.gov.pt/Servicos/Agendar-Cartao-de-Cidadao}
	\item gov.pt, renovar o Cartão de Cidadão, \url{https://www.gov.pt/servicos/renovar-o-cartao-de-cidadao}
\end{itemize}

\paragraph{No momento da recolha e/ou ativação do Cartão de Cidadão	na Conservatória do Registo	Civil}
\leavevmode\\[4pt]
Pode ativar a assinatura digital qualificada para facilitar a entrega de
documentos digitalmente assinados a terceiros, como entidades estatais
ou empresas. Este passo é opcional, mas geralmente é preferível ativar a
assinatura no momento da ativação do Cartão na conservatória por um
questão de comodidade. \\
\\
Os dados do portal \emph{Autenticação Gov} só são atualizados no momento
da entrega e ativação do novo Cartão de Cidadão bem como a Chave Móvel
Digital. O registo da Chave Móvel é possível ser feito utilizando o
mesmo número de telemóvel da Chave Móvel anterior aquando a sua ativação
na Conservatória do Registo Civil.

\subsection{Principais instituições/entidades}

\subsubsection{Introdução e	requisitos}

Para iniciar esta fase irá incidir sobre as outras principais
instituições e entidades nas quais precisa de garantir que os seus dados
pessoais encontram-se devidamente atualizados. \\
\\
\textbf{Custos totais diretos expectáveis nesta fase}: Nenhum. \\
\textbf{Tempo total expectável de execução da fase}: Cerca de 20 (vinte)
dias. \\
\\
\textbf{Pré-requisitos desta fase}:
\begin{itemize}
	\item Acesso ao portal da \textbf{Autoridade Tributária};
	\item Acesso ao portal \textbf{Segurança Social} direta;
	\item Acesso ao portal do \textbf{Sistema Nacional de Saúde} (\emph{SNS24});
	\item Acesso à \textbf{aplicação móvel \emph{gov.pt}} com os seus documentos;
	\item \textbf{Chave Móvel Digital} ativa.
\end{itemize}
\leavevmode\\
\textbf{Objetivos desta fase}:
\begin{itemize}
	\item Verificar os seus dados pessoais junto da \textbf{Autoridade Tributária}, \textbf{Segurança Social} e \textbf{Sistema Nacional de Saúde};
	\item Retificar os dados junto do \textbf{Ministério da Defesa Nacional};
	\item Obter uma nova \textbf{cédula militar};
	\item Obter o \textbf{estatuto de objetor de consciência} (opcional);
	\item Obter uma lista das suas contas e responsabilidades junto do \textbf{Banco de Portugal} e a titularidade.
\end{itemize}
\leavevmode\\
\textbf{Fluxo genérico de acontecimentos}:
\begin{itemize}
	\item Verificar os seus dados no portal da Autoridade Tributária;
	\item Verificar os seus dados no portal da Segurança Social direta;
	\item Verificar os seus dados no portal \emph{SNS24};
	\item Efetuar pedido de retificação dos seus dados junto do Ministério da Defesa Nacional (MDN);
	\item Obter uma nova cédula militar atualizada;
	\item Efetuar pedido do estatuto de objetor de consciência (opcional);
	\item Verificar e despoletar a atualização dos dados no recenseamento eleitoral;
	\item Verificar os seus dados pessoais junto do Banco de Portugal.
\end{itemize}

\subsubsection{Autoridade Tributária (AT)}

\textbf{Palavras-chave}: impostos, autoridade tributária, at, fiscal. \\
\\
Efetuado automaticamente após o pedido de novo cartão de cidadão. O
processo é finalizado instantaneamente sem intervenção. Inclui o portal
da Autoridade Tributária. \\
\\
\textbf{Documento necessário}:
\begin{itemize}
	\item Acesso ao portal da Autoridade Tributária.
\end{itemize}
\leavevmode\\
\textbf{Objetivo}: Atualização dos seus dados junto da Autoridade Tributária. \\
\textbf{Método}: Automático. \\
\textbf{Custo deste processo}: Gratuito. \\
\textbf{Tempo expectável de execução}: Imediato. \\
\\
\textbf{Ligações de referência}:
\begin{itemize}
	\item Acesso gov.pt, entrada no portal da Autoridade Tributária, \url{https://www.acesso.gov.pt/v2/loginForm?partID=PFAP\&path=/geral/dashboard}
	\item Portal da Finanças, entrada, \url{https://www.portaldasfinancas.gov.pt/at/html/index.html}
\end{itemize}

\subsubsection{Segurança Social (SS)}

\textbf{Palavras-chave}: seg social, ss, segurança social, pensões, subsídios, programas sociais. \\
\\
Efetuado automaticamente após o pedido de novo cartão de cidadão. O
processo é finalizado instantaneamente sem intervenção. Inclui o portal
Segurança Social Direta. \\
\\
\textbf{Documento necessário}:
\begin{itemize}
	\item Acesso ao portal da Segurança Social direta.
\end{itemize}
\leavevmode\\
\textbf{Objetivo}: Atualização dos seus dados junto da Segurança Social. \\
\textbf{Método}: Automático. \\
\textbf{Custo deste processo}: Gratuito. \\
\textbf{Tempo expectável de execução}: Imediato. \\
\\
\textbf{Ligações de referência}:
\begin{itemize}
	\item Segurança Social, página sobre a Segurança Social direta, \url{https://www.seg-social.pt/seguranca-social-direta}
	\item Segurança Social, página de início de sessão no portal da Segurança Social direta, \url{https://app.seg-social.pt/sso/login?service=https\%3A\%2F\%2Fapp.seg-social.pt\%2Fptss\%2Fcaslogin}
	\item Google Play Store, aplicação móvel da Segurança Social para sistema Android, \url{https://play.google.com/store/apps/details?id=pt.segsocial.mobile.segurancasocial\&hl=pt\_PT}
	\item Apple App Store, aplicação móvel da Segurança Social para sistema iOS, \url{https://apps.apple.com/pt/app/seguran\%C3\%A7a-social/id1469920521?l=en-GB}
\end{itemize}

\subsubsection{Sistema Nacional de Saúde (SNS)}

\textbf{Palavras-chave}: saúde, sns, sistema público de saúde. \\
\\
Efetuado automaticamente após o pedido de novo cartão de cidadão. O
processo é finalizado instantaneamente sem intervenção. Inclui o portal
\emph{SNS24}. Significando assim que as suas prescrições, exames e
outros documentos relevantes dentro deste âmbito já irão sair com o nome
correto.\\
\\
\textbf{Documento necessário}:
\begin{itemize}
	\item Acesso ao portal do Sistema Nacional de Saúde.
\end{itemize}
\leavevmode\\
\textbf{Objetivo}: Atualização dos seus dados junto do Sistema Nacional de Saúde. \\
\textbf{Método}: Automático. \\
\textbf{Custo deste processo}: Gratuito. \\
\textbf{Tempo expectável de execução}: Imediato. \\
\\
\textbf{Ligações de referência}:
\begin{itemize}
	\item SNS24, entrada, \url{https://www.sns24.gov.pt/pt/inicio}
	\item SNS24, início de sessão, \url{https://www.sns24.gov.pt/pt/login/utente}
	\item SNS24, ajuda da aplicação móvel SNS24, \url{https://www.sns24.gov.pt/pt/servico/app-sns-24}
	\item Google Play Store, Aplicação móvel SNS24 para sistema Android, \url{https://play.google.com/store/apps/details?id=pt.minsaude.spms.ces\&hl=pt\_PT}
	\item Apple App Store, Aplicação móvel SNS24 para sistema iOS, \url{https://apps.apple.com/pt/app/sns-24/id1192353854}
\end{itemize}

\newpage

\subsubsection{Ministério da Defesa	Nacional/DDN/BUD}

\textbf{Palavras-chave}: serviço militar, cédula militar, dia da defesa nacional.

\paragraph{Recenseamento militar (Cédula militar)}
\leavevmode\\[4pt]
Deve enviar um email para
\href{mailto:ddn@defesa.pt}{\nolinkurl{ddn@defesa.pt}} com o assunto
``ALTERAÇÃO DADOS'' incluindo o documento exportado da aplicação móvel
\emph{gov.pt} ou cópia do Cartão de Cidadão. Poderá utilizar o modelo de
mensagem de correio eletrónico para maior comodidade. \\
\\
\textbf{Documentos necessários}:
\begin{itemize}
	\item Modelo de mensagem de correio eletrónico para o Dia da Defesa Nacional/Balcão Único da Defesa (DDN/BUD);
	\item Cópia traçada do Cartão de Cidadão.
\end{itemize}
\leavevmode\\
\textbf{Objetivo}: Obter uma nova cédula militar atualizada. \\
\textbf{Método}: Envio de mensagem de correio eletrónico. \\
\textbf{Custo deste processo}: Gratuito. \\
\textbf{Entrega de nova cédula}: Virtual, no seu endereço de correio eletrónico. \\
\textbf{Tempo expectável de	execução}: Até 20 (vinte) dias. \\
\\
\textbf{Ligações de referência}:
\begin{itemize}
	\item Balcão Único da Defesa, cédula militar, \url{https://bud.gov.pt/ddn/cedula.html}
	\item Balcão Único da Defesa, convocação, \url{https://bud.gov.pt/ddn/convocacao.html}
	\item Balcão Único da Defesa, emitir cédula, \url{https://ddn.dgrdn.gov.pt/cedula\_bud.aspx}
\end{itemize}

\paragraph{Nota especialmente importante}
\leavevmode\\[4pt]
Os dados pessoais têm de ser obrigatoriamente atualizados em
conformidade com o Artigo 57º (quinquagésimo sétimo), alínea c da Lei do
Serviço Militar (Lei n.º 174/99, de 21 de setembro), o não cumprimento
do disposto implicará uma coima de aproximadamente 100,00€ (cem) a
500,00€ (quinhentos euros) de acordo com o Regulamento da Lei do Serviço
Militar (Decreto-Lei n.º 289/2000, de 14 de novembro) que poderá
agravar-se para o dobro em tempo de guerra. Ainda mais na eventualidade
de necessidade de recrutamento ao abrigo do Artigo 37º (trigésimo
sétimo) da LSM define que os ``contingentes da reserva de recrutamento a
classificar para efeitos da convocação (\ldots) obedece aos seguintes
fatores de preferência, por ordem de prioridade: a) Os cidadãos que
hajam injustificadamente faltado ao cumprimento de deveres militares''.

\paragraph{Nota adicional}
\leavevmode\\[4pt]
Poderá ser útil descarregar a cédula antiga antes de proceder ao pedido
de alteração, de modo a ter uma cópia da antiga. Deve descarregar uma
nova cédula assim que obter a confirmação da sua alteração no sistema do
Balcão Único da Defesa (BUD). \\

\paragraph{Como encontrar o seu Número de Identificação Militar (NIM)}
\leavevmode\\[4pt]
Para encontrar o seu Número de Identificação Militar (NIM) deverá
executar os seguintes passos:
\begin{itemize}
	\item Aceder ao portal do DDN através da ligação, \url{https://ddn.dgrdn.gov.pt/ddn\_editaispesq.aspx} ;
	\item Inserir o número de Cartão de Cidadão sem os dígitos de verificação;
	\item Inserir o nome completo em maiúsculas e sem acentuação;
	\item Clicar em ``Pesquisar''.
\end{itemize}

\paragraph{Como emitir a sua cédula militar}
\leavevmode\\[4pt]
Para emitir a sua cédula militar deverá executar os passos abaixo:
\begin{itemize}
	\item Aceder ao portal do DDN através da ligação, \url{https://ddn.dgrdn.gov.pt/cedula\_bud.aspx} ;
	\item Inserir o número de Cartão de Cidadão sem os dígitos de verificação;
	\item Inserir o endereço de correio eletrónico fornecido aquando a inscrição/entrega da cédula no Dia da Defesa Nacional;
	\begin{itemize}
		\item \textbf{Nota}: Na eventualidade de não saber qual o endereço terá de pedir a alteração do endereço de correio eletrónico associado no seu pedido de atualização de dados pessoais.
	\end{itemize}
	\item Confirmar a caixa ``não sou um robô'' e clicar em ``enviar''.
\end{itemize}

\subsubsection{Estatuto de objetor de consciência}

Poderá fazer sentido para algumas pessoas a requisição de estatuto de
objetor de consciência quando a sua convicção de ordem \textbf{moral},
\textbf{religiosa}, \textbf{humanística} ou \textbf{filosófica} não lhe
permite a utilização de \textbf{meios violentos de qualquer natureza
	contra o seu semelhante}, independentemente do âmbito, seja ele pessoal,
coletivo ou de defesa nacional. No entanto não impede a chamada a
efetuar serviços públicos ou comunitários em tempo de guerra. \\
\\
O \textbf{estatuto de objetor de consciência} uma vez reconhecido
\textbf{não permite} que a pessoa seja:
\begin{itemize}
	\item Titular de licença administrativa de detenção, uso e porte de arma de qualquer natureza;
	\item Titular de autorização de uso e porte de arma de defesa quando, por Lei, tal autorização seja inerente à função pública ou privada que exerça;
	\item Trabalhadora no fabrico, reparação ou comércio de armas de qualquer natureza ou no fabrico e comércio das respetivas munições, nem trabalhar em investigação científica relacionada com essas atividades.
	\item Praticante de qualquer outra atividade que exija o uso e porte de arma de qualquer natureza.
\end{itemize}
\leavevmode\\
O pedido do estatuto terá de ser sempre apresentado presencialmente
conjuntamente com todos os documentos devidamente preenchidos e
assinados. \\
\\

\newpage
\leavevmode\\
\textbf{Pedido dos documentos adicionais}
\begin{itemize}
	\item Certificado do registo criminal pode ser obtido online com um custo de 5,00€ (cinco euros);
	\item Certidão de Nascimento pode ser obtida online com um custo de 10,00€ (dez euros), ou 20,00€ (vinte euros) caso necessite em suporte de papel.
\end{itemize}
\leavevmode\\
\textbf{Documentos necessários}:
\begin{itemize}
	\item Certidão de Nascimento válida e verificável;
	\item Certificado do Registo Criminal do declarante;
	\item Formulários, Anexo E - Declaração de objeção de consciência perante o serviço militar;
	\item Anexo F - Declaração abonatória (Relativo ao estatuto de objetor de consciência).
\end{itemize}
\leavevmode\\
\textbf{Objetivo}: Obter o estatuto de objetor de consciência. \\
\textbf{Método}: Deslocação presencial ao Instituto Português do Desporto e Juventude (IPDJ). \\
\textbf{Custo deste processo}: Gratuito mas necessita de documentos pagos. \\
\textbf{Tempo expectável de execução}: Até 30 (trinta) dias. \\
\\
\textbf{Ligações de referência}:
\begin{itemize}
	\item IPDJ, objetores de consciência, \url{https://ipdj.gov.pt/objetores-de-consciencia}
	\item gov.pt, requisição do estatuto de objetor de consciência, \url{https://www2.gov.pt/servicos/requerer-estatuto-de-objetor-de-consciencia-de-servico-militar}
	\item IRN, pedir certidão de nascimento, \url{https://justica.gov.pt/servicos/pedir-certidao-de-nascimento}
	\item justica.gov.pt, pedir certificado do registo criminal, \url{https://registocriminal.justica.gov.pt/}
\end{itemize}

\newpage

\subsubsection{Recenseamento eleitoral}

\textbf{Palavras-chave}: votar, cadernos eleitorais, eleições, mesas de
voto. \\
\\
Por vezes o sistema poderá não atualizar o seu nome automaticamente,
para isso deverá efetuar um acesso ao \textbf{portal do recenseamento}
com o seu novo Cartão de Cidadão ou Chave Móvel Digital (CMD). Para
despoletar a atualização dos seus dados poderá ser útil atualizar os
seus dados de contacto. A alteração despoletada pelo acesso com o seu
novo documento de identificação e alteração dos dados de contacto
tipicamente torna-se visível no portal ao fim de 1 (uma) semana. Não
existem mecanismos de notificação das alterações ou atualização dos
dados pelo que terá de ir consultando para verificar se a alteração já
se encontra finalizada. \\
\\
\textbf{Documento necessário}:
\begin{itemize}
	\item Chave Móvel Digital (CMD) ativa.
\end{itemize}
\leavevmode\\
\textbf{Objetivo}: Atualizar os dados pessoais no recenseamento eleitoral. \\
\textbf{Método}: Autenticação no portal do recenseamento e atualização de contactos. \\
\textbf{Custo deste processo}: Gratuito. \\
\textbf{Tempo expectável de execução}: Até 7 (sete) dias. \\
\\
Aceder à sua página pessoal no portal do recenseamento eleitoral:
\begin{itemize}
	\item Aceder à página do portal do eleitor através da ligação, \url{https://www.eueleitor.mai.gov.pt/Login.aspx} ;
	\item Ir ao separador lateral ``Os meus dados pessoais'';
	\item Clicar no botão de ``editar contactos'';
	\item Atualizar o seu endereço de correio eletrónico e número de telemóvel e confirmar.
\end{itemize}
\leavevmode\\
Em alternativa pode submeter um pedido de alteração ou reclamação
através do formulário de contacto no portal do eleitor na página de
contactos em \url{https://www.portaldoeleitor.pt/pt/Contactos/Pages/default.aspx} . \\
\\
\textbf{Ligações de referência}:
\begin{itemize}
	\item Portal do Eleitor, entrada, \url{https://www.portaldoeleitor.pt/pt/Pages/default.aspx}
	\item Portal do Eleitor, início de sessão, \url{https://www.eueleitor.mai.gov.pt/Login.aspx}
	\item Portal do recenseamento, consulta dos cadernos de recenseamento, \url{https://www.recenseamento.pt/}
\end{itemize}
\leavevmode\\
\subsubsection{Banco de Portugal (BdP)}

\textbf{Palavras-chave}: Banco central, crc, créditos, beneficiário efetivo, contas bancárias. \\
\\
O acesso ao Banco de Portugal é relevante por 2 (duas) razões, a
consulta da titularidade das suas contas bancárias, e acesso à central
de responsabilidades de crédito. Estes dois itens permitem verificar uma
quantidade de informação, poderá verificar as titularidades e as contas
onde é beneficiário efetivo, por outro lado poderá ver os seus atuais
créditos, titularidade, garantias e outros demais incluindo
incumprimentos. Este acesso será maioritariamente útil para garantir que
o seu pedido de retificação de nome foi efetuado com sucesso junto das
várias entidades bancárias ou financeiras. \\
\\
\textbf{Documento necessário}:
\begin{itemize}
	\item Acesso da Autoridade Tributária ou Chave Móvel Digital ativa.
\end{itemize}
\leavevmode\\
\textbf{Objetivo}: Verificar que os dados junto de entidades bancárias ou financeiras ficaram devidamente alterados. \\
\textbf{Método}: Autenticação via acesso da Autoridade Tributária ou Chave Móvel Digital. \\
\textbf{Custo deste processo}: Gratuito. \\
\textbf{Tempo expectável de	execução}: Imediato. \\
\\
\textbf{Ligação de referência}:
\begin{itemize}
	\item Banco de Portugal, entrada, \url{https://www.bportugal.pt/}
	\item Banco de Portugal, área do Cidadão, \url{https://www.bportugal.pt/area-cidadao}
\end{itemize}

\subsection{Outros documentos de identificação}

\subsubsection{Passaporte Eletrónico Português (PEP)}

\textbf{Palavras-chave}: passaporte, viajar, identificação, vistos. \\
\\
Se for viajar, especialmente para fora do espaço \emph{Schengen} poderá
ser útil obter um novo passaporte, para isso é ideal o agendamento numa
Conservatória do Registo Civil, terá de levar o seu novo Cartão de
Cidadão. A entrega do Passaporte no domicílio acresce 10,00€ (dez euros). \\
\\
\textbf{Documento necessário}:
\begin{itemize}
	\item Cartão de Cidadão atualizado.
\end{itemize}
\leavevmode\\
\textbf{Objetivo}: Obter um novo passaporte atualizado. \\
\textbf{Método}: Deslocação a Conservatória do Registo Civil ou Loja do Cidadão. \\
\textbf{Custo deste processo}: A partir de 65,00€ (sessenta e cinco euros). \\
\textbf{Entrega de novo passaporte}: No local de pedido ou no domicílio. \\
\textbf{Tempo expectável de execução}: Até 5 (cinco) dias úteis. \\
\\
\newpage
\leavevmode\\
\textbf{Ligações de referência}:
\begin{itemize}
	\item justica.gov.pt, página do passaporte eletrónico, \url{https://justica.gov.pt/Registos/Identificacao/Passaporte-eletronico}
	\item gov.pt, pedir ou renovar o passaporte eletrónico, \url{https://www2.gov.pt/servicos/pedir-o-passaporte-eletronico-portugues}
\end{itemize}

\subsubsection{Carta de condução}

\textbf{Palavras-chave}: conduzir, licença de condução, veículos, viação. \\
\\
Deve efetuar o pedido de uma carta de substituição no portal IMT Online
com a justificação de mudança de nome. O pedido se for dentro de um
prazo razoável coincidente com a renovação da sua carta de condução
poderá ser preferível esperar pelo prazo elegível de renovação que
começa 6 meses antes do prazo de validade. Ao efetuar o pedido através
do IMT Online será gerada uma referência multibanco mas essa só ficará
ativa passadas 24 (vinte e quatro) horas após a submissão do pedido,
isto significa que o pagamento nunca é feito imediatamente, o prazo de
pagamento é de 10 (dez) dias após a submissão do pedido. \\
\\
\textbf{Documento necessário}:
\begin{itemize}
	\item Chave Móvel Digital (CMD) ativa ou acesso do portal da Autoridade Tributária.
\end{itemize}
\leavevmode\\
\textbf{Objetivo}: Obter uma nova carta de condução atualizada. \\
\textbf{Método}: Pedido de substituição por via eletrónica. \\
\textbf{Custo deste processo}: 27,00€ (vinte e sete euros) quando pedido online, 30,00€ (trinta euros) quando presencial. \\
\textbf{Entrega de novo	passaporte}: No domicílio. \\
\textbf{Tempo expectável de execução}: Até 20 (vinte) dias. \\
\\
\textbf{Ligações de referência}:
\begin{itemize}
	\item IMT, a minha carta de condução, \url{https://aminhacartadeconducao.imt-ip.pt/}
	\item IMT Online, carta de condução, \url{https://servicos.imt-ip.pt/Condutores/CartadeCondu\%C3\%A7\%C3\%A3o.aspx}
	\item gov.pt, revalidar Carta de Condução, \url{https://www.gov.pt/servicos/revalidar-a-carta-de-conducao}
	\item gov.pt, substituir Carta de Condução, \url{https://www2.gov.pt/servicos/substituir-a-carta-de-conducao}
	\item gov.pt, 2ª via da Carta de Condução, \url{https://www2.gov.pt/servicos/pedir-a-segunda-via-da-carta-de-conducao}
\end{itemize}
	
	%-------------------
	% ALTERAÇÕES GERAIS
	%-------------------
	%-------------------
% ALTERAÇÕES GERAIS
%-------------------

\newpage

\section{Alterações gerais}

\subsection{Nota introdutória}

Nesta secção são mencionadas todas as restantes alterações a efetuar
após a finalização da secção anterior das ``Alterações iniciais'' e
especialmente da secção de ``Identificação essencial''. Recomenda-se a
finalização da secção de ``Identificação essencial'' antes de prosseguir
com as restantes alterações. Esta secção cobre elementos e locais de
importância variada onde terá de ser o utilizador do roteiro a
determinar a sua prioridade e relevância. A partir daqui os passos e
alterações a executar vão depender das suas necessidades e
circunstâncias individuais. \\
\\
\textbf{Custos diretos totais expectáveis nesta fase}: Variável. \\
\textbf{Tempo total expectável de execução da fase}: Cerca de 40 (quarenta) dias. \\
\\
\textbf{Pré-requisitos desta fase}:
\begin{itemize}
	\item Fase de identificação essencial finalizada;
	\item O seu novo \textbf{Cartão de Cidadão};
	\item Cópia traçada do novo Cartão de Cidadão;
\end{itemize}
\leavevmode\\
\textbf{Objetivos desta fase}:
\begin{itemize}
	\item Atualizar os seus dados pessoais junto do seu empregador;
	\item Atualizar os seus dados pessoais em estabelecimentos de ensino;
	\item Obter um novo Cartão de Aluno;
	\item Obter um novo Certificado de Habilitações;
	\item Obter novas Cartas de Recomendação;
	\item Atualizar o seu Curriculum Vitae;
	\item Atualizar os seus dados em plataformas de emprego;
	\item Atualizar os seus dados pessoais relativo a estabelecimentos ligados a serviços de saúde;
	\item Atualizar os seus dados relativos a serviço ligados à mobilidade;
	\item Obter um novo Certificado de Matrícula;
	\item Obter um novo Passe de Transportes Públicos;
	\item Obter uma nova declaração de início de atividade da Autoridade Tributária;
	\item Atualizar os seus dados pessoais junto de entidades bancárias e financeiras;
	\item Obter um novo Cartão bancário.
\end{itemize}
\leavevmode\\
\newpage
\leavevmode\\
\textbf{Fluxo genérico de acontecimentos desta fase}:
\begin{itemize}
	\item Efetuar um pedido de atualização dos seus dados pessoais:
	\begin{itemize}
		\item No local de trabalho;
		\item No estabelecimento de ensino.
	\end{itemize}
	\item Requerer:
	\begin{itemize}
		\item Um novo Cartão do Aluno;
		\item Um novo Certificado de Habilitações.
	\end{itemize}
	\item Pedir novas Cartas de Recomendação;
	\item Emendar o Curriculum Vitae (CV);
	\item Alterar o nome em plataformas de recrutamento;
	\item Atualizar os seus dados pessoais:
	\begin{itemize}
		\item Em estabelecimentos de saúde;
		\item Em laboratórios de análises clínicas;
		\item Em relatórios médicos;
		\item Aplicações de saúde.
	\end{itemize}
	\item Obter um novo Certificado de Matrícula (DUA);
	\item Atualizar os seus dados pessoais:
	\begin{itemize}
		\item No portal IMT Online;
		\item No portal da ANSR.
	\end{itemize}
	\item Obter um novo Passe de Transportes Públicos;
	\item Atualizar os seus dados pessoais:
	\begin{itemize}
		\item Na Via Verde;
		\item \ldots{} .
	\end{itemize}
	\item Obter uma nova declaração de início de atividade da Autoridade Tributária;
	\item Atualizar os seus dados pessoais junto de entidades bancárias ou financeiras:
	\begin{itemize}
		\item Banco CTT;
		\item Caixa Geral de Depósitos (CGD);
		\item \ldots{} .
	\end{itemize}
	\newpage
	\item Atualizar os seus dados pessoais junto de companhias de seguros:
	\begin{itemize}
		\item Fidelidade;
		\item Logo Seguros;
		\item \ldots{} .
	\end{itemize}
	\item Atualizar os seus dados junto de empresas de telecomunicações:
	\begin{itemize}
		\item MEO;
		\item Vodafone;
		\item \ldots{} .
	\end{itemize}
\end{itemize}

\subsection{Local de trabalho}

\textbf{Palavras-chave}: colaborador, emprego, contrato, trabalhador. \\
\\
Deve entregar no seu local de trabalho os documentos de validação do
novo Cartão de Cidadão que tiver ao seu dispor para que os recursos
humanos (RH) e elementos coordenadores essenciais executem a alteração
dos seus dados nos vários locais, arquivos e serviços associados ao seu
posto de trabalho. \\
\\
A seguinte lista é uma demonstração prática dos locais, arquivos ou
sistemas a alterar:
\begin{itemize}
	\item \textbf{Ficha do trabalhador};
	\item \textbf{Endereço de correio eletrónico} (se aplicável), poderá pedir o encaminhamento do seu antigo endereço por forma a não perder mensagens potencialmente importantes;
	\item \textbf{Seguro de acidentes de trabalho};
	\item \textbf{Seguro de saúde} (se aplicável), poderá ser necessário emitir um novo cartão físico;
	\item \textbf{Sistema de processamento de vencimentos}, tipicamente alterado pela pessoa responsável pelos RH ou administrador de sistema;
	\item \textbf{Cartões ou outros documentos físicos} (conforme aplicável);
	\item \textbf{Sistemas de registo horário} (conforme aplicável), tais como por exemplo relógios de ponto;
	\item \textbf{Sistemas auxiliares de marcação de férias ou aplicações assistentes às funções de RH} (conforme aplicável);
	\item \textbf{Diretórios da empresa ou outros sistemas de autenticação integrada} (se aplicável);
	\item \textbf{Cartões de oferta e/ou refeição} (se aplicável), deve ser pedida a alteração de nome ao abrigo do RGPD ou pedir em alternativa novo cartão consoante o caso (por exemplo: Cartão Dá (Continente) terá de ser pedido novo cartão).
	\item \textbf{Outros sistemas internos que contenham dados de identificação}, outros arquivos ou aplicações onde possam existir dados pessoais sujeitos a retificação.
\end{itemize}
\leavevmode\\
Todas as alterações devem ficar expectavelmente finalizadas dentro de 15
(quinze) dias, no entanto, poderá haver sistemas ou situações onde a
intervenção de terceiros ou múltiplos elementos num único processo
poderá levar a atrasos mas nunca superiores a 30 (trinta) dias. \\
\\
Convém periodicamente ir verificando se todas as alterações já foram
executadas por forma a poder reforçar o pedido se necessário ou efetuar
uma intervenção alternativa para garantir a execução adequada do seu
pedido. \\
\\
\textbf{Documento necessário}:
\begin{itemize}
	\item Cópia traçada do novo Cartão de Cidadão ou Cópia do pedido de novo Cartão de Cidadão.
\end{itemize}
\leavevmode\\
\textbf{Objetivo}: Obter uma nova declaração de início de atividade atualizada. \\
\textbf{Método}: Pedido de alteração da atividade no portal das Finanças. \\
\textbf{Custo deste processo}: Gratuito. \\
\textbf{Tempo expectável de execução}: Até 30 (trinta) dias. \\

\subsection{Educação/Emprego}

\subsubsection{Universidades, escolas e outros estabelecimentos de ensino}

\textbf{Palavras-chave}: educação, ensino, aulas, educacional. \\
\\
Estando ainda a frequentar um estabelecimento de ensino ou formação
poderá deslocar-se à secretaria desse estabelecimento e efetuar o seu
pedido presencialmente, bastando assim levar o seu novo documento de
identificação e referir a necessidade de atualização dos seus dados no
seguimento de alteração do seu documento de identificação.\\
\\
\textbf{Documento necessário}:
\begin{itemize}
	\item Cópia traçada do novo Cartão de Cidadão.
\end{itemize}
\leavevmode\\
\textbf{Objetivo}: Atualização do dados do aluno/formando. \\
\textbf{Custo deste processo}: Gratuito. \\
\textbf{Tempo expectável de execução}: Imediato. \\

\paragraph{Cartão do Aluno}
\leavevmode\\[4pt]
Poderá ser necessária a emissão de um novo Cartão do Aluno ou a
retificação em mais que um sistema em algumas circunstâncias específicas
ou estabelecimentos, podendo assim representar um custo diferenciado e
tempo de execução superior ao expectável. \\
\\
\textbf{Documento necessário}:
\begin{itemize}
	\item Cópia traçada do novo Cartão de Cidadão.
\end{itemize}
\leavevmode\\
\textbf{Objetivo}: Obter um novo Cartão do Aluno ou atualizar os seus dados. \\
\textbf{Custo deste processo}: Variável, a rondar os 7,00€ (sete euros). \\
\textbf{Tempo expectável de execução}: Até 15 (quinze) dias. \\
\\
\subsubsection{Certificado de Habilitações}

\textbf{Palavras-chave}: diploma, qualificações, aprendizagem, escolaridade, certificação. \\
\\
Para obter um novo certificado de habilitações, terá de contactar o
estabelecimento de ensino ou formação utilizando o modelo de carta,
disponibilizado nos anexos, onde poderá adaptar, assinar digitalmente e
enviar através de mensagem de correio eletrónico, por correio registado
ou entregar em mãos conforme o que lhe for mais conveniente. Para que a
sua identidade seja verificada corretamente deverá anexar uma cópia do
seu novo documento para que sejam confirmadas as alterações mencionadas
na carta. Poderá ser útil ligar primeiro para o estabelecimento por
forma a conhecer qual o endereço de correio eletrónico mais adequado
para remeter o seu pedido. \\
\\
\textbf{Documentos necessários}:
\begin{itemize}
	\item Modelo de carta, pedido de 2ª via do Certificado de Habilitações;
	\item Cópia traçada do novo Cartão de Cidadão.
\end{itemize}
\leavevmode\\
\textbf{Objetivo}: Obter uma 2ª (segunda) via do Certificado de Habilitações. \\
\textbf{Método}: Entrega de carta física ou digital a partir do modelo. \\
\textbf{Custo deste processo}: Gratuito ou pago dependendo do estabelecimento. \\
\textbf{Entrega do novo certificado}: Virtual ou físico dependendo do estabelecimento. \\
\textbf{Tempo expectável de execução}: Até 15 (quinze) dias. \\
\\
\textbf{Ligações de referência}:
\begin{itemize}
	\item Secretaria-Geral da Educação e Ciência, certificados de habilitações, perguntas frequentes, \url{https://www.sec-geral.mec.pt/faqs/587}
	\item Doutor Finanças, como obter um certificado de habilitações, \url{https://www.doutorfinancas.pt/carreira-e-rendimentos/emprego/como-obter-um-certificado-de-habilitacoes/}
\end{itemize}

\subsubsection{Cartas de recomendação}

Consoante as suas circunstâncias específicas e pessoais poderá ser útil
pedir a retificação ou elaboração de novas cartas de recomendação com o
seu novo nome de modo a poder anexá-las ao seu Currículo Vitae e
garantir a consistência e coesão das suas referências e experiências
profissionais. \\
\\
\textbf{Objetivo}: Obter cartas de recomendação atualizadas. \\
\textbf{Custo deste processo}: Gratuito. \\
\\
\textbf{Ligação de referência}:
\begin{itemize}
	\item NValores, exemplos de cartas de recomendação, \url{https://www.nvalores.pt/exemplos-de-cartas-de-recomendacao/}
\end{itemize}

\subsubsection{Curriculum Vitae (CV)}

\textbf{Palavras-chave}: currículo, experiências, cv, trajeto
profissional, escolaridade. \\
\\
Deve atualizar o seu currículo com o novo nome e possivelmente todas as
suas experiências (se não o tiver feito) independentemente do seu estado
profissional atual, isto poderá implicar a mudança de pronomes ou nomes
com género onde a alteração poderá ser relevante. Esta alteração é
especialmente importante dando-lhe um ponto de partida para procura de
emprego se vier a necessitar de o fazer. A atualização do currículo é
imperativo como uma ferramenta de procura de trabalho e auxiliar na
eventualidade de imprevistos. Se não o souber fazer deverá procurar
ajuda de um profissional para auxiliar na composição e atualização do
documento. Há empregadores que preferem o modelo Europass e outros não,
deve utilizar o seu julgamento para adequar-se às necessidades e
potenciais empregadores. \\
\\
\textbf{Objetivo}: Obter um Curriculum Vitae com os dados atualizados. \\
\textbf{Método}: Alteração manual dos seus dados. \\
\textbf{Custo deste processo}: Gratuito. \\
\textbf{Tempo expectável de execução}: Imediato. \\
\\
\textbf{Ligações de referência}:
\begin{itemize}
	\item Europa, sobre criar um CV Europass, \url{https://europass.europa.eu/pt/create-europass-cv}
	\item Europa, criar um CV Europass, \url{https://europa.eu/europass/eportfolio/screen/cv-editor?lang=pt}
	\item Canva, criar um CV, \url{https://www.canva.com/pt\_pt/criar/cv/}
\end{itemize}

\subsubsection{Plataformas de emprego/recrutamento}

\textbf{Palavras-chave}: candidaturas, emprego, trabalho, recrutadores. \\
\\
Deve atualizar o seu nome e documentos carregados, nomeadamente
currículos, em plataformas de emprego para que potenciais recrutadores
tenham a sua informação devidamente atualizada. O processo de alteração
é muito variável consoante as plataformas mas é possível que possa
alterar o seu nome através das definições de perfil de utilizador da
plataforma em questão. Listam-se alguns exemplos de plataformas de
emprego: Sapo Emprego, Net-empregos, Michael Page, Landing.jobs, IT
Jobs, Indeed\ldots{} \\
\\
\textbf{Objetivo}: Atualizar o nome e género em plataformas de emprego. \\
\textbf{Método}: Alteração manual dos seus dados. \\
\textbf{Custo deste processo}: Gratuito. \\
\textbf{Tempo expectável de execução}: Imediato. \\

\newpage

\subsubsection{Sugestões de contribuição}

Se souber ou quiser contribuir com outros processos listam-se algumas
sugestões ou exemplos de contribuições:
\begin{itemize}
	\item Universidade de Lisboa;
	\item Universidade Aberta;
	\item Universidade Lusófona;
	\item Instituto do Emprego e Formação Profissional (IEFP);
	\item Bolsa de Emprego Público (BEP);
	\item Plataforma de recrutamento do Instituto dos Registos e Notariado (IRN);
	\item \ldots{} .
\end{itemize}

\subsection{Saúde}

\subsubsection{Hospitais e clínicas públicas e/ou privadas}

\textbf{Palavras-chave}: hospital, clínica, cuidados de saúde. \\
\\
Deve verificar no portal internet do estabelecimento (se este existir) a
existência da possibilidade de retificação dos seus dados pessoais. Em
alternativa deve através de comunicação eletrónica, enviar uma mensagem
de correio eletrónico, a efetuar o pedido de retificação dos seus dados
ao abrigo do RGPD. Na eventualidade de não conseguir através de uma das
opções anteriores, deverá pedir a retificação/alteração dos seus dados
no próprio estabelecimento. É importante notar que clínicas podem ser
dentárias, estéticas, integradas ou especializadas diversas. Exemplos:
Hospitais da CUF, Luz ou Lusíadas. \\
\\
\textbf{Documentos necessários}:
\begin{itemize}
	\item Modelo de carta, pedido de alteração/retificação de dados pessoais;
	\item Novo Cartão de Cidadão.
\end{itemize}
\leavevmode\\
\textbf{Objetivo}: Atualizar os seus dados num estabelecimento de saúde. \\
\textbf{Método}: Pedido presencial no balcão do estabelecimento. \\
\textbf{Custo deste processo}: Gratuito. \\
\textbf{Tempo expectável de	execução}: Imediato. \\

\newpage

\subsubsection{Laboratórios de análises clínicas}

\textbf{Palavras-chave}: laboratório, análises clínicas, níveis séricos, centros de saúde. \\
\\
Se estiver a efetuar uma transição médica é provável que recorra a um ou
mais laboratórios de análises clínicas. É importante que seja feita a
alteração do marcador nem que seja por uma questão de consistência da
informação e níveis séricos de referência que são baseados no sexo. O
pedido pode ser feito através de comunicação eletrónica ao laboratório
que utiliza. Exemplo prático, se efetuar as suas análises através da CUF
cujo o fornecedor de serviços de laboratório é a Germano de Sousa os
seus dados irão ser atualizados eventualmente por encaminhamento da CUF,
ou seja, ainda que na CUF possa já ter os dados atualizados, os dados do
laboratório só são atualizados numa data posterior. Poderá no entanto
pedir sempre a sua verificação se assim o entender. \\
\\
\textbf{Documentos necessários}:
\begin{itemize}
	\item
	Modelo de carta, pedido de alteração/retificação de dados pessoais;
	\item
	Cópia traçada do Cartão de Cidadão.
\end{itemize}
\leavevmode\\
\textbf{Objetivo}: Atualizar os seus dados num laboratório de análises
clínicas. \\
\textbf{Método}: Pedido através de comunicação eletrónica. \\
\textbf{Custo deste processo}: Gratuito. \\
\textbf{Tempo expectável de	execução}: Até 20 (vinte) dias. \\

\subsubsection{Relatórios médicos, diagnósticos e outros instrumentos de utilidade médica e diagnóstica permanentes}

\textbf{Palavras-chave}: relatório, diagnóstico. \\
\\
Deverá sempre que necessário e possível pedir a retificação do nome em
relatórios, diagnósticos e outros documentos com utilidade e/ou validade
vitalícia ao médico, entidade ou organização responsável pelo relatório.
Poderá invocar os seus direitos no âmbito do RGPD para maior facilidade
na obtenção de resposta ao pedido. \\
\\
\textbf{Documentos necessários}:
\begin{itemize}
	\item
	Modelo de carta, pedido de 2ª (segunda) via de Relatório Médico ou
	Avaliação Psicológica;
	\item
	Cópia traçada do novo Cartão de Cidadão.
\end{itemize}
\leavevmode\\
\textbf{Objetivo}: Obter novo relatório médico. \\ 
\textbf{Método}: Pedido por via eletrónica. \\
\textbf{Custo deste processo}: Gratuito, depende do caso. \\ 
\textbf{Tempo expectável de execução}: Até 30 (trinta) dias.

\newpage

\subsubsection{My Cuf}

\textbf{Palavras-chave}: cuf, hospital, saúde privada. \\
\\
É necessário cancelar o acesso atual e recriar novamente com novo
endereço por forma a poder alterar o endereço de correio eletrónico de
acesso. Para isso é preciso aceder à página do centro de apoio ao
cliente, selecionar o formulário de contacto. selecionar o separador my
cuf, apoio, preencher o formulário e indicar a principal unidade que
visita, na mensagem poderá escrever o seguinte: ``Boa tarde, pede-se o
cancelamento do acesso ao Portal My CUF com o nome utilizador
{[}NOME-DO-UTILIZADOR{]}, por resultado da alteração e descontinuação da
utilização dessa caixa de correio eletrónico. No entanto se for possível
apenas a alteração pede-se que o novo nome de utilizador seja
{[}NOVO-NOME-DE-UTILIZADOR{]}. Esta comunicação é feita de acordo com o
direito previsto no artigo 16.º (décimo sexto) do RGPD. Agradeço desde
já a atenção prestada.''\\
\\
\textbf{Objetivo}: Atualizar o seu nome de utilizador. \\
\textbf{Método}: Pedido por via eletrónica. \\
\textbf{Custo deste processo}: Gratuito. \\
\textbf{Tempo expectável de execução}: Até 15 (quinze) dias. \\
\\
\textbf{Ligações de referência}:
\begin{itemize}
	\item CUF, centro de apoio ao cliente questões relativas ao My CUF, \url{https://www.cuf.pt/centro-de-apoio-ao-cliente?faq\_categoria=66651}
	\item CUF, formulário específico de contacto, \url{https://www.cuf.pt/centro-de-apoio-ao-cliente\#66841673}
	\item CUF, centro de apoio ao cliente, \url{https://www.cuf.pt/centro-de-apoio-ao-cliente}
	\item Google Play Store, aplicação móvel para sistema Android, \url{https://play.google.com/store/apps/details?id=pt.saudecuf.myCUF\&hl=pt\_PT}
	\item Apple App Store, aplicação móvel para sistema iOS, \url{https://apps.apple.com/pt/app/my-cuf/id811304952}
\end{itemize}

\newpage

\subsubsection{Sugestões de contribuição}

Se souber ou quiser contribuir com outros processos listam-se algumas
sugestões ou exemplos de contribuições:
\begin{itemize}
	\item Clínica Santa Madalena;
	\item Lusíadas Saúde;
	\item HeyDoc;
	\item My Luz;
	\item Kardia;
	\item Unilabs;
	\item Joaquim Chaves;
	\item Trofa Saúde;
	\item Germano de Sousa;
	\item {\ldots} .
\end{itemize}

\subsection{Mobilidade}

\subsubsection{Documento Único Automóvel (DUA)}

\textbf{Palavras-chave}: automóvel, conservatória, dua, certificado de matrícula, registo. \\
\\
Deve efetuar o seu pedido presencialmente num Registo Automóvel ou numa
Loja do Cidadão. Efetuando o seu pedido online através do portal
Automóvel Online irá poupar 5,25€ pela razão do pedido ser feito
eletronicamente. Para proceder ao pedido do registo através do portal,
deve efetuar os seguintes passos:
\begin{itemize}
	\item Inserir o seu Cartão de Cidadão no leitor de cartões;
	\item Aceder ao portal internet Automóvel Online;
	\item Selecionar o seu certificado do Cartão de Cidadão;
	\item Introduzir o PIN de acesso do Cartão de Cidadão;
	\item Selecionar a opção ``Outros Pedidos'';
	\item Selecionar a opção de ``Alteração de Nome'';
	\item Preencher e submeter o respetivo pedido adequadamente.
\end{itemize}
\leavevmode\\
\newpage
\leavevmode\\
\textbf{Documento necessário}:
\begin{itemize}
	\item Novo Cartão de Cidadão (e leitor de cartões).
\end{itemize}
\leavevmode\\
\textbf{Objetivo}: Obter um novo Certificado de Matrícula. \\
\textbf{Método}: Pedido presencial ao balcão ou eletronicamente através do portal Automóvel Online. \\
\textbf{Custo deste processo}: 35,00€ (trinta e cinco euros) se presencial ou 29,80€ (vinte e nove euros e oitenta cêntimos) se por via eletrónica. \\
\textbf{Entrega do novo	Certificado de Matrícula}: No domicílio. \\
\textbf{Tempo expectável de	execução}: Até 45 (quarenta e cinco) dias. \\
\\
\textbf{Ligações de referência}:
\begin{itemize}
	\item Automóvel Online, entrada, \url{https://www.automovelonline.mj.pt/AutoOnlineProd/}
	\item Automóvel Online, outros pedidos, \url{https://www.automovelonline.mj.pt/AutoOnlineProd/conteudos/listaPedidos.jsp?num=8\#8}
	\item Automóvel Online, consultar pedidos, \url{https://www.automovelonline.mj.pt/AutoOnlineProd/Contribuintes/ContribuintesController?action=pedidosconsulta}
	\item Google Play Store, aplicação DUApp para sistema Android, \url{https://play.google.com/store/apps/details?id=incm.com.leitor.dua\&hl=pt\_PT}
	\item Apple App Store, aplicação DUApp para sistema iOS, \url{https://apps.apple.com/pt/app/duapp/id1474224031}
\end{itemize}

\subsubsection{Instituto da Mobilidade e dos Transportes (IMT)}

\textbf{Palavras-chave}: imt, mobilidade, etc. \\
\\
Efetuado automaticamente após o pedido de novo cartão de cidadão. O
processo é finalizado instantaneamente sem ser necessária intervenção.
Inclui o portal IMT Online. \\
\\
\textbf{Documento necessário}:
\begin{itemize}
	\item Chave Móvel Digital ativa ou acesso do portal da Autoridade Tributária.
\end{itemize}
\leavevmode\\
\textbf{Objetivo}: Atualização dos seus dados junto do IMT. \\
\textbf{Método}: Automático. \\
\textbf{Custo deste processo}: Gratuito. \\
\textbf{Tempo expectável de execução}: Imediato. \\
\\
\textbf{Ligações de referência}:
\begin{itemize}
	\item IMT Online, acesso ao portal, \url{https://servicos.imt-ip.pt/login.aspx?ReturnUrl=\%2fdefault.aspx}
	\item IMT Online, consultar o registo no portal, \url{https://servicos.imt-ip.pt/RegistonoPortal.aspx}
	\item IMT Online, consultar lista de pedidos, \url{https://servicos.imt-ip.pt/ListadePedidos.aspx}
\end{itemize}

\subsubsection{Autoridade Nacional de Segurança Rodoviária (ANSR)}

\textbf{Palavras-chave}: portal das contraordenações, multas, pontos, carta de condução. \\
\\
Deve alterar os dados de registo acedendo ao portal das contraordenações
com o seu acesso habitual e efetuar o pedido através do menu ``alterar
dados''. As alterações são imediatas após a submissão do pedido no
portal.\\
\\
\textbf{Documento necessário}:
\begin{itemize}
	\item Chave Móvel Digital (CMD) ativa ou acesso do portal da Autoridade Tributária.
\end{itemize}
\leavevmode\\
\textbf{Objetivo}: Atualização dos seus dados junto da ANSR. \\
\textbf{Método}: Pedido de alteração através do portal. \\
\textbf{Custo deste processo}: Gratuito. \\
\textbf{Tempo expectável de execução}: Imediato. \\
\\
\textbf{Ligações de referência}:
\begin{itemize}
	\item ANSR, acesso ao portal das contraordenações, \url{https://portalcontraordenacoes.ansr.pt/\_layouts/pages/login.aspx?ReturnUrl=\%2f\_layouts\%2fAuthenticate.aspx\%3fSource\%3d\%252F\%255Flayouts\%252Fpages\%252Fdefault\%252Easpx\&Source=\%2F\%5Flayouts\%2Fpages\%2Fdefault\%2Easpx}
	\item ANSR, portal das contraordenações, alterar dados, \url{https://portalcontraordenacoes.ansr.pt/\_layouts/Pages/AlterarDados.aspx}
	\item ANSR, portal das contraordenações, consulta de processos, \url{https://portalcontraordenacoes.ansr.pt/\_Layouts/Pages/ListaProcessos.aspx}
	\item ANSR, portal das contraordenações, consulta de pontos, \url{https://portalcontraordenacoes.ansr.pt/PortalCO/\_Layouts/Pages/ConsultaPontos.aspx}
\end{itemize}

\subsubsection{Passe Transportes Públicos da AML (Navegante)}

\textbf{Palavras-chave}: passe, comboios, autocarros, mobilidade. \\
\\
Efetuar um pedido de novo passe numa bilheteira de qualquer um dos
operadores de transportes públicos, com o comprovativo de pedido de novo
cartão de cidadão. Será necessário uma foto tipo passe que deverá
entregar obrigatoriamente junto com o seu pedido. \\
\\
\textbf{Documentos necessários}:
\begin{itemize}
	\item Formulários, Anexo D - Requisição de Passe Navegante (TML);
	\item Uma foto recente tipo passe;
	\item Cópia traçada do novo Cartão de Cidadão ou cópia do pedido de novo Cartão de Cidadão.
\end{itemize}
\leavevmode\\
\textbf{Objetivo}: Obter um novo passe de transportes atualizado. \\
\textbf{Método}: Pedido de novo cartão junto de um operador de transportes. \\
\textbf{Custo deste processo}: A partir de 7,00€ (sete euros). \\
\textbf{Entrega de novo passe}: Fisicamente numa bilheteira do operador de transportes ou entregue no domicílio. \\
\textbf{Tempo expectável de execução}: Até 15 (quinze) dias. \\
\\
\textbf{Ligações de referência}:
\begin{itemize}
	\item Navegante, cartões, \url{https://www.navegante.pt/viajar/cartoes}
	\item Navegante, loja online, \url{https://loja.navegante.pt/}
	\item Navegante, loja online, cartão personalizado, \url{https://loja.navegante.pt/product/cartao-navegante-personalizado}
	\item Google Play Store, aplicação móvel para sistema Android, \url{https://play.google.com/store/apps/details?id=pt.card4b.navegante\&hl=pt\_PT}
	\item Apple App Store, aplicação móvel para sistema iOS, \url{https://apps.apple.com/pt/app/navegante/id6484591306}
\end{itemize}

\subsubsection{Via Verde}

\textbf{Palavras-chave}: mobilidade, portagens, identificador. \\
\\
Deve enviar uma mensagem de correio eletrónico para
\href{mailto:cliente@viaverde.pt}{\nolinkurl{cliente@viaverde.pt}}
através do endereço de correio eletrónico de registo no portal com o
novo documento de identificação, poderá utilizar a exportação dos seus
dados para um ficheiro em formato \emph{PDF} a partir da aplicação móvel
\emph{gov.pt} em combinação com a certidão de morada fiscal da
Autoridade Tributária ou o ficheiro exportado da aplicação \emph{gov.pt}
após receção e ativação do novo cartão de cidadão. Em alternativa, deve
ser enviada carta registada para a sede da empresa a efetuar o pedido
com a documentação adequada em anexo. É pedido tipicamente que seja
enviado uma cópia do Cartão de Cidadão mas esse pedido não tem
sustentação legal adequada. Nem sempre é possível conseguir que a
alteração seja executada no primeiro pedido. \\
\\
\textbf{Documentos necessários}:
\begin{itemize}
	\item Modelo de carta, Pedido de alteração/retificação de dados pessoais;
	\item Cópia traçada do novo Cartão de Cidadão.
\end{itemize}
\leavevmode\\
\textbf{Objetivo}: Atualização dos seus dados de cliente junto da Via Verde. \\
\textbf{Método}: Pedido de retificação dos seus dados ao abrigo do RGPD por via eletrónica. \\
\textbf{Custo deste processo}: Gratuito. \\
\textbf{Tempo expectável de execução}: Até 15 (quinze) dias. \\
\\
\textbf{Ligações de referência}:
\begin{itemize}
	\item Via Verde, entrada, \url{https://www.viaverde.pt/particulares/minha-via-verde/}
	\item Via Verde, contratos, \url{https://www.viaverde.pt/particulares/minha-via-verde/contratos}
	\item Google Play, aplicação para sistema Android, \url{https://play.google.com/store/apps/details?id=pt.viaverde.clientes\&hl=pt\_PT}
	\item Apple App store, aplicação para sistema iOS, \url{https://apps.apple.com/pt/app/via-verde/id674583357}
\end{itemize}

\subsubsection{Miio}

\textbf{Palavras-chave}: carregamento elétrico, cartão. \\
\\
Pode alterar o nome das definições de utilizador através da aplicação
móvel. Poderá ser necessário solicitar um novo cartão. A alteração no
portal/aplicação é imediata. \\
\\
\textbf{Objetivo}: Atualização dos seus dados junto do Miio. \\
\textbf{Método}: Alteração dos dados na aplicação móvel. \\
\textbf{Custo deste processo}: Gratuito, ou 30,00€ (trinta euros) para o cartão. \\
\textbf{Tempo expectável de execução}: Imediato, 2 (duas) semanas para o cartão. \\
\\
\textbf{Ligações de referência}:
\begin{itemize}
	\item Miio, entrada, \url{https://www.miio.com/pt}
	\item Miio, aplicação móvel para o navegador de ambiente de trabalho, \url{https://app.miio.com/}
	\item Miio, aplicação móvel para sistema Android, \url{https://play.google.com/store/apps/details?id=com.muvext.miio}
	\item Miio, aplicação móvel para sistema iOS, \url{https://apps.apple.com/us/app/miio/id1462182013?pt=1462182013\&ct=homepage\&mt=8}
\end{itemize}

\subsubsection{Waze}

\textbf{Palavras-chave}: mapas, gps, app, mobilidade. \\
\\
Para atualizar o seu nome na aplicação móvel de navegação Waze deve
aceder ao menu, selecionar a opção o ``meu waze'' e tocar no nome de
utilizador, aí então poderá alterar o nome. A alteração é imediata. Na
eventualidade de também utilizar o mesmo nome de utilizador nos fóruns é
importante notar que irá perder o acesso às funcionalidades associadas
ao nome de utilizador anterior. \\
\\
\textbf{Ligação de referência}:
\begin{itemize}
	\item Waze, centro de ajuda, definir nome de utilizador, \url{https://support.google.com/waze/answer/6268711?hl=pt}
\end{itemize}

\subsubsection{Tesla}

\textbf{Palavras-chave}: automóveis, elétricos. \\
\\
Deve alterar o nome nas definições de conta do utilizador, ficará
alterado imediatamente após a submissão. Se tiver dados de faturação
associados terá de alterá-los posteriormente nos dados de faturação no
seguimento de uma compra ou pedido de assistência. \\
\\
\textbf{Ligação de referência}:
\begin{itemize}
	\item Tesla, atualizar conta, \url{https://www.tesla.com/pt\_pt/support/how-create-update-delete-tesla-account}
\end{itemize}

\subsubsection{Sugestões de contribuição}

Se souber ou quiser contribuir com outros processos listam-se algumas
sugestões de contribuição: \\
\\
\begin{itemize}
	\item Passe Andante;
	\item Carris;
	\item Metropolitano de Lisboa;
	\item Comboios de Portugal, Cartão CP;
	\item \ldots{} .
\end{itemize}

\subsection{Fiscalidade}

\subsubsection{Declaração de início de atividade da Autoridade Tributária (AT)}

Para retificar o nome deve aceder ao portal da finanças com o seu acesso
e entregar uma declaração de alteração de atividade, com essa entrega o
documento comprovativo de alteração já terá o novo nome associado. Se
não souber como efetuar essa entrega deverá consultar um contabilista ou
recurso adequado para o efeito. \\
\\
\textbf{Objetivo}: Obter uma nova declaração de início de atividade atualizada. \\
\textbf{Método}: Pedido de alteração da atividade no Portal das Finanças. \\ 
\textbf{Custo deste processo}: Gratuito. \\
\textbf{Tempo expectável de execução}: Imediato. \\
\\
\textbf{Ligações de referência}:
\begin{itemize}
	\item Autoridade Tributária, acesso, \url{https://www.acesso.gov.pt/v2/loginForm?partID=PFAP\&path=/geral/dashboard}
	\item Autoridade Tributária, declaração de atividade, \url{https://sitfiscal.portaldasfinancas.gov.pt/atividade/atividade/entregar}
\end{itemize}

\subsubsection{Sugestões de contribuição}

Se souber ou quiser contribuir com outros processos listam-se algumas
sugestões de contribuição:
\begin{itemize}
	\item Outras certidões ou declarações emitidas ou geridas pela Autoridade Tributária (AT);
	\item \ldots{} .
\end{itemize}
\leavevmode\\
\newpage

\subsection{\texorpdfstring{Entidades bancárias, financeiras e \emph{fintech}}{Entidades bancárias, financeiras e fintech}}

\subsubsection{Banco CTT}

Para atualizar os seus dados pessoais junto do Banco CTT deve
deslocar-se a uma agência e retirar uma senha, é importante ter o Cartão
de Cidadão consigo. A alteração é feita no momento mas poderá demorar
até 48 horas para que seja totalmente processada. Aparentemente não é
possível a atualização dos dados por via eletrónica. \\
\\
A atualização do nome no MBWay relativo ao cartão só é atualizado 
com o pedido de novo cartão. \\
\\
\textbf{Documento necessário}:
\begin{itemize}
	\item Novo Cartão de Cidadão.
\end{itemize}

\textbf{Objetivo}: Atualização dos seus dados junto do Banco CTT. \\
\textbf{Método}: Pedido presencial numa agência do Banco CTT. \\
\textbf{Tempo expectável de execução}: Até 48 (quarenta e oito) horas. \\
\textbf{Custo de novo cartão}: Desde 18,50€ (dezoito euros e cinquenta cêntimos). \\

\subsubsection{Caixa Geral de Depósitos (CGD)}

Para atualizar os seus dados junto da Caixa Geral de Depósitos (CGD)
deve aceder à sua conta através do portal de \emph{homebanking} e
fazendo recurso da sua Chave Móvel Digital (CMD) efetuar atualização dos
seus dados. A alteração tem efeitos imediatos. \\
\\
\textbf{Documento necessário}:
\begin{itemize}
	\item Chave Móvel Digital (CMD) ativa.
\end{itemize}
\leavevmode\\
\textbf{Objetivo}: Atualização dos seus dados pessoais junto da CGD. \\
\textbf{Método}: Atualização dos dados via Chave Móvel Digital (CMD). \\
\textbf{Tempo expectável de execução}: Imediato. \\
\textbf{Custo de novo cartão}: Desde 18,00€ (dezoito euros).

\subsubsection{ActivoBank (AB)}

Deve atualizar os dados entrando na aplicação do ActivoBank fazendo
recurso da sua Chave Móvel Digital (CMD), a alteração deverá ser
imediata mas em alguns casos poderá demorar algumas horas ou até mesmo
dar como recusada ainda que posteriormente seja então processada com
sucesso. \\
\\
\textbf{Documento necessário}:
\begin{itemize}
	\item Chave Móvel Digital (CMD) ativa.
\end{itemize}
\leavevmode\\
\textbf{Objetivo}: Atualização dos seus dados pessoais junto do ActivoBank. \\
\textbf{Método}: Atualização dos dados via Chave Móvel Digital (CMD). \\
\textbf{Tempo expectável de execução}: Até 24 (vinte e quatro) horas. \\
\textbf{Custo de novo cartão}: Desde 15,00€ (quinze euros).

\subsubsection{Moey}

Deve aceder à sua conta Moey e proceder à atualização dos seus dados via
Chave Móvel Digital (CMD), poderá ocorrer um erro na atualização, no
entanto os seus dados irão aparecer corrigidos após algumas horas. A
alteração deve ter efeito imediato mas nem sempre acontece. \\
\\
\textbf{Documento necessário}:
\begin{itemize}
	\item Chave Móvel Digital (CMD) ativa.
\end{itemize}
\leavevmode\\
\textbf{Objetivo}: Atualização dos seus dados pessoais junto do Moey. \\
\textbf{Método}: Atualização dos dados via Chave Móvel Digital (CMD). \\
\textbf{Tempo expectável de execução}: Até 12 (doze) horas. \\
\textbf{Custo de novo cartão}: 5,00€ (cinco euros). \\
\\
\textbf{Ligação de referência}:
\begin{itemize}
	\item Moey, ajuda com os dados pessoais, \url{https://support.moey.pt/hc/pt-pt/categories/360003029418-Conta-e-Perfil}
\end{itemize}

\subsubsection{N26}

Deve abrir uma nova conversação (chat) e pedir ao assistente a mudança
de nome e sexo no seu registo, o assistente irá enviar uma mensagem na
aplicação móvel onde então deverá anexar o novo Cartão de Cidadão e nova
Certidão de Nascimento. A alteração poderá levar alguns dias a ser
finalizada. \\
\\
\textbf{Documentos necessários}:
\begin{itemize}
	\item Cópia traçada do novo Cartão de Cidadão;
	\item Cópia da Certidão/Assento de Nascimento.
\end{itemize}
\leavevmode\\
\textbf{Objetivo}: Atualização dos seus dados pessoais junto do N26. \\
\textbf{Método}: Envio de pedido de suporte com anexação do CC e Certidão de Nascimento. \\
\textbf{Tempo expectável de execução}: Até 15 (quinze) dias. \\ 
\textbf{Custo de novo cartão}: Desde 10,00€ (dez euros). \\
\\
\textbf{Ligação de referência}:
\begin{itemize}
	\item N26, alterar o telemóvel, morada ou outros dados pessoais, \url{https://support.n26.com/en-eu/account-and-personal-details/personal-information-and-data/change-my-phone-number-address-or-other-personal-data}
\end{itemize}

\newpage

\subsubsection{Wise}

Para alterar os seus dados junto do Wise deve seguir os seguintes
passos: Após efetuar o início de sessão no portal deve aceder a
\url{https://wise.com/help/contact/flows/general/what-do-you-need-to-change/my-name}
onde poderá fazer o carregamento de um documento comprovativo da mudança
de nome. A alteração poderá não ser aceite na primeira tentativa.
Tipicamente finalizado dentro de 15 dias. Poderá ser necessário pedir um
segundo ticket ou ajuda no chat para retificar o \emph{wisetag}
associado à conta. \\
\\
\textbf{Documento necessário}:
\begin{itemize}
	\item Cópia traçada do novo Cartão de Cidadão.
\end{itemize}
\textbf{Objetivo}: Atualização dos seus dados pessoais junto do Wise. \\
\textbf{Método}: Envio de pedido de suporte anexando o CC. \\
\textbf{Tempo expectável de execução}: Até 15 (quinze) dias. \\
\textbf{Custo novo cartão}: Desde 6,00€ (seis euros).

\subsubsection{Revolut}

Para alterar o seu nome na aplicação móvel da Revolut deve seguir os
seguintes passos: \\
\begin{itemize}
	\item Tocar no ícone do seu perfil no canto superior esquerdo;
	\item Selecionar ``Conta'' e ``Dados pessoais'';
	\item Tocar no seu nome para o editar e seguir as instruções.
\end{itemize}
\leavevmode\\
\textbf{Objetivo}: Atualização dos seus dados pessoais junto da Revolut. \\
\textbf{Método}: Envio de pedido de suporte anexando o CC. \\
\textbf{Tempo expectável de execução}: Até 7 (sete) dias. \\
\textbf{Custo de novo cartão}: Desde 6,00€ (seis euros).\\
\\
\textbf{Ligação de referência}:
\begin{itemize}
	\item Revolut, alterar o meu nome, \url{https://help.revolut.com/pt-PT/help/profile-and-plan/profile-plan/profile-settings/how-do-i-change-my-name/}
\end{itemize}

\subsubsection{MBWay}

Deve alterar os dados no perfil de utilizador, a alteração tem efeito
imediato após a submissão. O nome que surge quando alguém lhe envia
dinheiro poderá estar dependente do nome associado ao cartão bancário,
isso significa que que poderá ser necessário primeiro pedir um novo
cartão antes deste ficar devidamente atualizado.

\subsubsection{Outras instituições bancárias, de crédito, financeiras ou relacionadas}

Se dispor de acesso a um portal da instituição deve procurar a
possibilidade de atualização de dados através desse. Se não for possível
deve procurar o contacto e enviar uma mensagem de correio eletrónico com
o pedido de alteração invocando os direitos à luz do RGPD. Em
alternativa secundária poderá deslocar-se a um estabelecimento da
instituição para proceder à alteração com o seu novo Cartão de Cidadão.

\subsubsection{Cartões bancários}

Deve pedir sempre que aplicável e assim que for conveniente um cartão de
substituição por cada onde tenha referência ao nome anterior, os valores
dependem da instituição bancária mas tipicamente até 20,00€ (vinte
euros) por cada cartão. O tempo de finalização varia consoante as
emissões de novos cartões.

\subsubsection{Sugestões de contribuição}

Se souber ou quiser contribuir com outros processos listam-se algumas
sugestões de contribuição:
\begin{itemize}
	\item Novo Banco;
	\item Santander;
	\item Millenium BCP;
	\item Montepio;
	\item Crédito Agrícola;
	\item BBVA;
	\item Cofidis;
	\item Cetelem;
	\item Oney;
	\item Credibom;
	\item Unicre;
	\item \ldots{} .
\end{itemize}

\subsection{Seguros}

\subsubsection{Fidelidade}

Deve efetuar um pedido junto do mediador da alteração de dados pessoais
com os novos documentos através de comunicação eletrónica, como mensagem
WhatsApp ou mensagem de correio eletrónico, ou deslocando-se a
fisicamente ao mediador. O processo é tipicamente finalizado em 3 dias. \\
\\
\textbf{Documento necessário}:
\begin{itemize}
	\item Cópia traçada do novo Cartão de Cidadão.
\end{itemize}
\leavevmode\\
\textbf{Objetivo}: Atualização dos seus dados pessoais junto da Fidelidade. \\ 
\textbf{Método}: Envio de pedido de retificação dos dados com CC anexo. \\
\textbf{Tempo expectável de execução}: Até 15 (quinze) dias. \\
\\
\textbf{Ligação de referência}:
\begin{itemize}
	\item Fidelidade, aplicação MyFidelidade, \url{https://www.fidelidade.pt/PT/particulares/Paginas/MyFidelidade.aspx}
\end{itemize}

\subsubsection{Logo Seguros}

Deve efetuar novo pedido no portal da Logo com o assunto ``direitos dos
titulares'', selecionar o produto correspondente e selecionar
``retificação de dados'', é possível que possa necessitar de fazer mais
de um pedido. O processo poderá levar até 30 dias. \\
\\
\textbf{Documento necessário}:
\begin{itemize}
	\item Cópia traçada do novo Cartão de Cidadão.
\end{itemize}
\leavevmode\\
\textbf{Objetivo}: Atualização dos seus dados pessoais junto da Logo Seguros. \\
\textbf{Método}: Envio de pedido de retificação dos dados com CC anexo. \\
\textbf{Tempo expectável de execução}: Até 30 (trinta) dias.

\subsubsection{AdvanceCare/Generali}

Deve efetuar pedido junto do seu mediador ou em alternativa efetuar o
seu pedido por escrito e enviar através de comunicação eletrónica para o
endereço de correio eletrónico
\href{mailto:clientes@tranquilidade.pt}{\nolinkurl{clientes@tranquilidade.pt}}. \\
\\
\textbf{Documentos necessários}:
\begin{itemize}
	\item Modelo de carta, Pedido de alteração/retificação de dados pessoais;
	\item Cópia traçada do novo Cartão de Cidadão.
\end{itemize}
\leavevmode\\
\textbf{Objetivo}: Atualização dos seus dados pessoais junto da Fidelidade. \\
\textbf{Método}: Envio de pedido de retificação dos dados com CC anexo. \\
\textbf{Tempo expectável de execução}: Até 15 (quinze) dias. \\
\\
\textbf{Ligação de referência}:
\begin{itemize}
	\item Generali/Tranquilidade, ajuda, \url{https://www.generalitranquilidade.pt/particulares/servicos-online/area-de-cliente}
\end{itemize}
\leavevmode\\
\newpage

\subsubsection{Sugestões de contribuição}

Se souber ou quiser contribuir com outros processos listam-se algumas
sugestões de contribuição:
\begin{itemize}
	\item Médis;
	\item MGEN;
	\item Medicare;
	\item KeepWells;
	\item Ageas Seguros;
	\item Allianz;
	\item Saúde Prime;
	\item Real Vida;
	\item N Seguros;
	\item Una Seguros;
	\item Zurich;
	\item \ldots{} .
\end{itemize}

\subsection{Telecomunicações}

\subsubsection{MEO/MEO Empresas}

Deve alterar os dados de registo no portal em combinação com alteração
dados de titularidade dos contratos também no portal, encontram-se nos
detalhes do serviço, carregando no botão de pedir alteração
titularidade, pedido de alteração de dados ao abrigo do RGPD e anexar o
novo Cartão de Cidadão ou PDF certificado do \emph{gov.pt} no formulário
e submeter. Em alternativa pode ser efetuado um pedido junto do gestor
de conta se existir. Em alternativa secundária poderá ser feito pedido
de alteração em loja física com o novo Cartão de Cidadão. Deve ser
confirmado o nome que consta na autorização de Débito Direto se
aplicável através do portal e retificar o nome editando o campo
disponível. \\
\\
\textbf{Documentos necessários}:
\begin{itemize}
	\item Modelo de carta, Pedido de alteração/retificação de dados pessoais;
	\item Cópia traçada do novo Cartão de Cidadão.
\end{itemize}
\leavevmode\\
\textbf{Objetivo}: Atualização dos seus dados pessoais junto da Fidelidade. \\
\textbf{Método}: Envio de pedido de retificação dos dados com CC anexo. \\
\textbf{Tempo expectável de execução}: Até 15 (quinze) dias. \\
\\
\textbf{Ligação de referência}:
\begin{itemize}
	\item MEO, alteração dos dados, \url{https://www.meo.pt/ajuda-e-suporte/produtos-meo/gerir-produtos/alteracoes-de-contrato}
\end{itemize}

\subsubsection{Vodafone}

Deve ir a definições do perfil no portal My Vodafone, editar perfil e
alterar o nome, imediato apos pedido, pedidos de alteração relativo aos
contratos é variável, em alternativa deve deslocar-se a uma loja ou
enviar um carta registada para a sede. \\
\\
\textbf{Documentos necessários}:
\begin{itemize}
	\item Modelo de carta, Pedido de alteração/retificação de dados pessoais;
	\item Cópia traçada do novo Cartão de Cidadão.
\end{itemize}
\leavevmode\\
\textbf{Objetivo}: Atualização dos seus dados pessoais junto da Fidelidade. \\
\textbf{Método}: Envio de pedido de retificação dos dados com CC anexo. \\
\textbf{Tempo expectável de execução}: Até 15 (quinze) dias. \\
\\
\textbf{Ligação de referência}:
\begin{itemize}
	\item My Vodafone, ajuda, \url{https://ajuda.vodafone.pt/a-minha-conta/dados-pessoais/como-altero-os-dados-pessoais-no-my-vodafone}
\end{itemize}

\subsubsection{Outros operadores}

Pedido numa loja física ou remeter um email com o pedido por escrito
devidamente assinado e com anexos necessários para verificação dos
dados. \\
\\
\textbf{Documentos necessários}:
\begin{itemize}
	\item Modelo de carta, Pedido de alteração/retificação de dados pessoais;
	\item Cópia traçada do novo Cartão de Cidadão.
\end{itemize}
\leavevmode\\
\textbf{Objetivo}: Atualização dos seus dados pessoais junto da Fidelidade. \\
\textbf{Método}: Envio de pedido de retificação dos dados com CC anexo. \\
\textbf{Tempo expectável de execução}: Até 15 (quinze) dias.

\subsubsection{Sugestões de contribuição}

Se souber ou quiser contribuir com outros processos listam-se algumas
sugestões de contribuição:
\begin{itemize}
	\item NOS;
	\item Digi/Nowo;
	\item Lycamobile;
	\item \ldots{} .
\end{itemize}
	
	%-------------------
	% OUTRAS ALTERAÇÕES
	%-------------------
	%-------------------
% OUTRAS ALTERAÇÕES
%-------------------

\newpage

\section{Outras alterações}

\subsection{Pessoas coletivas de utilidade pública e outros}

\subsubsection{Associações, Fundações e Cooperativas}

\paragraph{MUSSOC}

Deve ser enviado uma mensagem de correio eletrónico para o
mussoc@mussoc.com com o pedido de atualização dos seus dados incluindo
uma cópia traçada da sua identificação ou em alternativa efetuar o
pedido presencialmente.

\subsubsection{Sugestões de contribuição}

Se souber ou quiser contribuir com outros processos listam-se algumas
sugestões de contribuição:
\begin{itemize}
	\item Outras Associações, Fundações ou Cooperativas;
	\item Bibliotecas;
	\item Igrejas ou outras instituições religiosas ou espirituais;
	\item \ldots{} .
\end{itemize}

\subsection{\texorpdfstring{Plataformas da \emph{sharing-economy}}{Plataformas da sharing-economy}}

\subsubsection{Uber / Uber Eats}

Deve editar o nome no perfil de utilizador, ficará alterado de imediato
após a confirmar na aplicação. Ao alterar nome no perfil de utilizador
da aplicação móvel Uber esta também altera na aplicação móvel Eats,
sendo esse processo imediato. \\
\\
\textbf{Ligação de referência}:
\begin{itemize}
	\item Uber, ajuda, quero atualizar o meu perfil, \url{https://help.uber.com/pt-PT/riders/article/atualizar-o-meu-perfil?nodeId=8a661c77-fd7a-4016-857a-531a50084d42}
\end{itemize}

\subsubsection{Bolt}

Deve alterar o nome de utilizador nas definições de perfil da aplicação
móvel, ficará alterado imediatamente após confirmar.

\textbf{Ligação de referência}:
\begin{itemize}
	\item Bolt, ajuda, quero alterar os meus dados, \url{https://bolt.eu/pt-pt/support/articles/115002906553/}
\end{itemize}

\subsubsection{Glovo}

Deve alterar o nome nas definições do perfil tocando no nome. A
alteração tem efeito imediato.

\subsection{Redes sociais}

\subsubsection{Instagram/Facebook (Meta)}

Deve alterar o nome do perfil com efeito imediato através das
definições/centro de contas da Meta. Caso não seja possível, terá de ser
enviado um documento certificado em formato \emph{JPEG} (ou foto) nos
pedidos alternativos/contestação do nome. A alteração poderá ser
recusada na primeira tentativa. Em caso de recusa, o pedido poderá ser
aceite dentro de um dia após a notificação de falha. \\
\\
\textbf{Ligação de referência}:
\begin{itemize}
	\item Facebook, ajuda, mudar o nome, \url{https://www.facebook.com/help/contact/1417759018475333}
\end{itemize}

\subsubsection{LinkedIn}

Deve alterar o nome nas definições de utilizador através do seu
navegador. A alteração tem efeito imediato, se tiver verificação de
identidade ativa terá de a refazer. \\
\\
\textbf{Ligação de referência}:
\begin{itemize}
	\item LinkedIn, mudar nome, \url{https://www.linkedin.com/help/linkedin/answer/a1342836/modificar-a-exibicao-do-seu-nome-em-seu-perfil?lang=pt}
\end{itemize}

\subsubsection{Twitter/X}

Deve alterar nome nas definições de perfil de utilizador com efeito
imediato. \\
\\
\textbf{Ligação de referência}:
\begin{itemize}
	\item X, mudar nome, \url{https://help.x.com/pt/managing-your-account/change-x-handle}
\end{itemize}

\subsubsection{Reddit}

Deve alterar nome nas definições de utilizador com efeito imediato. No
entanto o \emph{username} não é alterável e será necessário apagar e
criar outra conta. \\
\\
\textbf{Ligação de referência}:
\begin{itemize}
	\item Reddit, mudar nome, \url{https://www.reddit.com/r/NovoNoReddit/comments/17s1vdh/como\_faz\_pra\_mudar\_o\_nome\_de\_usu\%C3\%A1rio/}
\end{itemize}

\subsection{Aplicações de comunicações}

\subsubsection{WhatsApp}

Deve alterar o nome nas definições da aplicação móvel. \\
\\
\textbf{Ligação de referência}:
\begin{itemize}
	\item WhatsApp, ajuda, atualizar informações, \url{https://faq.whatsapp.com/859240711908360/?locale=pt}
\end{itemize}

\subsubsection{Telegram}

Deve alterar o nome através das definições da aplicação móvel. \\
\\
\textbf{Ligação de referência}:
\begin{itemize}
	\item Telegram, ajuda, \url{https://telegram.org/faq/br}
\end{itemize}

\subsection{Portais de entretenimento}

\subsubsection{Disney Plus}

Deve alterar o nome nas definições de conta, ligado à faturação.

\subsubsection{Netflix}

Deve alterar o nome nas definições de conta, ligado à faturação. \\
\\
\textbf{Ligação de referência}:
\begin{itemize}
	\item Netflix, ajuda, atualizar as informações, \url{https://help.netflix.com/pt-pt/node/244}
\end{itemize}

\subsection{\texorpdfstring{Outros portais/aplicações na \emph{internet}}{Outros portais/aplicações na internet}}

\subsubsection{Google}

Deve alterar o nome nas definições de utilizador. \\
\\
\textbf{Ligação de referência}:
\begin{itemize}
	\item Google, suporte, alterar informações de conta, \url{https://support.google.com/accounts/answer/27442}
\end{itemize}

\subsubsection{Open AI (Chat GPT)}

Deve alterar o nome nas definições de utilizador. \\
\\
\textbf{Ligação de referência}:
\begin{itemize}
	\item OpenAI, suporte, alterar nome, \url{https://help.openai.com/en/articles/6640864-how-do-i-change-my-name-for-my-openai-account}
\end{itemize}

\subsubsection{OLX}

Deve alterar o nome nas definições de conta. \\
\\
\textbf{Ligação de referência}:
\begin{itemize}
	\item OLX, ajuda, alterar nome, \url{https://help.olx.pt/olxpthelp/s/article/dados-pessoais-V45}
\end{itemize}

\subsubsection{Sugestões de contribuição}

Se souber ou quiser contribuir com outros processos listam-se algumas
sugestões de contribuição:
\begin{itemize}
	\item Plataformas de vendas;
	\item Aplicações de internet;
	\item Vinted;
	\item CustoJusto;
	\item DeepSeek;
	\item Poe;
	\item Amazon Web Services;
	\item Plataformas de registo de domínios;
	\item \ldots{} .
\end{itemize}

\subsection{Estabelecimentos de venda a retalho diversos}

\subsubsection{Cartão Fnac}

Deve deslocar-se a uma loja e pedir a alteração da ficha cliente na
caixa ou secretária cartão Fnac com a aplicação móvel \emph{gov.pt} ou o
novo Cartão de Cidadão.

\subsubsection{Sacoor Brothers}

Deve enviar o pedido para o customercare@sacoor.com, anexando uma cópia
do seu documento devidamente traçado ou presencialmente em loja.

\subsubsection{Sugestões de contribuição}

Se souber ou quiser contribuir com outros processos listam-se algumas
sugestões de contribuição:
\begin{itemize}
	\item Centros comerciais Alegro;
	\item Galeria comercial UBBO, Cartão UBBO;
	\item Burger King;
	\item McDonalds;
	\item Celeiro;
	\item Continente;
	\item Pingo Doce;
	\item Auchan;
	\item Primark;
	\item Cortefiel;
	\item Gato Preto;
	\item IKEA;
	\item \ldots{} .
\end{itemize}

\subsection{Outras entidades}

\subsubsection{Entidades sujeitas ao direito comunitário}

Dados pessoais numa instituição, empresa ou organização sediada dentro
da União Europeia (UE) ou sujeita ao direito comunitário é obrigada à
atualização dos seus dados no âmbito do RGPD sendo que um pedido ao
abrigo deste é à partida suficiente.

\subsubsection{Entidades extra-comunitárias}

A atualização dos seus dados está condicionada à apreciação e aceitação
da contra parte, será difícil forçar a alteração ou retificação dos
dados se existir resistência ou o pedido for negado.

\subsubsection{Sugestões de contribuição}

Se souber ou quiser contribuir com outros processos listam-se algumas
sugestões de contribuição:
\begin{itemize}
	\item EUIPO;
	\item Fichas de cliente diversas;
	\item \ldots{} .
\end{itemize}

\subsection{Outra documentação de utilidade legal diversa}

\subsubsection{Contratos e outros instrumentos legais de especialinteresse}

Poderá consoante o caso também ter de retificar uma quantidade outros
documentos legais pela sua relevância ou potencial interesse entre os
quais listam-se adiante:
\begin{itemize}
	\item Contratos de arrendamento;
	\item Cedências;
	\item Comodatos;
	\item Usufrutos temporários ou vitalícios;
	\item Direito de superfície ou uso diverso;
	\item Contratos de crédito;
	\item Financiamentos;
	\item Amortizações;
	\item Registo de propriedade;
	\item Opções de compra;
	\item Renúncias;
	\item Gerência de sociedades comerciais;
	\item Quotas societárias em sociedades comerciais;
	\item Registos de Propriedade Intelectual ou Industrial;
	\item Declarações ou Certidões emitidas pela Segurança Social (SS);
	\item Declarações ou Certidões emitidas pelo Instituto dos Registos e Notariado (IRN);
	\item Autorização de residência no estrangeiro;
	\item Registo Central do Beneficiário Efetivo (RCBE);
	\item entre outros.
\end{itemize}
\leavevmode\\
Em situações de grande especificidade é preferível arranjar um suporte
especializado adequado às suas circunstâncias.

\subsubsection{Registo testamentário}

O registo testamentário só pode ser alterado através da realização de um
novo testamento. Para isso deve agendar a sua realização no Cartório da
sua preferência levando o seu testamento já feito e 2 (duas) testemunhas
de facto onde poderá optar por fazer um testamento público ou cerrado. \\
\\
\textbf{Documento necessário}:
\begin{itemize}
	\item Novo Cartão de Cidadão
\end{itemize}
\leavevmode\\
\textbf{Outro requisito}:
\begin{itemize}
	\item 2 (Duas) testemunhas de aceitação condicionada, consoante os requisitos legais.
\end{itemize}
\leavevmode\\
\textbf{Objetivo}: Atualizar o registo testamentário. \\
\textbf{Método}: Deslocação a um Cartório Notarial. \\
\textbf{Custo deste processo}: A partir de 159,00€ (cento e cinquenta e nove euros). \\
\textbf{Tempo expectável de execução}: Imediato. \\
\\
\textbf{Ligações de referência}:
\begin{itemize}
	\item gov.pt, fazer um testamento, \url{https://www2.gov.pt/servicos/realizar-testamento}
	\item justiça.gov.pt, saber se existe testamento, \url{https://justica.gov.pt/Servicos/Saber-se-existe-testamento}
\end{itemize}

\subsubsection{Sugestões de contribuição}

Se souber ou quiser contribuir com outros processos listam-se algumas
sugestões de contribuição:
\begin{itemize}
	\item Testamento Vital (DAV) (RENTEV);
	\item Contratos de arrendamento;
	\item Comodatos;
	\item Financiamentos;
	\item \ldots{} .
\end{itemize}
	
	%-------
	% FECHO
	%-------
	%-------
% FECHO
%-------

\section{Fecho}

\subsection{Notas finais}

Depois de finalizar aquilo que considero ser uma espécie de pesadelo
burocrático espero que a utilidade do roteiro perdure no tempo como um
pequeno farol no mar da mudança. Fecha-se assim um capítulo de esforço e
dedicação. Um desejo de esperança e alguma sorte aos utilizadores do
roteiro bem como os votos de sucesso e felicidades.

\subsection{Lista de abreviaturas}

Para maior facilidade na leitura e interpretação do roteiro pode
encontrar aqui todas as abreviaturas em menção ao longo do texto. \\
\\
\begin{itemize}
	\item AML, Área Metropolitana de Lisboa, inclui a Grande Lisboa e Margem Sul.
	\item ANSR, Autoridade Nacional de Segurança Rodoviária, serviço central da administração direta do Estado, foca-se nas matérias de segurança rodoviárias e aplicação do direito contraordenacional rodoviário.
	\item AT, Autoridade Tributária, organismo do Ministério das Finanças responsável por serviços relacionados com impostos e alfândega.
	\item BEP, Bolsa de Emprego Público, é uma base de informação que promove, simplifica e agiliza os processos de recrutamento de recursos humanos da Administração Pública.
	\item BUD, Balcão Único da Defesa, balcão central de acesso público aos serviços relacionados com a defesa nacional.
	\item CC, Cartão de Cidadão, principal documento de identificação.
	\item CGD, Caixa Geral de Depósitos, o maior banco em Portugal detido pelo Estado Português.
	\item CMD, Chave Móvel Digital, meio de autenticação e assinatura digital certificado pelo Estado Português associado a um número de telemóvel.
	\item CNPD, Comissão Nacional de Proteção de Dados, é autoridade nacional de controlo de dados pessoais dotada de independência administrativa.
	\item CPC, Código do Processo Civil, lei de regulamentação do trânsito do processo judicial civil.
	\item CRC, Código do Registo Civil, lei de regulamentação do registo civil.
	\item CRP, Constituição da República Portuguesa
	\item CV, Curriculum Vitae, currículo ou documento histórico que traça as experiências educacionais e profissionais.
	\item DAV, Diretiva Antecipada de Vontade, conhecido também por testamento vital, é o documento onde uma pessoa manifesta a sua vontade sobres os cuidados de saúde que deseja receber.
	\item DDN, Dia da Defesa Nacional, dia de sensibilização para a Defesa Nacional e divulgação das Forças Armadas, substitui o serviço militar obrigatório anterior.
	\item DUA, Documento Único Automóvel ou Certificado de Matrícula, é o documento de circulação dos veículos matriculados em Portugal.
	\item \emph{EUIPO}, \emph{European Union Intellectual Property Office}, é o gabinete de propriedade intelectual e industrial responsável pela gestão dos registos a nível europeu.
	\item \emph{GPS}, \emph{Global Positioning System}, é um sistema de navegação por satélite que fornece ao utilizador a sua localização e horário independentemente das condições atmosféricas.
	\item ID, refere-se simplesmente à palavra identificação.
	\item IEFP, Instituto do Emprego e Formação Profissional, é um instituto público com missão de promover o emprego de qualidade, a reintegração e também a formação profissional.
	\item IMT, Instituto da Mobilidade e dos Transportes, é um instituto público parte da administração indireta do Estado Português.
	\item INPI, Instituto Nacional da Propriedade Industrial, é um instituto público com a finalidade de proteger e promover a propriedade industrial.
	\item IPDJ, Instituto Português do Desporto e Juventude, é um organismo do Estado Português promovendo a inovação, o empreendedorismo, o desporto e temáticas ligadas à juventude.
	\item IRN, Instituto dos Registos e Notariado, é um organismo público cuja com a finalidade de executar e acompanhar as políticas e serviços de registo.
	\item \emph{JPEG}, \emph{Joint Photographic Experts Group}, é um formato de imagem digital que utiliza compressão com perdas de modo a reduzir o tamanho dos arquivos.
	\item LSM, Lei do Serviço Militar, lei que estabelece os termos do serviço militar.
	\item MAI, Ministério da Administração Interna, é um organismo responsável pelas políticas de segurança pública, de proteção e socorro, de imigração e asilo, bem como da prevenção e segurança rodoviária e administração eleitoral.
	\item MDN, Ministério da Defesa Nacional, organismo do Estado Português encarregue da preparação e execução da política de Defesa Nacional, fiscalização e garantia da administração das Forças Armadas.
	\item NIM, Número de Identificação Militar, é o número atribuído na presença e cumprimento do Dia da Defesa Nacional.
	\item \emph{PDF}, \emph{Portable Document Format}, é um formato de arquivo digital criado pela \emph{Adobe} para representar documentos fiavelmente independentemente da plataforma. É um formato tipicamente aceite em todos os estabelecimentos pela versatilidade e confiabilidade.
	\item PEP, Passaporte Eletrónico Português, documento indispensável para propósitos de viagem e identificação.
	\item PIN, \emph{Personal Identification Number}, geralmente refere-se a um código pessoal de acesso, confidencial e não deve ser partilhados com terceiros.
	\item PPI, \emph{Pixels per inch}, refere-se à resolução de uma imagem ou documento, quanto maior o seu valor maior será o detalhe e o tamanho do ficheiro.
	\item PT, Portugal, Português ou referência ao país, nacionalidade ou origem Portuguesa.
	\item RCBE, Registo Central do Beneficiário Efetivo, é o registo centralizado das pessoas que controlam uma determinada entidade jurídica.
	\item RGPD, Regulamento Geral sobre a Proteção de Dados, é o regulamento europeu que visa garantir a proteção dos dados e o direitos dos seus titulares.
	\item SGMAI, Secretaria-Geral do Ministério da Administração Interna, tem como objetivo o apoio técnico e administrativo aos gabinetes do Governo integrados no Ministério da Administração Interna.
	\item SS, Segurança Social ou Instituto da Segurança Social é um organismo dotado de autonomia administrativa, financeira e patrimonial encarregue de assegurar os direitos básicos de bem-estar, coesão social e oportunidades iguais dos cidadãos.
	\item TML, Transportes Metropolitanos de Lisboa, é a autoridade dos transportes públicos explorados na área metropolitana de Lisboa.
	\item UE, referente à União Europeia.
	\item \emph{gov.pt}, tipicamente refere-se à aplicação móvel de identificação e autenticação do Estado Português.
	\item \emph{q.b.}, equivalente a ``quanto baste''.
\end{itemize}
	
	% ÚLTIMA PÁGINA VAZIA
	\newpage
	\clearpage
	
%-------------------
% FIM DOS CONTEÚDOS
%-------------------
\end{document}
