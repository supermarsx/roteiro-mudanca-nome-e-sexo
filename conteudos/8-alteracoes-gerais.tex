%-------------------
% ALTERAÇÕES GERAIS
%-------------------

\newpage

\section{Alterações gerais}

\subsection{Nota introdutória}

Nesta secção são mencionadas todas as restantes alterações a efetuar
após a finalização da secção anterior das ``Alterações iniciais'' e
especialmente da secção de ``Identificação essencial''. Recomenda-se a
finalização da secção de ``Identificação essencial'' antes de prosseguir
com as restantes alterações. Esta secção cobre elementos e locais de
importância variada onde terá de ser o utilizador do roteiro a
determinar a sua prioridade e relevância. A partir daqui os passos e
alterações a executar vão depender das suas necessidades e
circunstâncias individuais. \\
\\
\textbf{Custos diretos totais expectáveis nesta fase}: Variável. \\
\textbf{Tempo total expectável de execução da fase}: Cerca de 40 (quarenta) dias. \\
\\
\textbf{Pré-requisitos desta fase}:
\begin{itemize}
	\item Fase de identificação essencial finalizada;
	\item O seu novo \textbf{Cartão de Cidadão};
	\item Cópia traçada do novo Cartão de Cidadão;
\end{itemize}
\leavevmode\\
\textbf{Objetivos desta fase}:
\begin{itemize}
	\item Atualizar os seus dados pessoais junto do seu empregador;
	\item Atualizar os seus dados pessoais em estabelecimentos de ensino;
	\item Obter um novo Cartão de Aluno;
	\item Obter um novo Certificado de Habilitações;
	\item Obter novas Cartas de Recomendação;
	\item Atualizar o seu Curriculum Vitae;
	\item Atualizar os seus dados em plataformas de emprego;
	\item Atualizar os seus dados pessoais relativo a estabelecimentos ligados a serviços de saúde;
	\item Atualizar os seus dados relativos a serviço ligados à mobilidade;
	\item Obter um novo Certificado de Matrícula;
	\item Obter um novo Passe de Transportes Públicos;
	\item Obter uma nova declaração de início de atividade da Autoridade Tributária;
	\item Atualizar os seus dados pessoais junto de entidades bancárias e financeiras;
	\item Obter um novo Cartão bancário.
\end{itemize}
\leavevmode\\
\newpage
\leavevmode\\
\textbf{Fluxo genérico de acontecimentos desta fase}:
\begin{itemize}
	\item Efetuar um pedido de atualização dos seus dados pessoais:
	\begin{itemize}
		\item No local de trabalho;
		\item No estabelecimento de ensino.
	\end{itemize}
	\item Requerer:
	\begin{itemize}
		\item Um novo Cartão do Aluno;
		\item Um novo Certificado de Habilitações.
	\end{itemize}
	\item Pedir novas Cartas de Recomendação;
	\item Emendar o Curriculum Vitae (CV);
	\item Alterar o nome em plataformas de recrutamento;
	\item Atualizar os seus dados pessoais:
	\begin{itemize}
		\item Em estabelecimentos de saúde;
		\item Em laboratórios de análises clínicas;
		\item Em relatórios médicos;
		\item Aplicações de saúde.
	\end{itemize}
	\item Obter um novo Certificado de Matrícula (DUA);
	\item Atualizar os seus dados pessoais:
	\begin{itemize}
		\item No portal IMT Online;
		\item No portal da ANSR.
	\end{itemize}
	\item Obter um novo Passe de Transportes Públicos;
	\item Atualizar os seus dados pessoais:
	\begin{itemize}
		\item Na Via Verde;
		\item \ldots{} .
	\end{itemize}
	\item Obter uma nova declaração de início de atividade da Autoridade Tributária;
	\item Atualizar os seus dados pessoais junto de entidades bancárias ou financeiras:
	\begin{itemize}
		\item Banco CTT;
		\item Caixa Geral de Depósitos (CGD);
		\item \ldots{} .
	\end{itemize}
	\newpage
	\item Atualizar os seus dados pessoais junto de companhias de seguros:
	\begin{itemize}
		\item Fidelidade;
		\item Logo Seguros;
		\item \ldots{} .
	\end{itemize}
	\item Atualizar os seus dados junto de empresas de telecomunicações:
	\begin{itemize}
		\item MEO;
		\item Vodafone;
		\item \ldots{} .
	\end{itemize}
\end{itemize}

\subsection{Local de trabalho}

\textbf{Palavras-chave}: colaborador, emprego, contrato, trabalhador. \\
\\
Deve entregar no seu local de trabalho os documentos de validação do
novo Cartão de Cidadão que tiver ao seu dispor para que os recursos
humanos (RH) e elementos coordenadores essenciais executem a alteração
dos seus dados nos vários locais, arquivos e serviços associados ao seu
posto de trabalho. \\
\\
A seguinte lista é uma demonstração prática dos locais, arquivos ou
sistemas a alterar:
\begin{itemize}
	\item \textbf{Ficha do trabalhador};
	\item \textbf{Endereço de correio eletrónico} (se aplicável), poderá pedir o encaminhamento do seu antigo endereço por forma a não perder mensagens potencialmente importantes;
	\item \textbf{Seguro de acidentes de trabalho};
	\item \textbf{Seguro de saúde} (se aplicável), poderá ser necessário emitir um novo cartão físico;
	\item \textbf{Sistema de processamento de vencimentos}, tipicamente alterado pela pessoa responsável pelos RH ou administrador de sistema;
	\item \textbf{Cartões ou outros documentos físicos} (conforme aplicável);
	\item \textbf{Sistemas de registo horário} (conforme aplicável), tais como por exemplo relógios de ponto;
	\item \textbf{Sistemas auxiliares de marcação de férias ou aplicações assistentes às funções de RH} (conforme aplicável);
	\item \textbf{Diretórios da empresa ou outros sistemas de autenticação integrada} (se aplicável);
	\item \textbf{Cartões de oferta e/ou refeição} (se aplicável), deve ser pedida a alteração de nome ao abrigo do RGPD ou pedir em alternativa novo cartão consoante o caso (por exemplo: Cartão Dá (Continente) terá de ser pedido novo cartão).
	\item \textbf{Outros sistemas internos que contenham dados de identificação}, outros arquivos ou aplicações onde possam existir dados pessoais sujeitos a retificação.
\end{itemize}
\leavevmode\\
Todas as alterações devem ficar expectavelmente finalizadas dentro de 15
(quinze) dias, no entanto, poderá haver sistemas ou situações onde a
intervenção de terceiros ou múltiplos elementos num único processo
poderá levar a atrasos mas nunca superiores a 30 (trinta) dias. \\
\\
Convém periodicamente ir verificando se todas as alterações já foram
executadas por forma a poder reforçar o pedido se necessário ou efetuar
uma intervenção alternativa para garantir a execução adequada do seu
pedido. \\
\\
\textbf{Documento necessário}:
\begin{itemize}
	\item Cópia traçada do novo Cartão de Cidadão ou Cópia do pedido de novo Cartão de Cidadão.
\end{itemize}
\leavevmode\\
\textbf{Objetivo}: Obter uma nova declaração de início de atividade atualizada. \\
\textbf{Método}: Pedido de alteração da atividade no portal das Finanças. \\
\textbf{Custo deste processo}: Gratuito. \\
\textbf{Tempo expectável de execução}: Até 30 (trinta) dias. \\

\subsection{Educação/Emprego}

\subsubsection{Universidades, escolas e outros estabelecimentos de ensino}

\textbf{Palavras-chave}: educação, ensino, aulas, educacional. \\
\\
Estando ainda a frequentar um estabelecimento de ensino ou formação
poderá deslocar-se à secretaria desse estabelecimento e efetuar o seu
pedido presencialmente, bastando assim levar o seu novo documento de
identificação e referir a necessidade de atualização dos seus dados no
seguimento de alteração do seu documento de identificação.\\
\\
\textbf{Documento necessário}:
\begin{itemize}
	\item Cópia traçada do novo Cartão de Cidadão.
\end{itemize}
\leavevmode\\
\textbf{Objetivo}: Atualização do dados do aluno/formando. \\
\textbf{Custo deste processo}: Gratuito. \\
\textbf{Tempo expectável de execução}: Imediato. \\

\paragraph{Cartão do Aluno}
\leavevmode\\[4pt]
Poderá ser necessária a emissão de um novo Cartão do Aluno ou a
retificação em mais que um sistema em algumas circunstâncias específicas
ou estabelecimentos, podendo assim representar um custo diferenciado e
tempo de execução superior ao expectável. \\
\\
\textbf{Documento necessário}:
\begin{itemize}
	\item Cópia traçada do novo Cartão de Cidadão.
\end{itemize}
\leavevmode\\
\textbf{Objetivo}: Obter um novo Cartão do Aluno ou atualizar os seus dados. \\
\textbf{Custo deste processo}: Variável, a rondar os 7,00€ (sete euros). \\
\textbf{Tempo expectável de execução}: Até 15 (quinze) dias. \\
\\
\subsubsection{Certificado de Habilitações}

\textbf{Palavras-chave}: diploma, qualificações, aprendizagem, escolaridade, certificação. \\
\\
Para obter um novo certificado de habilitações, terá de contactar o
estabelecimento de ensino ou formação utilizando o modelo de carta,
disponibilizado nos anexos, onde poderá adaptar, assinar digitalmente e
enviar através de mensagem de correio eletrónico, por correio registado
ou entregar em mãos conforme o que lhe for mais conveniente. Para que a
sua identidade seja verificada corretamente deverá anexar uma cópia do
seu novo documento para que sejam confirmadas as alterações mencionadas
na carta. Poderá ser útil ligar primeiro para o estabelecimento por
forma a conhecer qual o endereço de correio eletrónico mais adequado
para remeter o seu pedido. \\
\\
\textbf{Documentos necessários}:
\begin{itemize}
	\item Modelo de carta, pedido de 2ª via do Certificado de Habilitações;
	\item Cópia traçada do novo Cartão de Cidadão.
\end{itemize}
\leavevmode\\
\textbf{Objetivo}: Obter uma 2ª (segunda) via do Certificado de Habilitações. \\
\textbf{Método}: Entrega de carta física ou digital a partir do modelo. \\
\textbf{Custo deste processo}: Gratuito ou pago dependendo do estabelecimento. \\
\textbf{Entrega do novo certificado}: Virtual ou físico dependendo do estabelecimento. \\
\textbf{Tempo expectável de execução}: Até 15 (quinze) dias. \\
\\
\textbf{Ligações de referência}:
\begin{itemize}
	\item Secretaria-Geral da Educação e Ciência, certificados de habilitações, perguntas frequentes, \url{https://www.sec-geral.mec.pt/faqs/587}
	\item Doutor Finanças, como obter um certificado de habilitações, \url{https://www.doutorfinancas.pt/carreira-e-rendimentos/emprego/como-obter-um-certificado-de-habilitacoes/}
\end{itemize}

\subsubsection{Cartas de recomendação}

Consoante as suas circunstâncias específicas e pessoais poderá ser útil
pedir a retificação ou elaboração de novas cartas de recomendação com o
seu novo nome de modo a poder anexá-las ao seu Currículo Vitae e
garantir a consistência e coesão das suas referências e experiências
profissionais. \\
\\
\textbf{Objetivo}: Obter cartas de recomendação atualizadas. \\
\textbf{Custo deste processo}: Gratuito. \\
\\
\textbf{Ligação de referência}:
\begin{itemize}
	\item NValores, exemplos de cartas de recomendação, \url{https://www.nvalores.pt/exemplos-de-cartas-de-recomendacao/}
\end{itemize}

\subsubsection{Curriculum Vitae (CV)}

\textbf{Palavras-chave}: currículo, experiências, cv, trajeto
profissional, escolaridade. \\
\\
Deve atualizar o seu currículo com o novo nome e possivelmente todas as
suas experiências (se não o tiver feito) independentemente do seu estado
profissional atual, isto poderá implicar a mudança de pronomes ou nomes
com género onde a alteração poderá ser relevante. Esta alteração é
especialmente importante dando-lhe um ponto de partida para procura de
emprego se vier a necessitar de o fazer. A atualização do currículo é
imperativo como uma ferramenta de procura de trabalho e auxiliar na
eventualidade de imprevistos. Se não o souber fazer deverá procurar
ajuda de um profissional para auxiliar na composição e atualização do
documento. Há empregadores que preferem o modelo Europass e outros não,
deve utilizar o seu julgamento para adequar-se às necessidades e
potenciais empregadores. \\
\\
\textbf{Objetivo}: Obter um Curriculum Vitae com os dados atualizados. \\
\textbf{Método}: Alteração manual dos seus dados. \\
\textbf{Custo deste processo}: Gratuito. \\
\textbf{Tempo expectável de execução}: Imediato. \\
\\
\textbf{Ligações de referência}:
\begin{itemize}
	\item Europa, sobre criar um CV Europass, \url{https://europass.europa.eu/pt/create-europass-cv}
	\item Europa, criar um CV Europass, \url{https://europa.eu/europass/eportfolio/screen/cv-editor?lang=pt}
	\item Canva, criar um CV, \url{https://www.canva.com/pt\_pt/criar/cv/}
\end{itemize}

\subsubsection{Plataformas de emprego/recrutamento}

\textbf{Palavras-chave}: candidaturas, emprego, trabalho, recrutadores. \\
\\
Deve atualizar o seu nome e documentos carregados, nomeadamente
currículos, em plataformas de emprego para que potenciais recrutadores
tenham a sua informação devidamente atualizada. O processo de alteração
é muito variável consoante as plataformas mas é possível que possa
alterar o seu nome através das definições de perfil de utilizador da
plataforma em questão. Listam-se alguns exemplos de plataformas de
emprego: Sapo Emprego, Net-empregos, Michael Page, Landing.jobs, IT
Jobs, Indeed\ldots{} \\
\\
\textbf{Objetivo}: Atualizar o nome e género em plataformas de emprego. \\
\textbf{Método}: Alteração manual dos seus dados. \\
\textbf{Custo deste processo}: Gratuito. \\
\textbf{Tempo expectável de execução}: Imediato. \\

\newpage

\subsubsection{Sugestões de contribuição}

Se souber ou quiser contribuir com outros processos listam-se algumas
sugestões ou exemplos de contribuições:
\begin{itemize}
	\item Universidade de Lisboa;
	\item Universidade Aberta;
	\item Universidade Lusófona;
	\item Instituto do Emprego e Formação Profissional (IEFP);
	\item Bolsa de Emprego Público (BEP);
	\item Plataforma de recrutamento do Instituto dos Registos e Notariado (IRN);
	\item \ldots{} .
\end{itemize}

\subsection{Saúde}

\subsubsection{Hospitais e clínicas públicas e/ou privadas}

\textbf{Palavras-chave}: hospital, clínica, cuidados de saúde. \\
\\
Deve verificar no portal internet do estabelecimento (se este existir) a
existência da possibilidade de retificação dos seus dados pessoais. Em
alternativa deve através de comunicação eletrónica, enviar uma mensagem
de correio eletrónico, a efetuar o pedido de retificação dos seus dados
ao abrigo do RGPD. Na eventualidade de não conseguir através de uma das
opções anteriores, deverá pedir a retificação/alteração dos seus dados
no próprio estabelecimento. É importante notar que clínicas podem ser
dentárias, estéticas, integradas ou especializadas diversas. Exemplos:
Hospitais da CUF, Luz ou Lusíadas. \\
\\
\textbf{Documentos necessários}:
\begin{itemize}
	\item Modelo de carta, pedido de alteração/retificação de dados pessoais;
	\item Novo Cartão de Cidadão.
\end{itemize}
\leavevmode\\
\textbf{Objetivo}: Atualizar os seus dados num estabelecimento de saúde. \\
\textbf{Método}: Pedido presencial no balcão do estabelecimento. \\
\textbf{Custo deste processo}: Gratuito. \\
\textbf{Tempo expectável de	execução}: Imediato. \\

\newpage

\subsubsection{Laboratórios de análises clínicas}

\textbf{Palavras-chave}: laboratório, análises clínicas, níveis séricos, centros de saúde. \\
\\
Se estiver a efetuar uma transição médica é provável que recorra a um ou
mais laboratórios de análises clínicas. É importante que seja feita a
alteração do marcador nem que seja por uma questão de consistência da
informação e níveis séricos de referência que são baseados no sexo. O
pedido pode ser feito através de comunicação eletrónica ao laboratório
que utiliza. Exemplo prático, se efetuar as suas análises através da CUF
cujo o fornecedor de serviços de laboratório é a Germano de Sousa os
seus dados irão ser atualizados eventualmente por encaminhamento da CUF,
ou seja, ainda que na CUF possa já ter os dados atualizados, os dados do
laboratório só são atualizados numa data posterior. Poderá no entanto
pedir sempre a sua verificação se assim o entender. \\
\\
\textbf{Documentos necessários}:
\begin{itemize}
	\item
	Modelo de carta, pedido de alteração/retificação de dados pessoais;
	\item
	Cópia traçada do Cartão de Cidadão.
\end{itemize}
\leavevmode\\
\textbf{Objetivo}: Atualizar os seus dados num laboratório de análises
clínicas. \\
\textbf{Método}: Pedido através de comunicação eletrónica. \\
\textbf{Custo deste processo}: Gratuito. \\
\textbf{Tempo expectável de	execução}: Até 20 (vinte) dias. \\

\subsubsection{Relatórios médicos, diagnósticos e outros instrumentos de utilidade médica e diagnóstica permanentes}

\textbf{Palavras-chave}: relatório, diagnóstico. \\
\\
Deverá sempre que necessário e possível pedir a retificação do nome em
relatórios, diagnósticos e outros documentos com utilidade e/ou validade
vitalícia ao médico, entidade ou organização responsável pelo relatório.
Poderá invocar os seus direitos no âmbito do RGPD para maior facilidade
na obtenção de resposta ao pedido. \\
\\
\textbf{Documentos necessários}:
\begin{itemize}
	\item
	Modelo de carta, pedido de 2ª (segunda) via de Relatório Médico ou
	Avaliação Psicológica;
	\item
	Cópia traçada do novo Cartão de Cidadão.
\end{itemize}
\leavevmode\\
\textbf{Objetivo}: Obter novo relatório médico. \\ 
\textbf{Método}: Pedido por via eletrónica. \\
\textbf{Custo deste processo}: Gratuito, depende do caso. \\ 
\textbf{Tempo expectável de execução}: Até 30 (trinta) dias.

\newpage

\subsubsection{My Cuf}

\textbf{Palavras-chave}: cuf, hospital, saúde privada. \\
\\
É necessário cancelar o acesso atual e recriar novamente com novo
endereço por forma a poder alterar o endereço de correio eletrónico de
acesso. Para isso é preciso aceder à página do centro de apoio ao
cliente, selecionar o formulário de contacto. selecionar o separador my
cuf, apoio, preencher o formulário e indicar a principal unidade que
visita, na mensagem poderá escrever o seguinte: ``Boa tarde, pede-se o
cancelamento do acesso ao Portal My CUF com o nome utilizador
{[}NOME-DO-UTILIZADOR{]}, por resultado da alteração e descontinuação da
utilização dessa caixa de correio eletrónico. No entanto se for possível
apenas a alteração pede-se que o novo nome de utilizador seja
{[}NOVO-NOME-DE-UTILIZADOR{]}. Esta comunicação é feita de acordo com o
direito previsto no artigo 16.º (décimo sexto) do RGPD. Agradeço desde
já a atenção prestada.''\\
\\
\textbf{Objetivo}: Atualizar o seu nome de utilizador. \\
\textbf{Método}: Pedido por via eletrónica. \\
\textbf{Custo deste processo}: Gratuito. \\
\textbf{Tempo expectável de execução}: Até 15 (quinze) dias. \\
\\
\textbf{Ligações de referência}:
\begin{itemize}
	\item CUF, centro de apoio ao cliente questões relativas ao My CUF, \url{https://www.cuf.pt/centro-de-apoio-ao-cliente?faq\_categoria=66651}
	\item CUF, formulário específico de contacto, \url{https://www.cuf.pt/centro-de-apoio-ao-cliente\#66841673}
	\item CUF, centro de apoio ao cliente, \url{https://www.cuf.pt/centro-de-apoio-ao-cliente}
	\item Google Play Store, aplicação móvel para sistema Android, \url{https://play.google.com/store/apps/details?id=pt.saudecuf.myCUF\&hl=pt\_PT}
	\item Apple App Store, aplicação móvel para sistema iOS, \url{https://apps.apple.com/pt/app/my-cuf/id811304952}
\end{itemize}

\newpage

\subsubsection{Sugestões de contribuição}

Se souber ou quiser contribuir com outros processos listam-se algumas
sugestões ou exemplos de contribuições:
\begin{itemize}
	\item Clínica Santa Madalena;
	\item Lusíadas Saúde;
	\item HeyDoc;
	\item My Luz;
	\item Kardia;
	\item Unilabs;
	\item Joaquim Chaves;
	\item Trofa Saúde;
	\item Germano de Sousa;
	\item {\ldots} .
\end{itemize}

\subsection{Mobilidade}

\subsubsection{Documento Único Automóvel (DUA)}

\textbf{Palavras-chave}: automóvel, conservatória, dua, certificado de matrícula, registo. \\
\\
Deve efetuar o seu pedido presencialmente num Registo Automóvel ou numa
Loja do Cidadão. Efetuando o seu pedido online através do portal
Automóvel Online irá poupar 5,25€ pela razão do pedido ser feito
eletronicamente. Para proceder ao pedido do registo através do portal,
deve efetuar os seguintes passos:
\begin{itemize}
	\item Inserir o seu Cartão de Cidadão no leitor de cartões;
	\item Aceder ao portal internet Automóvel Online;
	\item Selecionar o seu certificado do Cartão de Cidadão;
	\item Introduzir o PIN de acesso do Cartão de Cidadão;
	\item Selecionar a opção ``Outros Pedidos'';
	\item Selecionar a opção de ``Alteração de Nome'';
	\item Preencher e submeter o respetivo pedido adequadamente.
\end{itemize}
\leavevmode\\
\newpage
\leavevmode\\
\textbf{Documento necessário}:
\begin{itemize}
	\item Novo Cartão de Cidadão (e leitor de cartões).
\end{itemize}
\leavevmode\\
\textbf{Objetivo}: Obter um novo Certificado de Matrícula. \\
\textbf{Método}: Pedido presencial ao balcão ou eletronicamente através do portal Automóvel Online. \\
\textbf{Custo deste processo}: 35,00€ (trinta e cinco euros) se presencial ou 29,80€ (vinte e nove euros e oitenta cêntimos) se por via eletrónica. \\
\textbf{Entrega do novo	Certificado de Matrícula}: No domicílio. \\
\textbf{Tempo expectável de	execução}: Até 45 (quarenta e cinco) dias. \\
\\
\textbf{Ligações de referência}:
\begin{itemize}
	\item Automóvel Online, entrada, \url{https://www.automovelonline.mj.pt/AutoOnlineProd/}
	\item Automóvel Online, outros pedidos, \url{https://www.automovelonline.mj.pt/AutoOnlineProd/conteudos/listaPedidos.jsp?num=8\#8}
	\item Automóvel Online, consultar pedidos, \url{https://www.automovelonline.mj.pt/AutoOnlineProd/Contribuintes/ContribuintesController?action=pedidosconsulta}
	\item Google Play Store, aplicação DUApp para sistema Android, \url{https://play.google.com/store/apps/details?id=incm.com.leitor.dua\&hl=pt\_PT}
	\item Apple App Store, aplicação DUApp para sistema iOS, \url{https://apps.apple.com/pt/app/duapp/id1474224031}
\end{itemize}

\subsubsection{Instituto da Mobilidade e dos Transportes (IMT)}

\textbf{Palavras-chave}: imt, mobilidade, etc. \\
\\
Efetuado automaticamente após o pedido de novo cartão de cidadão. O
processo é finalizado instantaneamente sem ser necessária intervenção.
Inclui o portal IMT Online. \\
\\
\textbf{Documento necessário}:
\begin{itemize}
	\item Chave Móvel Digital ativa ou acesso do portal da Autoridade Tributária.
\end{itemize}
\leavevmode\\
\textbf{Objetivo}: Atualização dos seus dados junto do IMT. \\
\textbf{Método}: Automático. \\
\textbf{Custo deste processo}: Gratuito. \\
\textbf{Tempo expectável de execução}: Imediato. \\
\\
\textbf{Ligações de referência}:
\begin{itemize}
	\item IMT Online, acesso ao portal, \url{https://servicos.imt-ip.pt/login.aspx?ReturnUrl=\%2fdefault.aspx}
	\item IMT Online, consultar o registo no portal, \url{https://servicos.imt-ip.pt/RegistonoPortal.aspx}
	\item IMT Online, consultar lista de pedidos, \url{https://servicos.imt-ip.pt/ListadePedidos.aspx}
\end{itemize}

\subsubsection{Autoridade Nacional de Segurança Rodoviária (ANSR)}

\textbf{Palavras-chave}: portal das contraordenações, multas, pontos, carta de condução. \\
\\
Deve alterar os dados de registo acedendo ao portal das contraordenações
com o seu acesso habitual e efetuar o pedido através do menu ``alterar
dados''. As alterações são imediatas após a submissão do pedido no
portal.\\
\\
\textbf{Documento necessário}:
\begin{itemize}
	\item Chave Móvel Digital (CMD) ativa ou acesso do portal da Autoridade Tributária.
\end{itemize}
\leavevmode\\
\textbf{Objetivo}: Atualização dos seus dados junto da ANSR. \\
\textbf{Método}: Pedido de alteração através do portal. \\
\textbf{Custo deste processo}: Gratuito. \\
\textbf{Tempo expectável de execução}: Imediato. \\
\\
\textbf{Ligações de referência}:
\begin{itemize}
	\item ANSR, acesso ao portal das contraordenações, \url{https://portalcontraordenacoes.ansr.pt/\_layouts/pages/login.aspx?ReturnUrl=\%2f\_layouts\%2fAuthenticate.aspx\%3fSource\%3d\%252F\%255Flayouts\%252Fpages\%252Fdefault\%252Easpx\&Source=\%2F\%5Flayouts\%2Fpages\%2Fdefault\%2Easpx}
	\item ANSR, portal das contraordenações, alterar dados, \url{https://portalcontraordenacoes.ansr.pt/\_layouts/Pages/AlterarDados.aspx}
	\item ANSR, portal das contraordenações, consulta de processos, \url{https://portalcontraordenacoes.ansr.pt/\_Layouts/Pages/ListaProcessos.aspx}
	\item ANSR, portal das contraordenações, consulta de pontos, \url{https://portalcontraordenacoes.ansr.pt/PortalCO/\_Layouts/Pages/ConsultaPontos.aspx}
\end{itemize}

\subsubsection{Passe Transportes Públicos da AML (Navegante)}

\textbf{Palavras-chave}: passe, comboios, autocarros, mobilidade. \\
\\
Efetuar um pedido de novo passe numa bilheteira de qualquer um dos
operadores de transportes públicos, com o comprovativo de pedido de novo
cartão de cidadão. Será necessário uma foto tipo passe que deverá
entregar obrigatoriamente junto com o seu pedido. \\
\\
\textbf{Documentos necessários}:
\begin{itemize}
	\item Formulários, Anexo D - Requisição de Passe Navegante (TML);
	\item Uma foto recente tipo passe;
	\item Cópia traçada do novo Cartão de Cidadão ou cópia do pedido de novo Cartão de Cidadão.
\end{itemize}
\leavevmode\\
\textbf{Objetivo}: Obter um novo passe de transportes atualizado. \\
\textbf{Método}: Pedido de novo cartão junto de um operador de transportes. \\
\textbf{Custo deste processo}: A partir de 7,00€ (sete euros). \\
\textbf{Entrega de novo passe}: Fisicamente numa bilheteira do operador de transportes ou entregue no domicílio. \\
\textbf{Tempo expectável de execução}: Até 15 (quinze) dias. \\
\\
\textbf{Ligações de referência}:
\begin{itemize}
	\item Navegante, cartões, \url{https://www.navegante.pt/viajar/cartoes}
	\item Navegante, loja online, \url{https://loja.navegante.pt/}
	\item Navegante, loja online, cartão personalizado, \url{https://loja.navegante.pt/product/cartao-navegante-personalizado}
	\item Google Play Store, aplicação móvel para sistema Android, \url{https://play.google.com/store/apps/details?id=pt.card4b.navegante\&hl=pt\_PT}
	\item Apple App Store, aplicação móvel para sistema iOS, \url{https://apps.apple.com/pt/app/navegante/id6484591306}
\end{itemize}

\subsubsection{Via Verde}

\textbf{Palavras-chave}: mobilidade, portagens, identificador. \\
\\
Deve enviar uma mensagem de correio eletrónico para
\href{mailto:cliente@viaverde.pt}{\nolinkurl{cliente@viaverde.pt}}
através do endereço de correio eletrónico de registo no portal com o
novo documento de identificação, poderá utilizar a exportação dos seus
dados para um ficheiro em formato \emph{PDF} a partir da aplicação móvel
\emph{gov.pt} em combinação com a certidão de morada fiscal da
Autoridade Tributária ou o ficheiro exportado da aplicação \emph{gov.pt}
após receção e ativação do novo cartão de cidadão. Em alternativa, deve
ser enviada carta registada para a sede da empresa a efetuar o pedido
com a documentação adequada em anexo. É pedido tipicamente que seja
enviado uma cópia do Cartão de Cidadão mas esse pedido não tem
sustentação legal adequada. Nem sempre é possível conseguir que a
alteração seja executada no primeiro pedido. \\
\\
\textbf{Documentos necessários}:
\begin{itemize}
	\item Modelo de carta, Pedido de alteração/retificação de dados pessoais;
	\item Cópia traçada do novo Cartão de Cidadão.
\end{itemize}
\leavevmode\\
\textbf{Objetivo}: Atualização dos seus dados de cliente junto da Via Verde. \\
\textbf{Método}: Pedido de retificação dos seus dados ao abrigo do RGPD por via eletrónica. \\
\textbf{Custo deste processo}: Gratuito. \\
\textbf{Tempo expectável de execução}: Até 15 (quinze) dias. \\
\\
\textbf{Ligações de referência}:
\begin{itemize}
	\item Via Verde, entrada, \url{https://www.viaverde.pt/particulares/minha-via-verde/}
	\item Via Verde, contratos, \url{https://www.viaverde.pt/particulares/minha-via-verde/contratos}
	\item Google Play, aplicação para sistema Android, \url{https://play.google.com/store/apps/details?id=pt.viaverde.clientes\&hl=pt\_PT}
	\item Apple App store, aplicação para sistema iOS, \url{https://apps.apple.com/pt/app/via-verde/id674583357}
\end{itemize}

\subsubsection{Miio}

\textbf{Palavras-chave}: carregamento elétrico, cartão. \\
\\
Pode alterar o nome das definições de utilizador através da aplicação
móvel. Poderá ser necessário solicitar um novo cartão. A alteração no
portal/aplicação é imediata. \\
\\
\textbf{Objetivo}: Atualização dos seus dados junto do Miio. \\
\textbf{Método}: Alteração dos dados na aplicação móvel. \\
\textbf{Custo deste processo}: Gratuito, ou 30,00€ (trinta euros) para o cartão. \\
\textbf{Tempo expectável de execução}: Imediato, 2 (duas) semanas para o cartão. \\
\\
\textbf{Ligações de referência}:
\begin{itemize}
	\item Miio, entrada, \url{https://www.miio.com/pt}
	\item Miio, aplicação móvel para o navegador de ambiente de trabalho, \url{https://app.miio.com/}
	\item Miio, aplicação móvel para sistema Android, \url{https://play.google.com/store/apps/details?id=com.muvext.miio}
	\item Miio, aplicação móvel para sistema iOS, \url{https://apps.apple.com/us/app/miio/id1462182013?pt=1462182013\&ct=homepage\&mt=8}
\end{itemize}

\subsubsection{Waze}

\textbf{Palavras-chave}: mapas, gps, app, mobilidade. \\
\\
Para atualizar o seu nome na aplicação móvel de navegação Waze deve
aceder ao menu, selecionar a opção o ``meu waze'' e tocar no nome de
utilizador, aí então poderá alterar o nome. A alteração é imediata. Na
eventualidade de também utilizar o mesmo nome de utilizador nos fóruns é
importante notar que irá perder o acesso às funcionalidades associadas
ao nome de utilizador anterior. \\
\\
\textbf{Ligação de referência}:
\begin{itemize}
	\item Waze, centro de ajuda, definir nome de utilizador, \url{https://support.google.com/waze/answer/6268711?hl=pt}
\end{itemize}

\subsubsection{Tesla}

\textbf{Palavras-chave}: automóveis, elétricos. \\
\\
Deve alterar o nome nas definições de conta do utilizador, ficará
alterado imediatamente após a submissão. Se tiver dados de faturação
associados terá de alterá-los posteriormente nos dados de faturação no
seguimento de uma compra ou pedido de assistência. \\
\\
\textbf{Ligação de referência}:
\begin{itemize}
	\item Tesla, atualizar conta, \url{https://www.tesla.com/pt\_pt/support/how-create-update-delete-tesla-account}
\end{itemize}

\subsubsection{Sugestões de contribuição}

Se souber ou quiser contribuir com outros processos listam-se algumas
sugestões de contribuição: \\
\\
\begin{itemize}
	\item Passe Andante;
	\item Carris;
	\item Metropolitano de Lisboa;
	\item Comboios de Portugal, Cartão CP;
	\item \ldots{} .
\end{itemize}

\subsection{Fiscalidade}

\subsubsection{Declaração de início de atividade da Autoridade Tributária (AT)}

Para retificar o nome deve aceder ao portal da finanças com o seu acesso
e entregar uma declaração de alteração de atividade, com essa entrega o
documento comprovativo de alteração já terá o novo nome associado. Se
não souber como efetuar essa entrega deverá consultar um contabilista ou
recurso adequado para o efeito. \\
\\
\textbf{Objetivo}: Obter uma nova declaração de início de atividade atualizada. \\
\textbf{Método}: Pedido de alteração da atividade no Portal das Finanças. \\ 
\textbf{Custo deste processo}: Gratuito. \\
\textbf{Tempo expectável de execução}: Imediato. \\
\\
\textbf{Ligações de referência}:
\begin{itemize}
	\item Autoridade Tributária, acesso, \url{https://www.acesso.gov.pt/v2/loginForm?partID=PFAP\&path=/geral/dashboard}
	\item Autoridade Tributária, declaração de atividade, \url{https://sitfiscal.portaldasfinancas.gov.pt/atividade/atividade/entregar}
\end{itemize}

\subsubsection{Sugestões de contribuição}

Se souber ou quiser contribuir com outros processos listam-se algumas
sugestões de contribuição:
\begin{itemize}
	\item Outras certidões ou declarações emitidas ou geridas pela Autoridade Tributária (AT);
	\item \ldots{} .
\end{itemize}
\leavevmode\\
\newpage

\subsection{\texorpdfstring{Entidades bancárias, financeiras e \emph{fintech}}{Entidades bancárias, financeiras e fintech}}

\subsubsection{Banco CTT}

Para atualizar os seus dados pessoais junto do Banco CTT deve
deslocar-se a uma agência e retirar uma senha, é importante ter o Cartão
de Cidadão consigo. A alteração é feita no momento mas poderá demorar
até 48 horas para que seja totalmente processada. Aparentemente não é
possível a atualização dos dados por via eletrónica. \\
\\
A atualização do nome no MBWay relativo ao cartão só é atualizado 
com o pedido de novo cartão. \\
\\
\textbf{Documento necessário}:
\begin{itemize}
	\item Novo Cartão de Cidadão.
\end{itemize}

\textbf{Objetivo}: Atualização dos seus dados junto do Banco CTT. \\
\textbf{Método}: Pedido presencial numa agência do Banco CTT. \\
\textbf{Tempo expectável de execução}: Até 48 (quarenta e oito) horas. \\
\textbf{Custo de novo cartão}: Desde 18,50€ (dezoito euros e cinquenta cêntimos). \\

\subsubsection{Caixa Geral de Depósitos (CGD)}

Para atualizar os seus dados junto da Caixa Geral de Depósitos (CGD)
deve aceder à sua conta através do portal de \emph{homebanking} e
fazendo recurso da sua Chave Móvel Digital (CMD) efetuar atualização dos
seus dados. A alteração tem efeitos imediatos. \\
\\
\textbf{Documento necessário}:
\begin{itemize}
	\item Chave Móvel Digital (CMD) ativa.
\end{itemize}
\leavevmode\\
\textbf{Objetivo}: Atualização dos seus dados pessoais junto da CGD. \\
\textbf{Método}: Atualização dos dados via Chave Móvel Digital (CMD). \\
\textbf{Tempo expectável de execução}: Imediato. \\
\textbf{Custo de novo cartão}: Desde 18,00€ (dezoito euros).

\subsubsection{ActivoBank (AB)}

Deve atualizar os dados entrando na aplicação do ActivoBank fazendo
recurso da sua Chave Móvel Digital (CMD), a alteração deverá ser
imediata mas em alguns casos poderá demorar algumas horas ou até mesmo
dar como recusada ainda que posteriormente seja então processada com
sucesso. \\
\\
\textbf{Documento necessário}:
\begin{itemize}
	\item Chave Móvel Digital (CMD) ativa.
\end{itemize}
\leavevmode\\
\textbf{Objetivo}: Atualização dos seus dados pessoais junto do ActivoBank. \\
\textbf{Método}: Atualização dos dados via Chave Móvel Digital (CMD). \\
\textbf{Tempo expectável de execução}: Até 24 (vinte e quatro) horas. \\
\textbf{Custo de novo cartão}: Desde 15,00€ (quinze euros).

\subsubsection{Moey}

Deve aceder à sua conta Moey e proceder à atualização dos seus dados via
Chave Móvel Digital (CMD), poderá ocorrer um erro na atualização, no
entanto os seus dados irão aparecer corrigidos após algumas horas. A
alteração deve ter efeito imediato mas nem sempre acontece. \\
\\
\textbf{Documento necessário}:
\begin{itemize}
	\item Chave Móvel Digital (CMD) ativa.
\end{itemize}
\leavevmode\\
\textbf{Objetivo}: Atualização dos seus dados pessoais junto do Moey. \\
\textbf{Método}: Atualização dos dados via Chave Móvel Digital (CMD). \\
\textbf{Tempo expectável de execução}: Até 12 (doze) horas. \\
\textbf{Custo de novo cartão}: 5,00€ (cinco euros). \\
\\
\textbf{Ligação de referência}:
\begin{itemize}
	\item Moey, ajuda com os dados pessoais, \url{https://support.moey.pt/hc/pt-pt/categories/360003029418-Conta-e-Perfil}
\end{itemize}

\subsubsection{N26}

Deve abrir uma nova conversação (chat) e pedir ao assistente a mudança
de nome e sexo no seu registo, o assistente irá enviar uma mensagem na
aplicação móvel onde então deverá anexar o novo Cartão de Cidadão e nova
Certidão de Nascimento. A alteração poderá levar alguns dias a ser
finalizada. \\
\\
\textbf{Documentos necessários}:
\begin{itemize}
	\item Cópia traçada do novo Cartão de Cidadão;
	\item Cópia da Certidão/Assento de Nascimento.
\end{itemize}
\leavevmode\\
\textbf{Objetivo}: Atualização dos seus dados pessoais junto do N26. \\
\textbf{Método}: Envio de pedido de suporte com anexação do CC e Certidão de Nascimento. \\
\textbf{Tempo expectável de execução}: Até 15 (quinze) dias. \\ 
\textbf{Custo de novo cartão}: Desde 10,00€ (dez euros). \\
\\
\textbf{Ligação de referência}:
\begin{itemize}
	\item N26, alterar o telemóvel, morada ou outros dados pessoais, \url{https://support.n26.com/en-eu/account-and-personal-details/personal-information-and-data/change-my-phone-number-address-or-other-personal-data}
\end{itemize}

\newpage

\subsubsection{Wise}

Para alterar os seus dados junto do Wise deve seguir os seguintes
passos: Após efetuar o início de sessão no portal deve aceder a
\url{https://wise.com/help/contact/flows/general/what-do-you-need-to-change/my-name}
onde poderá fazer o carregamento de um documento comprovativo da mudança
de nome. A alteração poderá não ser aceite na primeira tentativa.
Tipicamente finalizado dentro de 15 dias. Poderá ser necessário pedir um
segundo ticket ou ajuda no chat para retificar o \emph{wisetag}
associado à conta. \\
\\
\textbf{Documento necessário}:
\begin{itemize}
	\item Cópia traçada do novo Cartão de Cidadão.
\end{itemize}
\textbf{Objetivo}: Atualização dos seus dados pessoais junto do Wise. \\
\textbf{Método}: Envio de pedido de suporte anexando o CC. \\
\textbf{Tempo expectável de execução}: Até 15 (quinze) dias. \\
\textbf{Custo novo cartão}: Desde 6,00€ (seis euros).

\subsubsection{Revolut}

Para alterar o seu nome na aplicação móvel da Revolut deve seguir os
seguintes passos: \\
\begin{itemize}
	\item Tocar no ícone do seu perfil no canto superior esquerdo;
	\item Selecionar ``Conta'' e ``Dados pessoais'';
	\item Tocar no seu nome para o editar e seguir as instruções.
\end{itemize}
\leavevmode\\
\textbf{Objetivo}: Atualização dos seus dados pessoais junto da Revolut. \\
\textbf{Método}: Envio de pedido de suporte anexando o CC. \\
\textbf{Tempo expectável de execução}: Até 7 (sete) dias. \\
\textbf{Custo de novo cartão}: Desde 6,00€ (seis euros).\\
\\
\textbf{Ligação de referência}:
\begin{itemize}
	\item Revolut, alterar o meu nome, \url{https://help.revolut.com/pt-PT/help/profile-and-plan/profile-plan/profile-settings/how-do-i-change-my-name/}
\end{itemize}

\subsubsection{MBWay}

Deve alterar os dados no perfil de utilizador, a alteração tem efeito
imediato após a submissão. O nome que surge quando alguém lhe envia
dinheiro poderá estar dependente do nome associado ao cartão bancário,
isso significa que que poderá ser necessário primeiro pedir um novo
cartão antes deste ficar devidamente atualizado.

\subsubsection{Outras instituições bancárias, de crédito, financeiras ou relacionadas}

Se dispor de acesso a um portal da instituição deve procurar a
possibilidade de atualização de dados através desse. Se não for possível
deve procurar o contacto e enviar uma mensagem de correio eletrónico com
o pedido de alteração invocando os direitos à luz do RGPD. Em
alternativa secundária poderá deslocar-se a um estabelecimento da
instituição para proceder à alteração com o seu novo Cartão de Cidadão.

\subsubsection{Cartões bancários}

Deve pedir sempre que aplicável e assim que for conveniente um cartão de
substituição por cada onde tenha referência ao nome anterior, os valores
dependem da instituição bancária mas tipicamente até 20,00€ (vinte
euros) por cada cartão. O tempo de finalização varia consoante as
emissões de novos cartões.

\subsubsection{Sugestões de contribuição}

Se souber ou quiser contribuir com outros processos listam-se algumas
sugestões de contribuição:
\begin{itemize}
	\item Novo Banco;
	\item Santander;
	\item Millenium BCP;
	\item Montepio;
	\item Crédito Agrícola;
	\item BBVA;
	\item Cofidis;
	\item Cetelem;
	\item Oney;
	\item Credibom;
	\item Unicre;
	\item \ldots{} .
\end{itemize}

\subsection{Seguros}

\subsubsection{Fidelidade}

Deve efetuar um pedido junto do mediador da alteração de dados pessoais
com os novos documentos através de comunicação eletrónica, como mensagem
WhatsApp ou mensagem de correio eletrónico, ou deslocando-se a
fisicamente ao mediador. O processo é tipicamente finalizado em 3 dias. \\
\\
\textbf{Documento necessário}:
\begin{itemize}
	\item Cópia traçada do novo Cartão de Cidadão.
\end{itemize}
\leavevmode\\
\textbf{Objetivo}: Atualização dos seus dados pessoais junto da Fidelidade. \\ 
\textbf{Método}: Envio de pedido de retificação dos dados com CC anexo. \\
\textbf{Tempo expectável de execução}: Até 15 (quinze) dias. \\
\\
\textbf{Ligação de referência}:
\begin{itemize}
	\item Fidelidade, aplicação MyFidelidade, \url{https://www.fidelidade.pt/PT/particulares/Paginas/MyFidelidade.aspx}
\end{itemize}

\subsubsection{Logo Seguros}

Deve efetuar novo pedido no portal da Logo com o assunto ``direitos dos
titulares'', selecionar o produto correspondente e selecionar
``retificação de dados'', é possível que possa necessitar de fazer mais
de um pedido. O processo poderá levar até 30 dias. \\
\\
\textbf{Documento necessário}:
\begin{itemize}
	\item Cópia traçada do novo Cartão de Cidadão.
\end{itemize}
\leavevmode\\
\textbf{Objetivo}: Atualização dos seus dados pessoais junto da Logo Seguros. \\
\textbf{Método}: Envio de pedido de retificação dos dados com CC anexo. \\
\textbf{Tempo expectável de execução}: Até 30 (trinta) dias.

\subsubsection{AdvanceCare/Generali}

Deve efetuar pedido junto do seu mediador ou em alternativa efetuar o
seu pedido por escrito e enviar através de comunicação eletrónica para o
endereço de correio eletrónico
\href{mailto:clientes@tranquilidade.pt}{\nolinkurl{clientes@tranquilidade.pt}}. \\
\\
\textbf{Documentos necessários}:
\begin{itemize}
	\item Modelo de carta, Pedido de alteração/retificação de dados pessoais;
	\item Cópia traçada do novo Cartão de Cidadão.
\end{itemize}
\leavevmode\\
\textbf{Objetivo}: Atualização dos seus dados pessoais junto da Fidelidade. \\
\textbf{Método}: Envio de pedido de retificação dos dados com CC anexo. \\
\textbf{Tempo expectável de execução}: Até 15 (quinze) dias. \\
\\
\textbf{Ligação de referência}:
\begin{itemize}
	\item Generali/Tranquilidade, ajuda, \url{https://www.generalitranquilidade.pt/particulares/servicos-online/area-de-cliente}
\end{itemize}
\leavevmode\\
\newpage

\subsubsection{Sugestões de contribuição}

Se souber ou quiser contribuir com outros processos listam-se algumas
sugestões de contribuição:
\begin{itemize}
	\item Médis;
	\item MGEN;
	\item Medicare;
	\item KeepWells;
	\item Ageas Seguros;
	\item Allianz;
	\item Saúde Prime;
	\item Real Vida;
	\item N Seguros;
	\item Una Seguros;
	\item Zurich;
	\item \ldots{} .
\end{itemize}

\subsection{Telecomunicações}

\subsubsection{MEO/MEO Empresas}

Deve alterar os dados de registo no portal em combinação com alteração
dados de titularidade dos contratos também no portal, encontram-se nos
detalhes do serviço, carregando no botão de pedir alteração
titularidade, pedido de alteração de dados ao abrigo do RGPD e anexar o
novo Cartão de Cidadão ou PDF certificado do \emph{gov.pt} no formulário
e submeter. Em alternativa pode ser efetuado um pedido junto do gestor
de conta se existir. Em alternativa secundária poderá ser feito pedido
de alteração em loja física com o novo Cartão de Cidadão. Deve ser
confirmado o nome que consta na autorização de Débito Direto se
aplicável através do portal e retificar o nome editando o campo
disponível. \\
\\
\textbf{Documentos necessários}:
\begin{itemize}
	\item Modelo de carta, Pedido de alteração/retificação de dados pessoais;
	\item Cópia traçada do novo Cartão de Cidadão.
\end{itemize}
\leavevmode\\
\textbf{Objetivo}: Atualização dos seus dados pessoais junto da Fidelidade. \\
\textbf{Método}: Envio de pedido de retificação dos dados com CC anexo. \\
\textbf{Tempo expectável de execução}: Até 15 (quinze) dias. \\
\\
\textbf{Ligação de referência}:
\begin{itemize}
	\item MEO, alteração dos dados, \url{https://www.meo.pt/ajuda-e-suporte/produtos-meo/gerir-produtos/alteracoes-de-contrato}
\end{itemize}

\subsubsection{Vodafone}

Deve ir a definições do perfil no portal My Vodafone, editar perfil e
alterar o nome, imediato apos pedido, pedidos de alteração relativo aos
contratos é variável, em alternativa deve deslocar-se a uma loja ou
enviar um carta registada para a sede. \\
\\
\textbf{Documentos necessários}:
\begin{itemize}
	\item Modelo de carta, Pedido de alteração/retificação de dados pessoais;
	\item Cópia traçada do novo Cartão de Cidadão.
\end{itemize}
\leavevmode\\
\textbf{Objetivo}: Atualização dos seus dados pessoais junto da Fidelidade. \\
\textbf{Método}: Envio de pedido de retificação dos dados com CC anexo. \\
\textbf{Tempo expectável de execução}: Até 15 (quinze) dias. \\
\\
\textbf{Ligação de referência}:
\begin{itemize}
	\item My Vodafone, ajuda, \url{https://ajuda.vodafone.pt/a-minha-conta/dados-pessoais/como-altero-os-dados-pessoais-no-my-vodafone}
\end{itemize}

\subsubsection{Outros operadores}

Pedido numa loja física ou remeter um email com o pedido por escrito
devidamente assinado e com anexos necessários para verificação dos
dados. \\
\\
\textbf{Documentos necessários}:
\begin{itemize}
	\item Modelo de carta, Pedido de alteração/retificação de dados pessoais;
	\item Cópia traçada do novo Cartão de Cidadão.
\end{itemize}
\leavevmode\\
\textbf{Objetivo}: Atualização dos seus dados pessoais junto da Fidelidade. \\
\textbf{Método}: Envio de pedido de retificação dos dados com CC anexo. \\
\textbf{Tempo expectável de execução}: Até 15 (quinze) dias.

\subsubsection{Sugestões de contribuição}

Se souber ou quiser contribuir com outros processos listam-se algumas
sugestões de contribuição:
\begin{itemize}
	\item NOS;
	\item Digi/Nowo;
	\item Lycamobile;
	\item \ldots{} .
\end{itemize}