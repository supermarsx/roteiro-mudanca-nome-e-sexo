%---------------------
% ALTERAÇÕES INICIAIS
%---------------------

\newpage

\section{Alterações iniciais}

\subsection{Preparações prévias}

\textbf{Palavras-chave}: correio eletrónico, preparação, assinatura manual.

\subsubsection{Relevância das preparações prévias}

Antes de iniciar o seu processo de mudança de nome e sexo será útil
fazer umas preparações prévias de modo a agilizar e facilitar todos os
processos que irá iniciar posteriormente, entre as várias preparações
destacam-se a alteração do endereço de correio eletrónico e a nova
assinatura manuscrita.

\subsubsection{Endereço de correio eletrónico}

Para obter um novo endereço de correio eletrónico de forma gratuita
poderá inscrever-se junto de um do grandes fornecedores tais como a
\emph{Google} ou a \emph{Microsoft}. A \emph{Google} fornece correio
através do serviço \emph{GMail} e a \emph{Microsoft} através do serviço
Outlook online. É importante que o novo endereço seja composto por
termos neutrais, que inclua os seus nomes (pelo menos primeiro e
último), sem nomes estranhos (como por exemplo ``\emph{fofinhax}'' ou
``\emph{xpto}'') ou referências populares e profissionais, o seu futuro
eu irá agradecer. \\
\\
\textbf{Ligações de referência}:
\begin{itemize}
	\item Google, correio eletrónico da Google (GMail), \url{https://mail.google.com/mail/u/0/}
	\item Microsoft, correio eletrónico da Microsoft (Outlook), \url{https://outlook.live.com/mail/about/index\_pt.html}
\end{itemize}

\subsubsection{Assinatura manuscrita}

Ao mudar de nome provavelmente também terá de alterar a sua assinatura
manuscrita, para isso é importante que antes de ir fazer o seu novo
Cartão de Cidadão pratique pelo menos algumas vezes a sua nova
assinatura. Poderá achar que é uma boa altura para fazer uma assinatura
diferente do seu habitual, para conseguir fazer essa mudança com sucesso
é sugerido que faça uma pesquisa prévia, procure alguma inspiração na
internet e treine a bastante a sua nova assinatura. \\
\\
\textbf{Ligações de referência}:
\begin{itemize}
	\item WikiHow, como fazer uma assinatura bonita, \url{https://pt.wikihow.com/Ter-uma-Assinatura-Bonita}
\end{itemize}

\subsubsection{Ativar a Chave Móvel Digital}

Para ativar a sua chave móvel poderá utilizar um dos guias disponíveis
na internet. \\
\\
\textbf{Ligações de referência}:
\begin{itemize}
	\item Autenticação gov, pedido de Chave Móvel Digital, \url{https://www.autenticacao.gov.pt/cmd-pedido-chave}
	\item gov.pt, ativar a sua Chave Móvel, \url{https://www.gov.pt/servicos/ativar-a-chave-movel-digital}
\end{itemize}

\subsection{Identificação essencial}

\subsubsection{Introdução e requisitos}

Para iniciar o processo de mudança de nome e sexo terá de começar por
alterar dois documentos de identificação essenciais, a Certidão/Assento
de Nascimento e a seguir o seu Cartão de Cidadão. Estes primeiros passos
podem levar até um mês para serem finalizados. Há que notar que assim
que tenha a nova Certidão/Assento de Nascimento já poderá começar a
efetuar algumas das alterações sendo que o mais útil será esperar pelo
menos até ter o pedido de novo Cartão de Cidadão em mãos, pois
geralmente com esse documento já irá conseguir alterar outros com mais
facilidade. Ambos os processos apenas podem ser feitos presencialmente
numa Conservatória do Registo Civil sendo que o agendamento prévio não é
estritamente necessário. Poderá pedir aconselhamento enquanto estiver na
conservatória se fará sentido agendar a renovação do cartão de cidadão
por uma questão de conveniência ou facilidade processual. \\
\\
\textbf{Custos diretos totais expectáveis nesta fase}: Pelo menos 17,00€ (dezassete euros). \\
\textbf{Tempo total expectável de execução da fase}: Cerca de 22 (vinte e dois) dias. \\
\\
\textbf{Pré-requisitos desta fase:}
\begin{itemize}
	\item \textbf{Cartão de cidadão} válido e em relativo bom estado;
	\item \textbf{Requerimento} para \textbf{mudança da menção de nome e sexo} devidamente preenchido e assinado de acordo com a assinatura do Cartão de Cidadão;
	\item Presença dos pais (na eventualidade de se tratar de um menor entre os 16 (dezasseis) e os 18 (dezoito) anos).
\end{itemize}
\leavevmode\\
\textbf{Objetivos desta fase}:
\begin{itemize}
	\item Retificação e obtenção de novo \textbf{Assento de Nascimento};
	\item Obtenção de um novo \textbf{Cartão de Cidadão}.
\end{itemize}
\leavevmode\\
\textbf{Fluxo genérico de acontecimentos desta fase}:
\begin{itemize}
	\item Requer alteração do Assento de Nascimento;
	\item Obter o novo Assento de Nascimento;
	\item Pedir novo Cartão de Cidadão;
	\item Obter o novo Cartão de Cidadão;
	\item Ativar o novo Cartão de Cidadão;
	\item Ativar a assinatura digital qualificada;
	\item Ativar a Chave Móvel Digital.
\end{itemize}

\subsubsection{Certidão de nascimento}

\textbf{Palavras-chave}: marcador de sexo, assento de nascimento, registo civil, conservatória. \\
\\
Para efetuar o pedido de nova certidão de nascimento deverá deslocar-se
a uma conservatória do registo civil, com o requerimento de mudança de
nome e sexo já devidamente preenchido preferencialmente. No local poderá
tirar uma senha de ``Registo Civil - Certidões / Informações'', no
entanto se tiver a aplicação móvel \emph{SIGA} poderá tirar uma senha
digitalmente para maior comodidade e atendimento mais rápido. Quando a
sua senha for chamada deverá entregar no balcão o seu requerimento e
atual cartão de cidadão. Que estiver a efetuar o atendimento irá fazer
alguma questões e confirmar a vontade de alteração, preencher uns
documentos adicionais e pedir a sua assinatura no pedido formalizado.
Após efetuar o pedido será informado o tempo médio até à conclusão do
seu pedido onde provavelmente irão dizer qual a data para voltar e poder
recolher a sua nova certidão de nascimento. Neste processo poderá
dependendo da vontade do outro lado de agendar a execução do novo cartão
de cidadão. \\
\\
\textbf{Documentos necessários}:
\begin{itemize}
	\item Cartão de Cidadão em bom estado;
	\item Requerimento para mudança da menção de nome e sexo devidamente preenchido e assinado de acordo com a assinatura do Cartão de Cidadão;
	\item Presença dos pais (na eventualidade de se tratar de um menor).
\end{itemize}
\leavevmode\\
\textbf{Objetivo}: Obter uma nova Certidão de Nascimento. \\
\textbf{Método}: Deslocação a uma Conservatória do Registo Civil. \\
\textbf{Custo deste processo}: Gratuito. \\
\textbf{Entrega da nova	certidão}: No local do pedido em mãos. \\
\textbf{Tempo expectável de	execução}: Até 7 (sete) dias. \\
\\
\textbf{Ligações de referência}:
\begin{itemize}
	\item justiça.gov.pt, mudança de nome e sexo, \url{https://justica.gov.pt/Registos/Civil/Mudanca-de-sexo-e-de-nome-proprio}
	\item gov.pt, pedir o registo de mudança de sexo e de nome próprio, \url{https://www.gov.pt/servicos/pedir-o-registo-de-mudanca-de-sexo-e-de-nome-proprio}
\end{itemize}

\newpage

\paragraph{Dica para residentes em Lisboa}
\leavevmode\\
Como alternativa mais célere para os residentes em Lisboa, é possível o
envio de uma mensagem de correio eletrónico, para os seguintes
endereços:\\
\\
\href{mailto:direcao.apoio.civil.lisboa@irn.mj.pt}{\nolinkurl{direcao.apoio.civil.lisboa@irn.mj.pt}} \\
\href{mailto:civil.lisboa@irn.mj.pt}{\nolinkurl{civil.lisboa@irn.mj.pt}}
\\
\leavevmode\\
Na sua mensagem deve incluir a seguinte informação:
\begin{itemize}
	\item Nome Completo;
	\item Dados completos do atual Cartão de Cidadão (CC), poderá ser uma cópia
	traçada;
	\item Contacto telefónico;
	\item Endereço de correio eletrónico;
	\item Nome(s) pretendido(s);
	\item Género pretendido.
\end{itemize}
\leavevmode\\
A Conservatória do Registo Civil irá contactar posteriormente no sentido
de confirmar o agendamento e dar seguimento à confirmação, bem como uma
resposta acerca das informações necessárias em resposta à mensagem
enviada previamente. \\
\\
Existe relatos de situações onde não houve lugar a pagamento do Cartão
de Cidadão, não deve no entanto ser considerado a norma.

\paragraph{Certidão de nascimento dos filhos}
\leavevmode\\
O averbamento dos assentos de nascimento dos filhos da pessoa que
efetuou a mudança de nome e sexo tem algumas limitações importantes
relevar:
\begin{itemize}
	\item O averbamento da certidão de nascimento de filho(a) já nascido maior só poderá ser feito mediante pedido desse(a) filho(a);
	\item O averbamento da certidão de nascimento de filho(a) menor não é possível de acordo com a legislação vigente.
\end{itemize}

\paragraph{Certidão de casamento}
\leavevmode\\
Se a pessoa for casada só poderá ver o assento de casamento alterado por
consentimento de ambas as partes.

\paragraph{Reversão da alteração do assento}
\leavevmode\\
É importante notar que na eventualidade de a pessoa querer reverter o
seu assento de nascimento à menção de nome e sexo original só o poderá
ver feito mediante autorização judicial.

\paragraph{Confidencialidade do ato}
\leavevmode\\
A mudança da menção de nome e sexo é um ato secreto, onde apenas a
própria pessoa, os seus herdeiros, e entidades judiciais ou policiais no
seguimento de um processo de instrução criminal ou decisão judicial o
podem revelar. O pedido de certidão de nascimento que habitualmente é
aberto ao público nestes casos torna-se vedado, podendo apenas o próprio
requer tal certidão.

\subsubsection{Cartão de Cidadão}

\textbf{Palavras-chave}: identificação, essencial, bilhete de identidade, conservatória, marcador de sexo. \\

\paragraph{Início do processo}
\leavevmode\\
O novo cartão de cidadão deve ser pedido e feito numa conservatória do
registo civil. É preferível agendar previamente de modo a agilizar o
processo e evitar atrasos ou falta de senha na conservatória pretendida.
No dia do seu agendamento deve levar a nova certidão de nascimento (ou
se for levantar a sua nova certidão irão encaminhar a sua certidão) para
então fazer o novo cartão de cidadão, idealmente deverá ter praticado a
sua nova assinatura. Finalizado o processo irá obter um documento que
confirma o pedido de novo cartão de cidadão, com esse documento já
poderá pedir ou alterar alguns dos seus dados noutros locais, tais como,
passe de transporte ou uma ficha cliente. \\
\\
\textbf{Documentos necessários}:
\begin{itemize}
	\item Cartão de Cidadão atual;
	\item Novo Assento/Certidão de Nascimento.
\end{itemize}
\leavevmode\\
\textbf{Objetivo}: Obter um novo Cartão de Cidadão. \\
\textbf{Método}: Deslocação a uma Conservatória do Registo Civil. \\
\textbf{Custo deste	processo}: A partir de 17,00€ (dezassete euros). \\
\textbf{Entrega do novo	cartão de cidadão}: No local de pedido ou no domicílio. \\
\textbf{Tempo expectável de execução}: Até 15 (quinze) dias. \\
\\
\textbf{Ligações de referência}:
\begin{itemize}
	\item justica.gov.pt, Agendar Cartão de Cidadão, \url{https://justica.gov.pt/Servicos/Agendar-Cartao-de-Cidadao}
	\item gov.pt, renovar o Cartão de Cidadão, \url{https://www.gov.pt/servicos/renovar-o-cartao-de-cidadao}
\end{itemize}

\paragraph{No momento da recolha e/ou ativação do Cartão de Cidadão	na Conservatória do Registo	Civil}
\leavevmode\\[4pt]
Pode ativar a assinatura digital qualificada para facilitar a entrega de
documentos digitalmente assinados a terceiros, como entidades estatais
ou empresas. Este passo é opcional, mas geralmente é preferível ativar a
assinatura no momento da ativação do Cartão na conservatória por um
questão de comodidade. \\
\\
Os dados do portal \emph{Autenticação Gov} só são atualizados no momento
da entrega e ativação do novo Cartão de Cidadão bem como a Chave Móvel
Digital. O registo da Chave Móvel é possível ser feito utilizando o
mesmo número de telemóvel da Chave Móvel anterior aquando a sua ativação
na Conservatória do Registo Civil.

\subsection{Principais instituições/entidades}

\subsubsection{Introdução e	requisitos}

Para iniciar esta fase irá incidir sobre as outras principais
instituições e entidades nas quais precisa de garantir que os seus dados
pessoais encontram-se devidamente atualizados. \\
\\
\textbf{Custos totais diretos expectáveis nesta fase}: Nenhum. \\
\textbf{Tempo total expectável de execução da fase}: Cerca de 20 (vinte)
dias. \\
\\
\textbf{Pré-requisitos desta fase}:
\begin{itemize}
	\item Acesso ao portal da \textbf{Autoridade Tributária};
	\item Acesso ao portal \textbf{Segurança Social} direta;
	\item Acesso ao portal do \textbf{Sistema Nacional de Saúde} (\emph{SNS24});
	\item Acesso à \textbf{aplicação móvel \emph{gov.pt}} com os seus documentos;
	\item \textbf{Chave Móvel Digital} ativa.
\end{itemize}
\leavevmode\\
\textbf{Objetivos desta fase}:
\begin{itemize}
	\item Verificar os seus dados pessoais junto da \textbf{Autoridade Tributária}, \textbf{Segurança Social} e \textbf{Sistema Nacional de Saúde};
	\item Retificar os dados junto do \textbf{Ministério da Defesa Nacional};
	\item Obter uma nova \textbf{cédula militar};
	\item Obter o \textbf{estatuto de objetor de consciência} (opcional);
	\item Obter uma lista das suas contas e responsabilidades junto do \textbf{Banco de Portugal} e a titularidade.
\end{itemize}
\leavevmode\\
\textbf{Fluxo genérico de acontecimentos}:
\begin{itemize}
	\item Verificar os seus dados no portal da Autoridade Tributária;
	\item Verificar os seus dados no portal da Segurança Social direta;
	\item Verificar os seus dados no portal \emph{SNS24};
	\item Efetuar pedido de retificação dos seus dados junto do Ministério da Defesa Nacional (MDN);
	\item Obter uma nova cédula militar atualizada;
	\item Efetuar pedido do estatuto de objetor de consciência (opcional);
	\item Verificar e despoletar a atualização dos dados no recenseamento eleitoral;
	\item Verificar os seus dados pessoais junto do Banco de Portugal.
\end{itemize}

\subsubsection{Autoridade Tributária (AT)}

\textbf{Palavras-chave}: impostos, autoridade tributária, at, fiscal. \\
\\
Efetuado automaticamente após o pedido de novo cartão de cidadão. O
processo é finalizado instantaneamente sem intervenção. Inclui o portal
da Autoridade Tributária. \\
\\
\textbf{Documento necessário}:
\begin{itemize}
	\item Acesso ao portal da Autoridade Tributária.
\end{itemize}
\leavevmode\\
\textbf{Objetivo}: Atualização dos seus dados junto da Autoridade Tributária. \\
\textbf{Método}: Automático. \\
\textbf{Custo deste processo}: Gratuito. \\
\textbf{Tempo expectável de execução}: Imediato. \\
\\
\textbf{Ligações de referência}:
\begin{itemize}
	\item Acesso gov.pt, entrada no portal da Autoridade Tributária, \url{https://www.acesso.gov.pt/v2/loginForm?partID=PFAP\&path=/geral/dashboard}
	\item Portal da Finanças, entrada, \url{https://www.portaldasfinancas.gov.pt/at/html/index.html}
\end{itemize}

\subsubsection{Segurança Social (SS)}

\textbf{Palavras-chave}: seg social, ss, segurança social, pensões, subsídios, programas sociais. \\
\\
Efetuado automaticamente após o pedido de novo cartão de cidadão. O
processo é finalizado instantaneamente sem intervenção. Inclui o portal
Segurança Social Direta. \\
\\
\textbf{Documento necessário}:
\begin{itemize}
	\item Acesso ao portal da Segurança Social direta.
\end{itemize}
\leavevmode\\
\textbf{Objetivo}: Atualização dos seus dados junto da Segurança Social. \\
\textbf{Método}: Automático. \\
\textbf{Custo deste processo}: Gratuito. \\
\textbf{Tempo expectável de execução}: Imediato. \\
\\
\textbf{Ligações de referência}:
\begin{itemize}
	\item Segurança Social, página sobre a Segurança Social direta, \url{https://www.seg-social.pt/seguranca-social-direta}
	\item Segurança Social, página de início de sessão no portal da Segurança Social direta, \url{https://app.seg-social.pt/sso/login?service=https\%3A\%2F\%2Fapp.seg-social.pt\%2Fptss\%2Fcaslogin}
	\item Google Play Store, aplicação móvel da Segurança Social para sistema Android, \url{https://play.google.com/store/apps/details?id=pt.segsocial.mobile.segurancasocial\&hl=pt\_PT}
	\item Apple App Store, aplicação móvel da Segurança Social para sistema iOS, \url{https://apps.apple.com/pt/app/seguran\%C3\%A7a-social/id1469920521?l=en-GB}
\end{itemize}

\subsubsection{Sistema Nacional de Saúde (SNS)}

\textbf{Palavras-chave}: saúde, sns, sistema público de saúde. \\
\\
Efetuado automaticamente após o pedido de novo cartão de cidadão. O
processo é finalizado instantaneamente sem intervenção. Inclui o portal
\emph{SNS24}. Significando assim que as suas prescrições, exames e
outros documentos relevantes dentro deste âmbito já irão sair com o nome
correto.\\
\\
\textbf{Documento necessário}:
\begin{itemize}
	\item Acesso ao portal do Sistema Nacional de Saúde.
\end{itemize}
\leavevmode\\
\textbf{Objetivo}: Atualização dos seus dados junto do Sistema Nacional de Saúde. \\
\textbf{Método}: Automático. \\
\textbf{Custo deste processo}: Gratuito. \\
\textbf{Tempo expectável de execução}: Imediato. \\
\\
\textbf{Ligações de referência}:
\begin{itemize}
	\item SNS24, entrada, \url{https://www.sns24.gov.pt/pt/inicio}
	\item SNS24, início de sessão, \url{https://www.sns24.gov.pt/pt/login/utente}
	\item SNS24, ajuda da aplicação móvel SNS24, \url{https://www.sns24.gov.pt/pt/servico/app-sns-24}
	\item Google Play Store, Aplicação móvel SNS24 para sistema Android, \url{https://play.google.com/store/apps/details?id=pt.minsaude.spms.ces\&hl=pt\_PT}
	\item Apple App Store, Aplicação móvel SNS24 para sistema iOS, \url{https://apps.apple.com/pt/app/sns-24/id1192353854}
\end{itemize}

\newpage

\subsubsection{Ministério da Defesa	Nacional/DDN/BUD}

\textbf{Palavras-chave}: serviço militar, cédula militar, dia da defesa nacional.

\paragraph{Recenseamento militar (Cédula militar)}
\leavevmode\\[4pt]
Deve enviar um email para
\href{mailto:ddn@defesa.pt}{\nolinkurl{ddn@defesa.pt}} com o assunto
``ALTERAÇÃO DADOS'' incluindo o documento exportado da aplicação móvel
\emph{gov.pt} ou cópia do Cartão de Cidadão. Poderá utilizar o modelo de
mensagem de correio eletrónico para maior comodidade. \\
\\
\textbf{Documentos necessários}:
\begin{itemize}
	\item Modelo de mensagem de correio eletrónico para o Dia da Defesa Nacional/Balcão Único da Defesa (DDN/BUD);
	\item Cópia traçada do Cartão de Cidadão.
\end{itemize}
\leavevmode\\
\textbf{Objetivo}: Obter uma nova cédula militar atualizada. \\
\textbf{Método}: Envio de mensagem de correio eletrónico. \\
\textbf{Custo deste processo}: Gratuito. \\
\textbf{Entrega de nova cédula}: Virtual, no seu endereço de correio eletrónico. \\
\textbf{Tempo expectável de	execução}: Até 20 (vinte) dias. \\
\\
\textbf{Ligações de referência}:
\begin{itemize}
	\item Balcão Único da Defesa, cédula militar, \url{https://bud.gov.pt/ddn/cedula.html}
	\item Balcão Único da Defesa, convocação, \url{https://bud.gov.pt/ddn/convocacao.html}
	\item Balcão Único da Defesa, emitir cédula, \url{https://ddn.dgrdn.gov.pt/cedula\_bud.aspx}
\end{itemize}

\paragraph{Nota especialmente importante}
\leavevmode\\[4pt]
Os dados pessoais têm de ser obrigatoriamente atualizados em
conformidade com o Artigo 57º (quinquagésimo sétimo), alínea c da Lei do
Serviço Militar (Lei n.º 174/99, de 21 de setembro), o não cumprimento
do disposto implicará uma coima de aproximadamente 100,00€ (cem) a
500,00€ (quinhentos euros) de acordo com o Regulamento da Lei do Serviço
Militar (Decreto-Lei n.º 289/2000, de 14 de novembro) que poderá
agravar-se para o dobro em tempo de guerra. Ainda mais na eventualidade
de necessidade de recrutamento ao abrigo do Artigo 37º (trigésimo
sétimo) da LSM define que os ``contingentes da reserva de recrutamento a
classificar para efeitos da convocação (\ldots) obedece aos seguintes
fatores de preferência, por ordem de prioridade: a) Os cidadãos que
hajam injustificadamente faltado ao cumprimento de deveres militares''.

\paragraph{Nota adicional}
\leavevmode\\[4pt]
Poderá ser útil descarregar a cédula antiga antes de proceder ao pedido
de alteração, de modo a ter uma cópia da antiga. Deve descarregar uma
nova cédula assim que obter a confirmação da sua alteração no sistema do
Balcão Único da Defesa (BUD). \\

\paragraph{Como encontrar o seu Número de Identificação Militar (NIM)}
\leavevmode\\[4pt]
Para encontrar o seu Número de Identificação Militar (NIM) deverá
executar os seguintes passos:
\begin{itemize}
	\item Aceder ao portal do DDN através da ligação, \url{https://ddn.dgrdn.gov.pt/ddn\_editaispesq.aspx} ;
	\item Inserir o número de Cartão de Cidadão sem os dígitos de verificação;
	\item Inserir o nome completo em maiúsculas e sem acentuação;
	\item Clicar em ``Pesquisar''.
\end{itemize}

\paragraph{Como emitir a sua cédula militar}
\leavevmode\\[4pt]
Para emitir a sua cédula militar deverá executar os passos abaixo:
\begin{itemize}
	\item Aceder ao portal do DDN através da ligação, \url{https://ddn.dgrdn.gov.pt/cedula\_bud.aspx} ;
	\item Inserir o número de Cartão de Cidadão sem os dígitos de verificação;
	\item Inserir o endereço de correio eletrónico fornecido aquando a inscrição/entrega da cédula no Dia da Defesa Nacional;
	\begin{itemize}
		\item \textbf{Nota}: Na eventualidade de não saber qual o endereço terá de pedir a alteração do endereço de correio eletrónico associado no seu pedido de atualização de dados pessoais.
	\end{itemize}
	\item Confirmar a caixa ``não sou um robô'' e clicar em ``enviar''.
\end{itemize}

\subsubsection{Estatuto de objetor de consciência}

Poderá fazer sentido para algumas pessoas a requisição de estatuto de
objetor de consciência quando a sua convicção de ordem \textbf{moral},
\textbf{religiosa}, \textbf{humanística} ou \textbf{filosófica} não lhe
permite a utilização de \textbf{meios violentos de qualquer natureza
	contra o seu semelhante}, independentemente do âmbito, seja ele pessoal,
coletivo ou de defesa nacional. No entanto não impede a chamada a
efetuar serviços públicos ou comunitários em tempo de guerra. \\
\\
O \textbf{estatuto de objetor de consciência} uma vez reconhecido
\textbf{não permite} que a pessoa seja:
\begin{itemize}
	\item Titular de licença administrativa de detenção, uso e porte de arma de qualquer natureza;
	\item Titular de autorização de uso e porte de arma de defesa quando, por Lei, tal autorização seja inerente à função pública ou privada que exerça;
	\item Trabalhadora no fabrico, reparação ou comércio de armas de qualquer natureza ou no fabrico e comércio das respetivas munições, nem trabalhar em investigação científica relacionada com essas atividades.
	\item Praticante de qualquer outra atividade que exija o uso e porte de arma de qualquer natureza.
\end{itemize}
\leavevmode\\
O pedido do estatuto terá de ser sempre apresentado presencialmente
conjuntamente com todos os documentos devidamente preenchidos e
assinados. \\
\\

\newpage
\leavevmode\\
\textbf{Pedido dos documentos adicionais}
\begin{itemize}
	\item Certificado do registo criminal pode ser obtido online com um custo de 5,00€ (cinco euros);
	\item Certidão de Nascimento pode ser obtida online com um custo de 10,00€ (dez euros), ou 20,00€ (vinte euros) caso necessite em suporte de papel.
\end{itemize}
\leavevmode\\
\textbf{Documentos necessários}:
\begin{itemize}
	\item Certidão de Nascimento válida e verificável;
	\item Certificado do Registo Criminal do declarante;
	\item Formulários, Anexo E - Declaração de objeção de consciência perante o serviço militar;
	\item Anexo F - Declaração abonatória (Relativo ao estatuto de objetor de consciência).
\end{itemize}
\leavevmode\\
\textbf{Objetivo}: Obter o estatuto de objetor de consciência. \\
\textbf{Método}: Deslocação presencial ao Instituto Português do Desporto e Juventude (IPDJ). \\
\textbf{Custo deste processo}: Gratuito mas necessita de documentos pagos. \\
\textbf{Tempo expectável de execução}: Até 30 (trinta) dias. \\
\\
\textbf{Ligações de referência}:
\begin{itemize}
	\item IPDJ, objetores de consciência, \url{https://ipdj.gov.pt/objetores-de-consciencia}
	\item gov.pt, requisição do estatuto de objetor de consciência, \url{https://www2.gov.pt/servicos/requerer-estatuto-de-objetor-de-consciencia-de-servico-militar}
	\item IRN, pedir certidão de nascimento, \url{https://justica.gov.pt/servicos/pedir-certidao-de-nascimento}
	\item justica.gov.pt, pedir certificado do registo criminal, \url{https://registocriminal.justica.gov.pt/}
\end{itemize}

\newpage

\subsubsection{Recenseamento eleitoral}

\textbf{Palavras-chave}: votar, cadernos eleitorais, eleições, mesas de
voto. \\
\\
Por vezes o sistema poderá não atualizar o seu nome automaticamente,
para isso deverá efetuar um acesso ao \textbf{portal do recenseamento}
com o seu novo Cartão de Cidadão ou Chave Móvel Digital (CMD). Para
despoletar a atualização dos seus dados poderá ser útil atualizar os
seus dados de contacto. A alteração despoletada pelo acesso com o seu
novo documento de identificação e alteração dos dados de contacto
tipicamente torna-se visível no portal ao fim de 1 (uma) semana. Não
existem mecanismos de notificação das alterações ou atualização dos
dados pelo que terá de ir consultando para verificar se a alteração já
se encontra finalizada. \\
\\
\textbf{Documento necessário}:
\begin{itemize}
	\item Chave Móvel Digital (CMD) ativa.
\end{itemize}
\leavevmode\\
\textbf{Objetivo}: Atualizar os dados pessoais no recenseamento eleitoral. \\
\textbf{Método}: Autenticação no portal do recenseamento e atualização de contactos. \\
\textbf{Custo deste processo}: Gratuito. \\
\textbf{Tempo expectável de execução}: Até 7 (sete) dias. \\
\\
Aceder à sua página pessoal no portal do recenseamento eleitoral:
\begin{itemize}
	\item Aceder à página do portal do eleitor através da ligação, \url{https://www.eueleitor.mai.gov.pt/Login.aspx} ;
	\item Ir ao separador lateral ``Os meus dados pessoais'';
	\item Clicar no botão de ``editar contactos'';
	\item Atualizar o seu endereço de correio eletrónico e número de telemóvel e confirmar.
\end{itemize}
\leavevmode\\
Em alternativa pode submeter um pedido de alteração ou reclamação
através do formulário de contacto no portal do eleitor na página de
contactos em \url{https://www.portaldoeleitor.pt/pt/Contactos/Pages/default.aspx} . \\
\\
\textbf{Ligações de referência}:
\begin{itemize}
	\item Portal do Eleitor, entrada, \url{https://www.portaldoeleitor.pt/pt/Pages/default.aspx}
	\item Portal do Eleitor, início de sessão, \url{https://www.eueleitor.mai.gov.pt/Login.aspx}
	\item Portal do recenseamento, consulta dos cadernos de recenseamento, \url{https://www.recenseamento.pt/}
\end{itemize}
\leavevmode\\
\subsubsection{Banco de Portugal (BdP)}

\textbf{Palavras-chave}: Banco central, crc, créditos, beneficiário efetivo, contas bancárias. \\
\\
O acesso ao Banco de Portugal é relevante por 2 (duas) razões, a
consulta da titularidade das suas contas bancárias, e acesso à central
de responsabilidades de crédito. Estes dois itens permitem verificar uma
quantidade de informação, poderá verificar as titularidades e as contas
onde é beneficiário efetivo, por outro lado poderá ver os seus atuais
créditos, titularidade, garantias e outros demais incluindo
incumprimentos. Este acesso será maioritariamente útil para garantir que
o seu pedido de retificação de nome foi efetuado com sucesso junto das
várias entidades bancárias ou financeiras. \\
\\
\textbf{Documento necessário}:
\begin{itemize}
	\item Acesso da Autoridade Tributária ou Chave Móvel Digital ativa.
\end{itemize}
\leavevmode\\
\textbf{Objetivo}: Verificar que os dados junto de entidades bancárias ou financeiras ficaram devidamente alterados. \\
\textbf{Método}: Autenticação via acesso da Autoridade Tributária ou Chave Móvel Digital. \\
\textbf{Custo deste processo}: Gratuito. \\
\textbf{Tempo expectável de	execução}: Imediato. \\
\\
\textbf{Ligação de referência}:
\begin{itemize}
	\item Banco de Portugal, entrada, \url{https://www.bportugal.pt/}
	\item Banco de Portugal, área do Cidadão, \url{https://www.bportugal.pt/area-cidadao}
\end{itemize}

\subsection{Outros documentos de identificação}

\subsubsection{Passaporte Eletrónico Português (PEP)}

\textbf{Palavras-chave}: passaporte, viajar, identificação, vistos. \\
\\
Se for viajar, especialmente para fora do espaço \emph{Schengen} poderá
ser útil obter um novo passaporte, para isso é ideal o agendamento numa
Conservatória do Registo Civil, terá de levar o seu novo Cartão de
Cidadão. A entrega do Passaporte no domicílio acresce 10,00€ (dez euros). \\
\\
\textbf{Documento necessário}:
\begin{itemize}
	\item Cartão de Cidadão atualizado.
\end{itemize}
\leavevmode\\
\textbf{Objetivo}: Obter um novo passaporte atualizado. \\
\textbf{Método}: Deslocação a Conservatória do Registo Civil ou Loja do Cidadão. \\
\textbf{Custo deste processo}: A partir de 65,00€ (sessenta e cinco euros). \\
\textbf{Entrega de novo passaporte}: No local de pedido ou no domicílio. \\
\textbf{Tempo expectável de execução}: Até 5 (cinco) dias úteis. \\
\\
\newpage
\leavevmode\\
\textbf{Ligações de referência}:
\begin{itemize}
	\item justica.gov.pt, página do passaporte eletrónico, \url{https://justica.gov.pt/Registos/Identificacao/Passaporte-eletronico}
	\item gov.pt, pedir ou renovar o passaporte eletrónico, \url{https://www2.gov.pt/servicos/pedir-o-passaporte-eletronico-portugues}
\end{itemize}

\subsubsection{Carta de condução}

\textbf{Palavras-chave}: conduzir, licença de condução, veículos, viação. \\
\\
Deve efetuar o pedido de uma carta de substituição no portal IMT Online
com a justificação de mudança de nome. O pedido se for dentro de um
prazo razoável coincidente com a renovação da sua carta de condução
poderá ser preferível esperar pelo prazo elegível de renovação que
começa 6 meses antes do prazo de validade. Ao efetuar o pedido através
do IMT Online será gerada uma referência multibanco mas essa só ficará
ativa passadas 24 (vinte e quatro) horas após a submissão do pedido,
isto significa que o pagamento nunca é feito imediatamente, o prazo de
pagamento é de 10 (dez) dias após a submissão do pedido. \\
\\
\textbf{Documento necessário}:
\begin{itemize}
	\item Chave Móvel Digital (CMD) ativa ou acesso do portal da Autoridade Tributária.
\end{itemize}
\leavevmode\\
\textbf{Objetivo}: Obter uma nova carta de condução atualizada. \\
\textbf{Método}: Pedido de substituição por via eletrónica. \\
\textbf{Custo deste processo}: 27,00€ (vinte e sete euros) quando pedido online, 30,00€ (trinta euros) quando presencial. \\
\textbf{Entrega de novo	passaporte}: No domicílio. \\
\textbf{Tempo expectável de execução}: Até 20 (vinte) dias. \\
\\
\textbf{Ligações de referência}:
\begin{itemize}
	\item IMT, a minha carta de condução, \url{https://aminhacartadeconducao.imt-ip.pt/}
	\item IMT Online, carta de condução, \url{https://servicos.imt-ip.pt/Condutores/CartadeCondu\%C3\%A7\%C3\%A3o.aspx}
	\item gov.pt, revalidar Carta de Condução, \url{https://www.gov.pt/servicos/revalidar-a-carta-de-conducao}
	\item gov.pt, substituir Carta de Condução, \url{https://www2.gov.pt/servicos/substituir-a-carta-de-conducao}
	\item gov.pt, 2ª via da Carta de Condução, \url{https://www2.gov.pt/servicos/pedir-a-segunda-via-da-carta-de-conducao}
\end{itemize}