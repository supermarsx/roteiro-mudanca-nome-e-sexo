%---------------
% NOTAS PRÉVIAS
%---------------

\newpage

\section{Notas prévias}

\subsection{Pressupostos}

\textbf{Palavras-chave}: condicionantes, utilidade, invalidez, circunstâncias, individualidades. \\
\\
A extensão da utilidade deste roteiro baseia-se em alguns pressupostos
que podem impactar significativamente a eficácia e utilidade do
documento, ainda que estas possam ser afetadas pelo não cumprimento de
pressupostos, esses pressupostos não determinam a invalidade do roteiro
como um todo mas sim apenas o facto da não aplicabilidade à
circunstância específica da pessoa. \\
\\
Apontam-se alguns desses pressupostos:
\begin{itemize}
	\item Ter \textbf{nacionalidade Portuguesa};
	\item Ser detentor de \textbf{Cartão de Cidadão};
	\item Estar fisicamente \textbf{presente em território nacional};
	\item Ser \textbf{maior de idade};
	\item Ter capacidade psicológica, autonomia de decisão e \textbf{idoneidade};
	\item Ter competências informáticas básicas;
	\item Não estar a cumprir uma pena de prisão;
	\item Não estar numa situação de internamento;
	\item Não estar em processo de insolvência;
	\item Não estar numa circunstância de incapacidade agravada ou permanente;
	\item \ldots{} .
\end{itemize}
\leavevmode\\
A lista de pressupostos é demonstrativa e não exaustiva, não são
mencionadas todas as potenciais condicionantes dado a existência de
muitas experiências de vida diferentes onde seria difícil enumerar todos
os ângulos da experiência de vida de modo a refletir uma visão completa
das condicionantes.

\subsection{Pré-requisitos}

\textbf{Palavras-chave}: requisitos necessários, requisitos preferenciais, documentos, recursos financeiros. \\
\\
Antes de começar o seu processo de mudança de nome e sexo, é fundamental
observar um conjunto de requisitos, a existência dos elementos listados
como pré-requisito irão facilitar os processos individuais posteriores
que irá despoletar ao iniciar o seu processo.

\subsubsection{Requisitos essenciais}

Identifica-se assim os requisitos essenciais para iniciar o seu
processo:
\begin{itemize}
	\item \textbf{Cartão de cidadão}, deve estar em bom estado, legível e o respetivo chip funcional (fator preferencial);
	\item \textbf{Tempo}, poderá ter de dispensar até 6 (seis) horas na identificação essencial, no entanto pode chegar a mais de 30 (trinta) dependendo da quantidade de locais e dados a alterar;
	\item \textbf{Financeiros}: No mínimo 17,00€ (dezassete euros) para o Cartão de Cidadão e concluir a identificação essencial, podendo no entanto ser superior a 150,00€ (cento e cinquenta euros) consoante o número de documentos, cartões e processos pelos quais tenha de pagar.
\end{itemize}

\subsubsection{Requisitos preferenciais}

Os seguintes requisitos são preferenciais para que consiga efetuar o seu
processo de forma mais célere, menos burocrática e mais prática:
\begin{itemize}
	\item \textbf{Chave Móvel Digital (CMD)}, deve estar ativa e sem bloqueios;
	\item \textbf{Aplicação móvel \emph{gov.pt}}, ativada com todos os seus documentos de identificação relevantes, tais como o Cartão de Cidadão (CC), Certificado de Matrícula (DUA), Carta de Condução, e outros);
	\item \textbf{Acesso ao portal internet da Autoridade Tributária (AT)}, ativo e sem bloqueios;
	\item \textbf{Acesso ao portal internet Segurança Social Direta}, ativo e sem bloqueios;
	\item \textbf{Acesso ao portal e/ou aplicação móvel \emph{SNS24}}, ativo e sem bloqueios;
	\item \textbf{Acessos individuais aos vários portais de cada uma das empresas, organizações ou instituições com as quais tenha relações comerciais ou institucionais preestabelecidas}, acessos devem estar funcionais e sem condicionantes;
	\item \textbf{Acesso a novo endereço de correio eletrónico adequado para a nova identificação}, é ideal a existência de um novo endereço de correio eletrónico adequado, contendo por exemplo o novo nome;
	\item \textbf{Leitor de cartões (opcional)}, em bom estado e funcional, poderá ser útil em determinadas circunstâncias;
	\item \textbf{Aplicação móvel \emph{SIGA}}, poderá facilitar o agendamento ou obtenção de uma senha para o serviço pretendido;
	\item \textbf{Domicílio/Morada atualizada}, é importante por uma questão de facilidade ter a sua morada do cartão de cidadão atualizada;
	\item \textbf{Número de telemóvel atualizado}, poderá ser importante garantir previamente que o número de telemóvel associado aos vários acessos esteja atualizado, especialmente na Chave Móvel Digital (CMD).
\end{itemize}

\subsubsection{Perguntas e respostas}

\textbf{Palavras-chave}: Q\&A, FAQ, perguntas frequentes. \\

\paragraph{É necessário algum relatório médico para efetuar a mudança?}
\leavevmode\\
\textbf{Maiores de idades} não necessitam de qualquer relatório médico; \\
\textbf{Menores de idade} (entre os 16 e os 18 anos) necessitam apenas
de um relatório médico de um médico ou psicólogo registado na respetiva
Ordem profissional a atestar a decisão livre e informada da pessoa. \\

\paragraph{Menores que 16 anos podem efetuar a mudança de nome na sua identificação?}
\leavevmode\\
\textbf{Não}, podem apenas pedir a anotação do nome e identidade
preferencial ao abrigo da legislação em vigor, relativo à utilização de
nome social/preferencial. \\

\paragraph{É possível reverter o processo de mudança de nome?}
\leavevmode\\
\textbf{Sim (mas)}, no entanto apenas poderá fazê-lo através de
autorização judicial. É importante dar a mudança como permanente e
evitar a necessidade de reversão do processo. \\

\subsubsection{Inventariação dos itens a alterar}

\textbf{Palavras-chave}: organização, lista, to-do, a fazeres. \\
\\
Para que consiga efetuar a sua mudança com sucesso será útil
\textbf{inventariar por itens todas as alterações} que terá de efetuar
futuramente, esta listagem irá permitir dar a \textbf{prioridade}
adequada aos itens consoante a sua relevância e/ou impacto no seu
dia-a-dia. \\
\\
A melhor forma de começar a listagem é pensar naquilo que utiliza na
vida diária sem exceção e a partir daí ir pensando no restante que
utiliza com menos frequência. Outro método para conseguir listar os
itens poderá ser através da análise dos seus apontamentos (se tiver) ou
listagens de acessos a plataformas online ou sites, isso poderá dar um
ponto de partida alternativo ou até mesmo auxiliar o inventário daquilo
que irá precisar de alterar no futuro. \\
\\
A título de exemplo podemos considerar o seguinte pensamento, ``no
dia-a-dia o mais essencial é a minha \textbf{identificação} (o Cartão de
Cidadão), o \textbf{passe de transportes} públicos, e como também
conduzo com frequência também irei precisar de uma nova \textbf{Carta de
	Condução}''. O objetivo é conseguir determinar aquilo que é o
\textbf{mais importante} e daí ir listando outros itens menos
importantes ou prioritários para mais tarde. \\

\newpage

\paragraph{Lista de exemplo}
\leavevmode\\\\
Processo de mudança de nome, alterações a efetuar:
\begin{itemize}
	\item Assento ou Certidão de Nascimento;
	\item Cartão de Cidadão;
	\item Passe de transportes públicos;
	\item Carta de Condução;
	\item Cartão de funcionário;
	\item Certificado de Matrícula (DUA);
	\item \ldots{} .
\end{itemize}

\subsection{Considerações prévias}

As considerações prévias são um conjunto de notas importantes a ter em
conta no início e durante todos os processos, ainda que não obrigatórias
ou estritamente necessárias, irão dar uma ajuda no controlo e sucesso de
cada processo. Por outro lado serve também como forma de acautelar
situações imprevisíveis ou aspetos que possam estar fora do controlo do
iniciante do processo. \\

\subsubsection{Envio de mensagens de correio eletrónico}

\textbf{Palavras-chave}: correio eletrónico, avisos, recibos, leitura, entrega, confirmações. \\
\\
Ao enviar quaisquer mensagens de correio eletrónico deve idealmente ter
o cuidado de (quando possível) \textbf{pedir um recibo de entrega e um
de leitura} de forma a poder ter maior controlo e informação sobre o
estado de uma mensagem. É importante notar que alguns serviços tais como
o \emph{Gmail} (correio eletrónico da \emph{Google}) não permitem esse
tipo de pedido através da aplicação de navegador internet. No entanto é
possível pedir o recibo de entrega e de leitura se utilizar uma
aplicação ``clássica'' de correio eletrónico num computador, como por
exemplo o ``Outlook clássico''. Poderá pesquisar na internet como é que
deve proceder para ativar ambos os recibos de acordo com a sua situação
e configuração específica. \\
\\
\textbf{Ligações de referência}:
\begin{itemize}
	\item Microsoft, adicionar e pedir recibos de leitura e notificação de entrega, \url{https://support.microsoft.com/pt-pt/office/adicionar-e-pedir-recibos-de-leitura-e-notifica\%C3\%A7\%C3\%B5es-de-entrega-no-outlook-a34bf70a-4c2c-4461-b2a1-12e4a7a92141}
	\item Google, peça ou devolva um recibo de leitura, \url{https://support.google.com/mail/answer/9413651?hl=pt}
\end{itemize}

\subsubsection{Inconsistências durante os processos de identificação essencial}

\textbf{Palavras-chave}: cartão de cidadão, assento de nascimento, pedidos, erros. \\
\\
\textbf{Nota}: Entende-se por identificação essencial os processos de
alteração da certidão ou assento de nascimento e o pedido de novo Cartão
de Cidadão. \\
\\
Durante o processo de obtenção de um novo Cartão de Cidadão podem
verificar-se inconsistências, especialmente nos \textbf{processos
	``automáticos''} despoletados por este até que o Cartão de Cidadão e
Chave Móvel Digital estejam devidamente ativos, portanto, devem ser
consideradas finalizadas apenas as alterações dos processos
\textbf{verificadas após a ativação do Cartão de Cidadão} e entrada com
a Chave Móvel Digital em cada portal (preferível para maior comodidade).

\subsubsection{Dificuldades genéricas na alteração de dados}

\textbf{Palavras-chave}: ajuda, pedidos, mudança, alteração, alternativas. \\
\\
Em caso de dúvida ou dificuldade nas alterações, é preferível que se
desloque-se a um \textbf{estabelecimento físico}, contacte o
\textbf{suporte ou apoio ao cliente} do respectivo serviço, empresa ou
instituição ou \textbf{envie uma mensagem de correio eletrónico} de modo
a obter mais informações acerca de qual será a melhor opção.\\
\\
Na eventualidade de não ser possível através de apoio telefónico,
correio eletrónico ou balcão de atendimento da empresa ou instituição em
questão, deverá assim ser \textbf{enviada uma carta registada com aviso
	de receção} para a morada da empresa ou instituição com pedido de
retificação dos seus dados pessoais e os anexos necessários para a sua
validação, tal como uma cópia do documento com marca de água adequada se
possível.\\
\\
Em último recurso, a retificação dos seus dados só poderá ser resolvida
\textbf{judicialmente}, isto caso nenhum dos pedidos previamente
apresentados através dos métodos anteriormente referidos seja
respeitado. Todavia, poderá sempre desistir do processo de alteração ou
tentar novamente noutro momento que seja mais oportuno. \\
\\
\textbf{Recursos úteis}:
\begin{itemize}
	\item Modelo de carta - Pedido de alteração/retificação de dados pessoais;
	\item Suportes jurídicos, diversos em especial o RGPD (2016/679 de 27 de abril de 2016) e execução nacional (Lei n.º 58/2019, de 8 de agosto).
\end{itemize}

\newpage

\subsubsection{Casos específicos de dificuldade na alteração de dados}

\textbf{Palavras-chave}: rgpd, proteção de dados, questões jurídicas, alterações, pedidos. \\
\\
Em certos casos de menor importância onde o seu acesso, perfil ou conta
existente em portais, aplicações e lojas será mais difícil ou até mesmo
impossível de alterar ou retificar, poderá fazer mais sentido recriar a
conta eliminando-a e criando outra dado a rigidez ou falta de
flexibilidade administrativa desses sítios. Poderá no entanto sempre
invocar os seus direitos ao abrigo do Regulamento Geral de Proteção de
Dados (RGPD) com o pedido de retificação de dados, ainda que
possivelmente seja mais rápida a utilização de processos já bem
estabelecidos tais como eliminação e criação de conta. \\
\\
Podem existir sistemas ou suportes de dados nos quais a alteração possa
ser dificultada ou até mesmo praticamente impossível, aponta-se aqui por
exemplo dados históricos em arquivos ou suporte de papel, e/ou outros
suportes duradouros e de difícil acesso. Nesses casos será complicado ou
até mesmo impossível proceder à alteração dos dados. \\
\\
\textbf{Recursos úteis}:
\begin{itemize}
	\item Modelo de carta - Pedido de alteração/retificação de dados pessoais;
	\item Suportes jurídicos, diversos em especial o RGPD (2016/679 de 27 de abril de 2016) e execução nacional (Lei n.º 58/2019, de 8 de agosto).
\end{itemize}

\subsubsection{\texorpdfstring{Utilizar a aplicação móvel \emph{gov.pt}}{Utilizar a aplicação móvel gov.pt}}

\textbf{Palavras-chave}: app, identificação, cc, cartão de cidadão. \\
\\
Ao longo dos vários processos poderá ser útil obter e configurar a
aplicação móvel \emph{gov.pt}, esta aplicação permite centralizar os
seus documentos de identificação bem como exportá-los, apresentando-se
como uma alternativa e um complemento útil. Ao autenticar-se com a chave
móvel a aplicação permite que aceda com maior facilidade aos vários
serviços centralizando também os códigos de acesso únicos que costumam
ser gerados no momento de início de sessão. Esta aplicação é
particularmente útil para aquelas pessoas que têm dificuldades na
organização e lembrarem-se onde deixam as coisas, os documentos uma vez
na aplicação têm a mesma validade legal e podem ser utilizados no
dia-a-dia sem agravar o risco de perda desses documentos. \\
\\
\textbf{Documento de referência}:
\begin{itemize}
	\item Anexos gerais, Anexo D, Manual da aplicação móvel \emph{gov.pt}
\end{itemize}
\leavevmode\\
\textbf{Ligações de referência}:
\begin{itemize}
	\item AMA, página da aplicação \emph{gov.pt}, \url{https://www.ama.gov.pt/web/agencia-para-a-modernizacao-administrativa/gov.pt}
	\item gov.pt, adicionar documentos de identificação na aplicação \emph{gov.pt}, \url{https://www.gov.pt/servicos/adicionar-documentos-de-identificacao-na-app-id-gov-pt}
\end{itemize}

\subsubsection{\texorpdfstring{Exportação da identificação através da aplicação móvel \emph{gov.pt}}{Exportação da identificação através da aplicação móvel gov.pt}}

\textbf{Palavras-chave}: cópia de documentos, cópia cc, cartão de cidadão, certificação. \\
\\
A primeira exportação do Cartão de Cidadão através da aplicação móvel
\emph{gov.pt} após o pedido de novo Cartão de Cidadão não irá conter o
seu número de identificação fiscal (NIF), para suprir essa lacuna,
poderá solicitar uma certidão de morada fiscal no portal da Autoridade
Tributária. \\
\\
Após a receção e ativação do seu novo Cartão de Cidadão, será então
possível exportar um documento em formato \emph{PDF} assinado com o
número de identificação fiscal (NIF) utilizando a aplicação móvel
\emph{gov.pt} ou a aplicação para ambiente de trabalho
\emph{Autenticação Gov}. \\
\\
\textbf{Documento de referência}:
\begin{itemize}
	\item Anexos gerais, Anexo D, Manual da aplicação móvel \emph{gov.pt}
\end{itemize}

\subsubsection{Processos para cidadãos estrangeiros}

\textbf{Palavras-chave}: estrangeiros, nacionalidade, apátridas, identidade. \\
\\
Todos os cidadãos estrangeiros e sem nacionalidade portuguesa terão de
obrigatoriamente seguir a legislação em vigor no seu país de origem,
logo os processos iniciais de identificação essencial não serão
aplicáveis. Por vezes a execução desse processo poderá ser inviável,
pelas mais diversas razões legais ou processuais, em alternativa poderá
optar pela invocação da lei que lhe confere o direito à autodeterminação
e expressão da identidade de género (Lei n.º 38/2018 de 7 de agosto),
nos registos de identificação das entidades terceiras. Há no entanto uma
limitação relevante, a invocação do direito à autodeterminação poderá
ter uma baixa taxa de sucesso devido à ausência de suportes informáticos
adequados à acomodação e integração dos seus dados adicionais, poderá em
muitos casos ser mesmo inviável ou impossível.

\subsubsection{Digitalização de documentos}

\textbf{Palavras-chave}: digitalizações, scan, segurança. \\
\\
Em certas circunstâncias é importante garantir a qualidade dos
documentos digitalizados que poderá vir a precisar de enviar no
seguimento de alguns processos. É preferível e sempre que possível deve
optar por digitalizar os documentos utilizando um equipamento
digitalizador adequado tal como uma impressora multifunções, nesta
deverá escolher a qualidade máxima ou uma qualidade superior a 300 PPI.
Caso não saiba alterar essas definições faça apenas uma digitalização
normal. Em alternativa e caso não tenha acesso a uma digitalizadora
adequada poderá utilizar uma aplicação móvel para ``digitalização'' de
documentos usando a câmara do seu equipamento, ainda que esta deva ser
uma opção de recurso. \\
\\
Aconselha-se no âmbito da segurança documental, que sempre que
digitalizar um documento de identificação, antes de o enviar, tirar uma
cópia e traçar essa mesma cópia com identificação do propósito da cópia
do documento e só aí voltar a digitalizar e enviar de modo a combater a
potencial utilização indevida de documentos de identificação. \\
\\
\textbf{Ligação de referência}:
\begin{itemize}
	\item WikiHow, como digitalizar documentos, \url{https://pt.wikihow.com/Digitalizar-Documentos}
\end{itemize}

\subsubsection{Como traçar um documento de identificação}

\textbf{Palavras-chave}: digitalização segura, segurança, identidade. \\
\\
Para traçar um documento de identificação ou outro de particular
sensibilidade, para fins de segurança documental e individual deve
efetuar os seguintes passos:
\begin{itemize}
	\item Fazer uma cópia do documento;
	\item Efetuar dois traços diagonais por cima do documento;
	\item Nas linhas diagonais deve escrever o propósito da cópia;
	\item Poderá opcionalmente assinar;
	\item Se o pedido foi em formato digital deve digitalizar a cópia traçada e	enviar.
\end{itemize}
\leavevmode\\
Ao traçar os documentos antes de os enviar estará a diminuir a
possibilidade de os seus documentos serem utilizados de forma
fraudulenta ou ilegal. É importante fazer aquilo que estiver ao seu
alcance para combater o roubo de identidade e a fraude.

\subsubsection{Esclarecimentos das menções a aplicações}

\textbf{Palavras-chave}: apps, referências, portais, aplicações, telemóvel. \\
\\
Para melhor entendimento do utilizador do roteiro são feitas menções ao
longo do documento a aplicações e portais e é importante garantir o
entendimento e a distinção de cada uma delas:
\begin{itemize}
	\item \textbf{Aplicação móvel \emph{gov.pt}}, é uma aplicação para telemóvel que permite exportar os seus dados de identificação e auxiliar na autenticação em diversos locais utilizando a Chave Móvel Digital.
	\item \textbf{Portal \emph{Autenticação Gov}}, é um portal acessível através do seu navegador utilizando um leitor de catões e o seu Cartão de Cidadão ou Chave Móvel Digital.
	\item \textbf{Aplicação \emph{Autenticação Gov}}, é uma aplicação de ambiente de trabalho que permite assinar documentos, verificar os seus dados de identificação e inclusivamente exportá-los.
\end{itemize}
\leavevmode\\
\textbf{Ligações de referência}:
\begin{itemize}
	\item Autenticação gov.pt, aplicação móvel gov.pt, \url{https://www.autenticacao.gov.pt/aplicacao/autenticacao-gov-movel}
	\item Autenticação gov.pt, aplicação Autenticação Gov, \url{https://www.autenticacao.gov.pt/web/guest/cc-aplicacao}
	\item Autenticação gov.pt, portal Autenticação Gov, \url{https://www.autenticacao.gov.pt/}
\end{itemize}

\subsubsection{Escolha do(s) nome(s) próprio(s)}

\textbf{Palavras-chave}: escolher um nome, questões jurídicas, lista de nomes. \\
\\
É importante notar que a escolha do(s) nome(s) próprio(s) tem de
obrigatoriamente obedecer à lista de nomes aprovados pelo Instituto dos
Registos e Notariado (IRN), anexo c do roteiro, isto significa que só
poderá pedir o registo de um nome que exista na lista em anexo. Além
disso é importante que este cumpra com as regras de composição do nome
em vigor. Se o nome preferido não constar na lista de nomes aprovados
poderá pedir um parecer onomástico com um custo de 75,00€ (setenta e
cinco euros). Noutras circunstâncias específicas e que saiam do contexto
de aprovação regular é preferível que consulte um especialista,
preferencialmente um advogado ou solicitador. \\
\\
\textbf{Documento de referência}:
\begin{itemize}
	\item Anexos gerais, Anexo C - Lista de nomes próprios aprovados pelo IRN.
\end{itemize}
\leavevmode\\
\textbf{Ligações de referência}:
\begin{itemize}
	\item IRN, composição do nome, \url{https://irn.justica.gov.pt/Servicos/Cidadao/Nascimento/Composicao-do-nome}
	\item IRN, lista dos nomes próprios, \url{https://irn.justica.gov.pt/Portals/33/Regras\%20Nome\%20Proprio/Lista\%20Nomes\%20Pr\%C3\%B3prios.pdf?ver=WNDmmwiSO3uacofjmNoxEQ\%3D\%3D}
	\item IRN, custos dos serviços, \url{https://irn.justica.gov.pt/Custos-dos-servicos}
\end{itemize}

\subsubsection{Conservação dos documentos anteriores}

É importante conservar todos os documentos relativos ao seu processo de
mudança de nome e sexo, nomeadamente os seus documentos de
identificação, isto porque numa situação onde não seja possível
identificar a pessoa pelo novo documento, por lapso documental ou
problema de sistema, o seu anterior documento dá-lhe uma forma
alternativa de garantir a sua identificação. Aponta-se assim um exemplo
prático, numa situação onde o caderno eleitoral não está atualizado
devidamente, o facto de ter o seu anterior documento irá permitir que
possa exercer o seu direito ao voto, de outra forma o direito ao voto
poderia ser vedado. 

\newpage

\subsubsection{Aplicações úteis}

Adiante estão listadas as aplicações lhe podem ser úteis neste mudança e
para o futuro:
\begin{itemize}
	\item \textbf{Aplicação móvel \emph{gov.pt}}, para os seus documentos de identificação e autenticação em vários serviços;
	\item \textbf{Aplicação móvel \emph{SIGA}}, mais cómodo para obter senha nos serviços que necessita tais como o Cartão de Cidadão ou Passaporte;
	\item \textbf{Aplicação móvel Segurança Social direta}, útil para rever os seus dados e informações junto da Segurança Social no futuro;
	\item \textbf{Aplicação móvel \emph{SNS24}}, útil para rever as as suas informações mas também ter acesso às suas consultas, exames e prescrições.
\end{itemize}
\leavevmode\\
\textbf{Ligações de referência}:
\begin{itemize}
	\item Aplicação móvel para sistema Android:
	\begin{itemize}
		\item \emph{gov.pt}, \url{https://play.google.com/store/apps/details?id=id.gov.pt\&hl=pt\_PT}
		\item \emph{SIGA}, \url{https://play.google.com/store/apps/details?id=pt.segsocial.iies.sigaapp.prod\&hl=pt\_PT}
		\item Segurança Social, \url{https://play.google.com/store/apps/details?id=pt.segsocial.mobile.segurancasocial\&hl=pt\_PT}
		\item \emph{SNS24}, \url{https://play.google.com/store/apps/details?id=pt.minsaude.spms.ces\&hl=pt\_PT}
	\end{itemize}
	\item Aplicação móvel para sistema iOS:
	\begin{itemize}
		\item \emph{gov.pt}, \url{https://apps.apple.com/pt/app/gov-pt/id1384884826}
		\item \emph{SIGA}, \url{https://apps.apple.com/pt/app/sigaapp/id1127868225}
		\item Segurança Social, \url{https://apps.apple.com/pt/app/seguran\%C3\%A7a-social/id1469920521}
		\item \emph{SNS24}, \url{https://apps.apple.com/pt/app/sns-24/id1192353854}
	\end{itemize}
\end{itemize}

\subsubsection{Consideração dos custos financeiros}

Ao longo do roteiro são apontados custos financeiros a cargo do
utilizador, esses apenas representam os encargos direitos na execução
dos processos, esses custos diretos não englobam todos os outros
potenciais custos, considerados como indiretos, tais como, impressões,
deslocações, consultorias externas, e outros demais serviços, itens ou
elementos de suporte administrativo e prático aos vários processos. Isto
significa que a execução de um determinado processo poderá ser
significativamente mais dispendiosa do que o antecipado pois o roteiro
apenas prevê o custo e processo genérico. A sua circunstância poderá ser
significativamente diferente das condições ideais previstas agravando
assim os custos totais de execução englobando todos os custos.