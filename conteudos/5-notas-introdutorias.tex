%---------------------
% NOTAS INTRODUTÓRIAS
%---------------------

\newpage

\section{Notas Introdutórias}

\subsection{Introdução}

Este roteiro nasce da necessidade de \textbf{identificar os passos} a
tomar no \textbf{processo de alteração de nome e sexo} bem como as suas
ramificações numa base individual e personalizada. Este roteiro foi
elaborado a pensar nas pessoas que estão a \textbf{ponderar} efetuar a
mudança de nome e sexo bem como aquelas que \textbf{já o fizeram}, ou
até mesmo outras que já o fizeram mas \textbf{não completaram todas as 
	alterações} por \textbf{desconhecimento} ou até mesmo
\textbf{impossibilidade}. \\
\\
O roteiro é abrangente e pretende atender a um número alargado de
pessoas, ainda que o \textbf{principal público-alvo seja maiores de
	idade}, com algumas competências básicas e com quotidianos profundamente
estabelecidos, onde uma mudança de nome e sexo pode implicar uma
\textbf{sequência extensa de processos burocráticos} algo demorados e
inconvenientes. Ainda que nem todo o roteiro seja necessariamente
acessível a todos os tipos de público certamente poderá ajudar boa parte
deles. Existem alguns pressupostos sobre os quais o roteiro foi
estabelecido onde poderá existir lacunas informacionais para alguns tipo
público. \\
\\
Pretende-se desta forma criar um \textbf{suporte de relativa
	fiabilidade} que com base na experiência e aconselhamento prévio irá
facilitar e responder a muitas das \textbf{questões} que a pessoa que
pretenda mudar de nome e sexo poderá ter. É importante salientar que o
propósito central deste roteiro é \textbf{evidenciar os processos
	administrativos}, algumas questões ligadas ao direito, e
\textbf{respostas práticas} ao inverso de outros guias cujo foco é mais
universalista e incidente sobre as questões sociais, psicológicas,
médicas ou até mesmo políticas. \\
\\
Ainda que o roteiro tenha como objetivo ser bastante abrangente haverá
situações e circunstâncias específicas difíceis de cobrir numa base
comum, isto é, sem ser demasiado detalhista. Nesses cenários menos
incomuns ou invulgares a pessoa deverá recorrer ou fazer uso de recursos
alternativos para melhor suprir as suas necessidades específicas e
individuais. \textbf{Este roteiro não é um substituto de acompanhamento
	profissional especializado}. \\
\\
Este roteiro foi escrito de acordo com o \textbf{novo acordo
	ortográfico}.
	
\subsection{Limitação de responsabilidade}

É importante salientar que \textbf{poderão existir erros}, gralhas,
falhas de descrição processual, ou até mesmo falhas nos processos quer
por atualização desses processos quer por \textbf{simples lapso
	documental}. Dado o contexto de criação do roteiro, é impossível que
este acompanhe todas as atualizações futuras aos processos. Este roteiro
portanto \textbf{não é representativo de aconselhamento especializado}
no âmbito das questões do direito e o seu \textbf{conteúdo é apenas
	informativo}. Sendo o conteúdo apenas de caráter puramente
informacional, não poderá representar aconselhamento jurídico,
significando assim que não são dadas quaisquer garantias de qualquer
informação ao utilizador do roteiro. Fora do âmbito do roteiro e em caso
de dúvidas que possa vir a ter, \textbf{deverá recorrer a recursos de
	acompanhamento especializado} tais como advogados ou solicitadores para
esses efeitos. Outras dúvidas ou problemas que possa encontrar relativo
a outro foro deve então obter apoio externo nas respetiva área de
competência, como por exemplo, psicólogos para questões sociais.

\newpage

\subsection{Anexos}

Para que o roteiro ficasse o mais completo possível foram anexadas todas
as \textbf{bases}, \textbf{modelos} e \textbf{documentação relevante}. O
conjunto de documentos cobre todas as bases e fundamentos legais para as
mudanças que irá fazer, modelos com os quais poderá com maior facilidade
efetuar as alterações junto de empresas, entidades ou organizações. Os
anexos são fornecidos numa base de facilidade burocrática e
administrativa, não representando assim uma fonte absoluta de verdade ou
atualidade dos factos, poderão em certas circunstâncias existir versões
mais recentes dos documentos anexados. Cabe ao utilizador do roteiro
verificar a veracidade, atualidade e aplicabilidade dos anexos à sua
circunstância específica. \\
\\
\textbf{Lista de anexos} \\[4pt]
\\
\textbf{Anexos gerais}:
\begin{itemize}
	\item Anexo A - Destaques jurídicos de especial interesse, temática militar;
	\item Anexo B - Destaques jurídicos de especial interesse, temática de dados pessoais e legislação comunitária;
	\item Anexo C - Lista de nomes próprios aprovados pelo Instituto dos Registos e Notariado (IRN);
	\item Anexo D - Manual da aplicação móvel \emph{gov.pt};
	\item Anexo E - Condições Gerais de Emissão e Utilização do Cartão navegante;
	\item Anexo F - O documento eletrónico: suporte e formato, Ordem dos Advogados.
\end{itemize}
\leavevmode\\
\textbf{Formulários}:
\begin{itemize}
	\item Anexo A - Requerimento para mudança da menção do sexo e nome próprio, maiores de idade;
	\item Anexo B - Requerimento para mudança da menção do sexo e nome próprio, menores de idade;
	\item Anexo C - Requerimento de Registo Automóvel;
	\item Anexo D - Requisição de Passe Navegante (TML);
	\item Anexo E - Declaração de objeção de consciência perante o serviço militar;
	\item Anexo F - Declaração abonatória (Relativo ao estatuto de objetor de consciência);
	\item Anexo G - Formulário para a atualização de dados e consentimento para o tratamento de dados pessoais, cartão navegante;
	\item Anexo H - Formulário de exercício de direitos dos titulares de dados pessoais, cartão navegante.
\end{itemize}
\leavevmode\\

\newpage

\textbf{Modelos}:
\begin{itemize}
	\item Modelo de carta - Pedido de alteração/retificação de dados pessoais;
	\item Modelo de carta - Pedido de 2ª (segunda) via do Certificado de Habilitações;
	\item Modelo de carta - Pedido de 2ª (segunda) via de Relatório Médico ou Avaliação Psicológica;
	\item Modelo de carta - Reconhecimento de nome social escolar;
	\item Modelo de mensagem de correio eletrónico para o Dia da Defesa Nacional/Balcão Único da Defesa (DDN/BUD);
	\item Modelo de formulário de renúncia ao estatuto de objetor de consciência.
\end{itemize}
\leavevmode\\
\textbf{Suportes jurídicos}:
\begin{itemize}
	\item Carta dos direitos fundamentais da União Europeia n.º 2016/C 202/02;
	\item Código de Processo Civil (CPC), Lei n.º 41/2013, de 26 de junho;
	\item Código do Registo Civil (CRC), Decreto-Lei n.º 131/95, de 6 de junho;
	\item Constituição da República Portuguesa (CRP), Decreto de Aprovação da Constituição, de 10 de abril;
	\item Execução nacional do RGPD, Lei n.º 58/2019, de 8 de agosto;
	\item Direito à autodeterminação da identidade de género e expressão de género e à proteção das características sexuais de cada pessoa, Lei n.º 38/2018 de 7 de agosto;
	\item Lei do Serviço Militar (LSM), Lei n.º 174/99, de 21 de setembro;
	\item Procedimento de mudança de sexo, Lei n.º 7/2011, de 15 de março;
	\item Regulamento da Lei do Serviço Militar, Decreto-Lei n.º 289/2000, de 14 de novembro;
	\item Regulamento Geral da Proteção de Dados, Regulamento da União Europeia n.º 2016/679 de 27 de abril de 2016;
	\item Sistema alternativo e voluntário de autenticação dos cidadãos (\ldots) denominado Chave Móvel Digital, Lei n.º 37/2014, de 26 de junho;
	\item Alteração às Leis (\ldots) que cria o Cartão de Cidadão (\ldots) Chave Móvel Digital, e (\ldots) recenseamento eleitoral (\ldots), Lei n.º19-A/2024, de 7 de fevereiro;
	\item Regulamento eIDAS, identificação eletrónica, Regulamento da União Europeia 910/2014 de 23 de julho de 2014;
	\item Execução nacional do eIDAS, Decreto-Lei n.º 12/2021, de 9 de fevereiro.
\end{itemize}

\newpage

\subsection{Inteligência Artificial}
 
O roteiro foi composto recorrendo \textbf{ocasionalmente} a ferramentas
de inteligência artificial de modo a facilitar a composição e
organização dos conteúdos, no entanto, todos os textos foram finalizados
e revistos manualmente dadas as limitações dessas ferramentas. \\
\\
Foi também criado um GPT personalizado com base neste roteiro e todos os
seus anexos, tais como modelos e diplomas legislativos e regulamentares.
Esta poderá ser uma ferramenta útil para navegar os processos numa base
mais individual e específica. Há que apontar que estas ferramentas por
vezes não são 100\% fiáveis portanto deve confirmar as informações dadas
por esta. \\
\\
\textbf{Ligação de referência}:
\begin{itemize}
	\item ChatGPT, guia para a mudança de nome e sexo, \url{https://chatgpt.com/g/g-68504ffb04848191a5df58246b201561-guia-para-a-mudanca-de-nome-e-sexo}
\end{itemize}

\subsection{Afiliações e conflitos de interesse}

Nenhuma das referências a marcas, nomes, tecnologias e outras menções
específicas sujeitas ou relacionadas com direitos ou propriedade
intelectual publicamente reconhecíveis utilizadas neste roteiro
representam uma qualquer afiliação, incentivo, promoção ou de qualquer
outro modo um apelo à ação ou consumo de uma determinada referência, não
existindo assim nenhum motivo financeiro ou qualquer outro para a
influência do utilizador deste roteiro que não o auxílio na retificação
dos seus dados pessoais. Todas e quaisquer recomendações ou sugestões
são apenas representativas de um método ou forma de ação baseada numa ou
mais experiências individuais relatadas. O utilizador do roteiro é
responsável pelo seguimento ou utilização das referências externas
fornecidas sendo que os autores e distribuidores do roteiro não são
responsáveis pelos conteúdos de destino nas referências mencionadas. É
autodeclarada a inexistência de conflitos de interesse no prosseguimento
dos fins do roteiro.

\subsection{Contexto jurisdicional}

O roteiro foi pensado no contexto jurisdicional Português, sendo que a
sua aplicabilidade é circunscrita a Portugal, à exceção dos itens
explicitamente mencionados onde a aplicabilidade pode estender-se à UE.
A utilidade fora do contexto jurisdicional previamente entendido é
limitada e o roteiro não deve ser considerado nesses casos pela sua
razão inadequada ou incorreta.

\subsection{Agradecimentos}

Queria agradecer todo o apoio que recebi e a todos os que tornaram este
roteiro possível. É com uma grande felicidade que dou o meu especial
agradecimento à Rute que tudo fez para me apoiar neste caminho. Agradeço
também em ponto grande ao Bruno, ao João, e ao Emanuel, pessoas pelas
quais tenho elevada consideração, bem como à MUSSOC por ser um pilar de
apoio da comunidade com neutralidade e integridade. Agradeço também a
todos os que de alguma forma contribuíram para a elaboração do roteiro.
Sem vós este roteiro e a sua inspiração não seria possível, muito
obrigada.

\newpage

\subsection{Nota da autora}

Este roteiro inspira-se numa jornada difícil, que se aparentou fácil
inicialmente, onde não sabia para onde virar-me com tanta coisa que era
preciso alterar e verificar. Senti-me completamente perdida, como é que
iria tratar dos mais de 50 (cinquenta) itens, locais e demais papeladas.
A minha pesquisa inicial deu pouco frutos, retornou apenas o básico,
aquilo que é óbvio e estritamente necessário mudar mas esquece-se
completamente do resto, algumas delas importantes mas escondidas. A
minha vida é feita de mais que um Cartão de Cidadão, é conduzir, viajar,
trabalhar, é o quer que seja que eu queira. O facto? Nenhum guia
mostrou-me a realidade disto e o lado prático do que é mesmo mudar algo
tão ``simples'' como o nome e o sexo na identificação. Isto é um guia
completo, estruturado q.b. de tudo aquilo que é possível fazer para
tratar de todos os registos que precisa ou precisará de alterar. É um
contributo simbólico para as pessoas que estão a embarcar nesta jornada
saberem ao que vêm e ter um guia com um bom grau de solidez e confiança.
Neste caminho consultei advogados, psicólogos e amigos que de alguma
forma tentaram ajudar a completar e organizar a informação. Criei
modelos para facilitar os processos e as comunicações. É importante
notar que sou apenas profissional das tecnologias de informação e
curiosa sobre temas do meu especial interesse. Li de fio a pavio peças
de legislação, regulamentos e uns tantos outros para construir e
executar algo que acho que vale a pena.

\subsection{Edição}

Este roteiro está na sua primeira 1ª (primeira) edição, esta é
correspondente e consistente com as informações reunidas e atualizadas
entre janeiro e junho de 2025. A versão é apenas um dos indicadores da
validade e fiabilidade da informação apresentada. \\
\\
\textbf{Histórico de edições} \\
\\
\textbf{2025-1}: Primeira edição: a primeira compilação extensa e detalhada do
processo de mudança de nome e sexo bem como as suas ramificações,
inclusão de anexos úteis, suportes informacionais e práticos.

\subsection{Sugestões, opiniões ou outras comunicações}

No decorrer do roteiro poderá notar a existência de erros, lacunas
informacionais ou imprecisões processuais. Na eventualidade de querer
contribuir com novas informações, correções de erros e imprecisões
pede-se que envie as suas sugestões utilizando como base o modelo de
mensagem de correio eletrónico (que poderá encontrar nos anexos) para o
endereço de correio eletrónico roteiromudanca@imariana.com. Poderá
também enviar as suas opiniões e outras comunicações relativas a este
documento que achar pertinentes para o mesmo endereço. \\
\\
\textbf{Ligação de referência}:
\begin{itemize}
	\item Tally.so, formulário para comunicações relativas ao roteiro, enviar a sua opinião, \url{https://tally.so/r/3lXd1p}
\end{itemize}