%-------
% FECHO
%-------

\section{Fecho}

\subsection{Notas finais}

Depois de finalizar aquilo que considero ser uma espécie de pesadelo
burocrático espero que a utilidade do roteiro perdure no tempo como um
pequeno farol no mar da mudança. Fecha-se assim um capítulo de esforço e
dedicação. Um desejo de esperança e alguma sorte aos utilizadores do
roteiro bem como os votos de sucesso e felicidades.

\subsection{Lista de abreviaturas}

Para maior facilidade na leitura e interpretação do roteiro pode
encontrar aqui todas as abreviaturas em menção ao longo do texto. \\
\\
\begin{itemize}
	\item AML, Área Metropolitana de Lisboa, inclui a Grande Lisboa e Margem Sul.
	\item ANSR, Autoridade Nacional de Segurança Rodoviária, serviço central da administração direta do Estado, foca-se nas matérias de segurança rodoviárias e aplicação do direito contraordenacional rodoviário.
	\item AT, Autoridade Tributária, organismo do Ministério das Finanças responsável por serviços relacionados com impostos e alfândega.
	\item BEP, Bolsa de Emprego Público, é uma base de informação que promove, simplifica e agiliza os processos de recrutamento de recursos humanos da Administração Pública.
	\item BUD, Balcão Único da Defesa, balcão central de acesso público aos serviços relacionados com a defesa nacional.
	\item CC, Cartão de Cidadão, principal documento de identificação.
	\item CGD, Caixa Geral de Depósitos, o maior banco em Portugal detido pelo Estado Português.
	\item CMD, Chave Móvel Digital, meio de autenticação e assinatura digital certificado pelo Estado Português associado a um número de telemóvel.
	\item CNPD, Comissão Nacional de Proteção de Dados, é autoridade nacional de controlo de dados pessoais dotada de independência administrativa.
	\item CPC, Código do Processo Civil, lei de regulamentação do trânsito do processo judicial civil.
	\item CRC, Código do Registo Civil, lei de regulamentação do registo civil.
	\item CRP, Constituição da República Portuguesa
	\item CV, Curriculum Vitae, currículo ou documento histórico que traça as experiências educacionais e profissionais.
	\item DAV, Diretiva Antecipada de Vontade, conhecido também por testamento vital, é o documento onde uma pessoa manifesta a sua vontade sobres os cuidados de saúde que deseja receber.
	\item DDN, Dia da Defesa Nacional, dia de sensibilização para a Defesa Nacional e divulgação das Forças Armadas, substitui o serviço militar obrigatório anterior.
	\item DUA, Documento Único Automóvel ou Certificado de Matrícula, é o documento de circulação dos veículos matriculados em Portugal.
	\item \emph{EUIPO}, \emph{European Union Intellectual Property Office}, é o gabinete de propriedade intelectual e industrial responsável pela gestão dos registos a nível europeu.
	\item \emph{GPS}, \emph{Global Positioning System}, é um sistema de navegação por satélite que fornece ao utilizador a sua localização e horário independentemente das condições atmosféricas.
	\item ID, refere-se simplesmente à palavra identificação.
	\item IEFP, Instituto do Emprego e Formação Profissional, é um instituto público com missão de promover o emprego de qualidade, a reintegração e também a formação profissional.
	\item IMT, Instituto da Mobilidade e dos Transportes, é um instituto público parte da administração indireta do Estado Português.
	\item INPI, Instituto Nacional da Propriedade Industrial, é um instituto público com a finalidade de proteger e promover a propriedade industrial.
	\item IPDJ, Instituto Português do Desporto e Juventude, é um organismo do Estado Português promovendo a inovação, o empreendedorismo, o desporto e temáticas ligadas à juventude.
	\item IRN, Instituto dos Registos e Notariado, é um organismo público cuja com a finalidade de executar e acompanhar as políticas e serviços de registo.
	\item \emph{JPEG}, \emph{Joint Photographic Experts Group}, é um formato de imagem digital que utiliza compressão com perdas de modo a reduzir o tamanho dos arquivos.
	\item LSM, Lei do Serviço Militar, lei que estabelece os termos do serviço militar.
	\item MAI, Ministério da Administração Interna, é um organismo responsável pelas políticas de segurança pública, de proteção e socorro, de imigração e asilo, bem como da prevenção e segurança rodoviária e administração eleitoral.
	\item MDN, Ministério da Defesa Nacional, organismo do Estado Português encarregue da preparação e execução da política de Defesa Nacional, fiscalização e garantia da administração das Forças Armadas.
	\item NIM, Número de Identificação Militar, é o número atribuído na presença e cumprimento do Dia da Defesa Nacional.
	\item \emph{PDF}, \emph{Portable Document Format}, é um formato de arquivo digital criado pela \emph{Adobe} para representar documentos fiavelmente independentemente da plataforma. É um formato tipicamente aceite em todos os estabelecimentos pela versatilidade e confiabilidade.
	\item PEP, Passaporte Eletrónico Português, documento indispensável para propósitos de viagem e identificação.
	\item PIN, \emph{Personal Identification Number}, geralmente refere-se a um código pessoal de acesso, confidencial e não deve ser partilhados com terceiros.
	\item PPI, \emph{Pixels per inch}, refere-se à resolução de uma imagem ou documento, quanto maior o seu valor maior será o detalhe e o tamanho do ficheiro.
	\item PT, Portugal, Português ou referência ao país, nacionalidade ou origem Portuguesa.
	\item RCBE, Registo Central do Beneficiário Efetivo, é o registo centralizado das pessoas que controlam uma determinada entidade jurídica.
	\item RGPD, Regulamento Geral sobre a Proteção de Dados, é o regulamento europeu que visa garantir a proteção dos dados e o direitos dos seus titulares.
	\item SGMAI, Secretaria-Geral do Ministério da Administração Interna, tem como objetivo o apoio técnico e administrativo aos gabinetes do Governo integrados no Ministério da Administração Interna.
	\item SS, Segurança Social ou Instituto da Segurança Social é um organismo dotado de autonomia administrativa, financeira e patrimonial encarregue de assegurar os direitos básicos de bem-estar, coesão social e oportunidades iguais dos cidadãos.
	\item TML, Transportes Metropolitanos de Lisboa, é a autoridade dos transportes públicos explorados na área metropolitana de Lisboa.
	\item UE, referente à União Europeia.
	\item \emph{gov.pt}, tipicamente refere-se à aplicação móvel de identificação e autenticação do Estado Português.
	\item \emph{q.b.}, equivalente a ``quanto baste''.
\end{itemize}